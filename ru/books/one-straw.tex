\documentclass[a4paper]{book}
\usepackage{fontspec}
\setmainfont[
    Ligature=TeX,
    Extension=.otf,
    BoldFont=cmunbx,
    ItalicFont=cmunti,
    BoldItalicFont=cmunbi,
]{cmunrm}
\usepackage{polyglossia}
\setmainlanguage{russian}
\begin{document}
Автор книги - Масанобу Фукуока - более сорока лет занимался выращиванием риса и
мандаринов. За это время он провел множество экспериментов по выращиванию урожаев
разными методами в разных условиях и в конце концов пришел к тому, что сейчас называют
натуральным или органическим земледелием. В данной книге Фукуока рассказывает о том,
как выращивать большие урожаи без вспашки земли, без удобрений, без борьбы с
вредителями и сорняками, с любовью и по науке знания Природы.

Введение
Глава I Посмотрите на эти поля зерновых
Возвращение в деревню
Путь к методу «ничего-не-делания»
Возвращение к источнику
Почему натуральное земледелие не получило широкого распространения?
Человечество не знает природы
Глава II Четыре принципа натурального земледелия
Культивация (вспашка, рыхление)
Удобрения
Как справиться с проблемой сорняков
Защита от «вредителей»
Культурные растения среди сорняков
Земледелие и солома
Плодовые деревья
Не убивайте естественных хищников.
Почва плодового сада
Выращивание овощей как диких растений
Можно ли отказаться от химикатов?
Ограничения научного метода
Глава III С точки зрения фермера
Простые средства для решения сложной проблемы
Плоды трудных времён
Торговля натуральными продуктами
Коммерческое земледелие потерпит поражение
Исследования для чьей пользы?
Что такое пища человека?
Милосердный конец ячменя
Просто следуй природе, и всё будет хорошо
Различные школы натурального земледелия
Глава IV Заблуждения относительно пищи
Мандала натуральной пищи
Культура питания

Жизнь на одном хлебе
Общие принципы питания
Пища и земледелие
Глава V Глупость рядится под находчивость
Кто глупец?
Я был рождён для того, чтобы ходить в детский сад
Теория относительности
Деревня без войны и мира
Революция одной соломинки

Масанобу Фукуока
Революция одной соломинки
(Введение в натуральное земледелие)

Введение
На своей ферме, расположенной по соседству с маленькой деревушкой на острове
Шикоку в Южной Японии, Масанобу Фукуока создал метод натурального ведения
фермерского хозяйства, который мог бы помочь повернуть вспять деградационные
тенденции современного земледелия. Натуральное хозяйство не нуждается ни в машинах, ни
в ядохимикатах и требует минимума прополки. М‑р Фукуока не пашет почву и не использует
компост. Он не заливает водой свои рисовые поля в течение вегетационного периода, как
принято было делать на протяжении многих веков на Востоке и во всём мире. Почва на его
полях остаётся невспаханной в течение свыше двадцати пяти лет, и всё же он получает
урожаи, сравнимые с урожаями наиболее продуктивных японских ферм. Его метод
возделывания почвы требует меньше труда, чем любой другой. Он не способствует
загрязнению среды и не требует использования ископаемого горючего.
Когда я впервые услышал рассказы о м-ре Фукуока, отнёсся к ним скептически. Разве
возможно ежегодно получать, высокие урожаи риса и озимых зерновых, просто разбрасывая
семена по поверхности невспаханного поля? Здесь должно быть скрыто что-то большее.
В течение нескольких лет я жил с группой друзей на ферме в горах к северу от Киото.
Мы применяли традиционные методы японского земледелия, выращивая рис, ячмень, рожь,
сою и различные овощи. Приезжающие посетить ферму часто говорили о работе м-ра
Фукуока. Никто из этих людей не жил достаточно долго на его ферме, чтобы изучить детали
его техники, но их разговоры возбудили моё любопытство. Всякий раз, когда в нашей работе
появлялся просвет, я уезжал в другие части страны, останавливаясь на фермах и в общинах и
принимая участие в их работе. В одну из таких поездок я нанёс визит на ферму м-ра
Фукуока. чтобы самому изучить работу этого человека.
Я не очень хорошо помню, каким я ожидал увидеть его, но после того, как я столько
слышал об этом великом учителе, я был несколько удивлён, увидев, что он носит ботинки и
рабочую одежду среднего японского фермера. Но его негустая седая борода и живая,
уверенная манера поведения придавали ему вид в высшей степени необычного человека.
В этот первый визит я оставался на ферме м-ра Фукуока в течение нескольких месяцев,
работая на полях и в цитрусовом саду. Там, в глинобитных хижинах, во время вечерних
дискуссий со студентами-работниками фермы, метод м-ра Фукуока и лежащая в его основе
философия постепенно становились понятными мне.
Сад м-ра Фукуока расположен на склонах холмов, обращённых в сторону залива
Матсуяма. Это «гора», где живут и работают его студенты. Многие из них прибыли так же,
как и я, с рюкзаком за спиной и не представляя, что их здесь ждёт. Они остаются на
несколько дней или несколько недель и затем снова исчезают, уходя с горы вниз. Но обычно
здесь есть центральная группа, состоящая из четырёх-пяти человек, которые живут здесь
около года. За прошедшие годы многие люди, мужчины и женщины, приходили сюда, чтобы
на какое-то время остаться здесь и работать. Здесь нет современных удобств, питьевую воду
приносят в вёдрах из источника, пищу готовят на открытом очаге на дровах, а по вечерам
освещают хижины свечами и керосиновыми лампами. Гора снабжает их дикими травами и
овощами. Рыбу и моллюсков можно собрать в ближайшем ручье, а морские водоросли – во
Внутреннем море за несколько миль отсюда.
Работа меняется в зависимости от погоды и сезона. Рабочий день начинается около

восьми, один час отводится на ленч (два или три часа в жару в середине лета), студенты
возвращаются с работы в хижины как раз перед заходом солнца. Помимо
сельскохозяйственных работ, здесь есть ежедневные обязанности: принести воду, наколоть
дрова, накормить кур и собрать их яйца, сварить еду, приготовить горячую ванну, ухаживать
за козами и пчёлами, ремонтировать, а иногда строить новые хижины и готовить «мисо»
(соевую пасту) и «тофу» (соевый творог).
М‑р Фукуока выделяет 10 000 иен (около 35 долларов) в месяц на расходы всей общины.
Большая часть этой суммы идёт на покупку соевого соуса, растительного масла и других
необходимых продуктов, которые непрактично производить самим в небольших количествах.
Остальные потребности студенты должны удовлетворять полностью за счёт тех культур,
которые они выращивают, ресурсов окружающей среды и свой изобретательности. М‑р
Фукуока намеренно заставляет своих студентов вести такой полупримитивный образ жизни.
Так он сам жил в течение многих лет, так как считает, что такой образ жизни развивает
интуицию, необходимую, чтобы вести фермерское хозяйство его натуральным методом.
В области Шикоку, где живёт Фукуока, рис выращивают на прибрежных равнинах, а
цитрусовые – на окружающих их холмах. Ферма м-ра Фукуока включает в себя рисовые поля
площадью 0,5 га и мандариновые сады площадью 5 га. Такая ферма западному фермеру не
покажется большой, но поскольку вся работа делается с помощью традиционных японских
ручных орудий труда, то требуется немало усилий, чтобы обрабатывать даже такую
маленькую площадь.
М‑р Фукуока работает вместе со студентами в полях и в саду, но никто точно не знает,
когда он посетит то или иное рабочее место. Он, кажется, обладает способностью
появляться в то время, когда студенты меньше всего этого ожидают. Он энергичный человек
и всегда охотно объясняет ту или иную вещь. Время от времени он собирает студентов
вместе, чтобы обсудить работу, которую они делают, иногда при этом указывая способ, с
помощью которого эта работа может быть закончена быстрее и легче. В других случаях он
рассказывает о жизненном цикле сорняка или возбудителя грибкового заболевания плодовых
деревьев, а иногда он делает отступление, чтобы вспомнить и рассказать случай из своей
фермерской практики. Помимо объяснения своей техники, м-р Фукуока учит также основам
сельскохозяйственного мастерства. Он подчёркивает важность заботливого отношения к
орудиям труда и никогда не устаёт демонстрировать их возможности.
Если новоприбывший думает, что «натуральное хозяйство» означает, что всё делается
само собой естественным путём, в то время, как он сам сидит и наблюдает, то м-р Фукуока
скоро научит его, что «натуральное хозяйство» требует громадного объёма знаний и работы.
Если понимать буквально, то единственное «натуральное» хозяйство – это охота и собирание
естественной пищи. Выращивание сельскохозяйственных растений – это следующая
культурная ступень, требующая знаний и постоянных усилий. Главная особенность метода
м-ра Фукуока заключается в том, что он ведёт своё хозяйство путём кооперации с природой,
не пытаясь покорить её или улучшить. Отсюда и название его метода – «натуральный», то
есть естественный или природный.
Многие посетители приезжают на ферму только на послеобеденное время и м-р
Фукуока терпеливо показывает им своё хозяйство. Привычная картина – видеть его бодро
поднимающимся по горной тропинке с пыхтящей позади него группой из 10–15 визитёров.
Однако не всегда здесь было так много посетителей. В течение тех лет, когда он
разрабатывал свой метод, м-р Фукуока имел немного контактов с кем-либо за пределами

свой деревни.
Молодым человеком м-р Фукуока покинул свой родной дом и поехал в Иокогаму, чтобы
стать микробиологом. Он стал специалистом по болезням растений и в течение нескольких
лет работал в лаборатории в качестве таможенного сельскохозяйственного инспектора.
Именно в это время, будучи молодым человеком 25 лет, м-р Фукуока пережил то прозрение,
которое сформировало основу для его работы, ставшей его жизненной задачей, темой этой
книги «Революция одной соломинки». Он оставил свою работу на таможне и вернулся в
родную деревню, чтобы испытать на своих собственных полях жизнеспособность своих
идей.
Основная идея пришла к нему однажды, когда он случайно проходил мимо старого поля,
заброшенного и испаханного в течение многих лет. Там он увидел здоровые ростки риса,
пробивающиеся через сплетение трав и сорняков. С этого времени он перестал затоплять
водой своё рисовое поле. Он перестал сеять рис весной и вместо этого высевал семена
осенью прямо на поверхность почвы, как они и должны были бы рассеваться естественным
путём – просто падать на поверхность почвы из зрелых метёлок. Вместо того, чтобы
уничтожать сорняки с помощью вспашки почвы, он научился контролировать их численность
путём постоянного поддержания более или менее постоянного покрова из белого клевера и
мульчирования рисовой и ячменной соломой. Убедившись, что такие условия
благоприятствуют развитию культурных растений, м-р Фукуока старался как можно меньше
вмешиваться в жизнь растительных и животных сообществ на своих полях.
Поскольку многие западные фермеры не знакомы с севооборотом риса и озимых
зерновых и поскольку м-р Фукуока в книге «Революция одной соломинки» много места
уделяет технологии выращивания риса, может быть, полезно несколько слов сказать о
традиционном японском земледелии.
В древние времена семена риса просто разбрасывали по затопленной водой речной
долине в сезон муссонов. С течением времени в долинах реки стали делать террасы для
задержания воды после окончания сезонного разлива. В соответствии с традиционным
методом, используемым плоть до конца Второй Мировой Войны, семена риса высевали в
тщательно подготовленный питомник. Компост и навоз разбрасывали по полю, которое
затем затопляли, и после вспашки почва питомника приобретала консистенцию горохового
супа. Когда проростки достигали приблизительно 20 см высоты, их вручную пересаживали в
поле. Опытный фермер мог засадить за день около 0,13 га, но почти всегда эту работу делало
много людей, работая вместе.
После того, как рис был пересажен, поле слегка рыхлили в междурядьях, затем вручную
пололи и часто мульчировали. В течение трёх месяцев поле было затоплено водой, слой воды
над поверхностью почвы достигал 2,5 и более сантиметров. Урожай убирали вручную
серпами. Рис. связывали в снопы и на несколько недель подвешивали сушиться перед
обмолотом на деревянных или бамбуковых жердях. От пересадки до уборки урожая каждый
дюйм поля был по меньшей мере четыре раза пройден руками. После завершения уборки
риса, поле перепахивали и из почвы формировали плоские гребни приблизительно 0,5 м
шириной, разделёнными дренажными бороздами. Семена ржи или ячменя разбрасывали по
поверхности гребней и заделывали в почву. Такой севооборот был возможен только при
хорошо спланированном распорядке работ и постоянной заботе об обеспечении полей
органическим удобрением и важнейшими питательными веществами. Интересно отметить,
что, используя традиционный метод, японские фермеры выращивали рис и озимые зерновые

каждый год на одном и том же поле в течение столетий без снижения плодородия почвы.
Признавая многие ценные достижения метода, м-р Фукуока предполагает, что он
включает в себя и те работы, без которых можно обойтись. Он говорит о своём собственном
методе как о методе «ничего-не-делания» и считает, что, используя его, даже «воскресный
фермер» может вырастить достаточно пищи для своей семьи. Однако, он не имеет в виду, что
этот способ возделывания культур исключает всякое усилие. Его ферма держится на графике
регулярных работ полевых рабочих. То, что делается, должно быть сделано своевременно и с
пониманием. Если фермер решил, что на этом участке земли должен расти рис или овощи и
посеял семена, это значит, что он принял на себя ответственность за этот участок земли.
Разрушить природу и затем бросить её – это безответственно и пагубно.
Осенью м-р Фукуока высыпает семена риса, белого клевера и озимых зерновых
одновременно на одно поле и покрывает их толстым слоем рисовой соломы. Ячмень (или
рожь) и клевер прорастают сразу же, семена риса находятся в покое до весны.
Пока озимые зерновые растут и зреют на нижних полях, сады на склонах холмов
становятся центром активной работы. Сбор цитрусовых продолжается с середины ноября до
апреля.
Рожь и ячмень убирают в мае и расстилают для просушки на 7–10 дней. Затем их
обмолачивают, провеивают и убирают в мешки для просушки. Солому расстилают на поле
как мульчу. Воду держат на полях короткое время в период июньских муссонных дождей,
чтобы ослабить клевер и сорняки и дать возможность рису прорасти через соломенное
покрытие. После того, как вода спущена с поля, клевер оправляется и разрастается под
покровом риса. С этого момента и до уборки урожая (для традиционного фермера это время
напряжённого труда) единственная работа на поле м-ра Фукуока – это поддерживать в
порядке дренажные канавы и узкие пешеходные дорожки между полями. Рис. убирают в
октябре. Снопы подвешивают для просушки и затем обмолачивают. Осенний сев
заканчивается как раз к тому времени, когда ранние сорта мандарина созревают и готовы для
уборки. М‑р Фукуока собирает от 50 до 58 ц риса с гектара. Это приблизительно такой же
урожай, который получают или с применением химикатов или традиционным методом в
соседних хозяйствах. Урожай озимых зерновых на полях м-ра Фукуока часто выше, чем у его
соседей, использующих гребневой метод выращивания озимых.
Все три метода (натуральный, традиционный и химический) дают сравнимые урожаи,
но значительно отличаются по своему воздействию на почву. Почва на полях м-ра Фукуока с
каждым сезоном становится лучше. В течение последних двадцати пяти лет с тех пор, как он
прекратил вспашку, постоянно повышается плодородие его полей, улучшается структура и
водоудерживающая способность почвы. При традиционном методе плодородие почвы в
течение многих лет остаётся на постоянном уровне. Урожай, который получает фермер,
прямо пропорционален тому количеству компоста и навоза, которое он внесёт в почву. При
химическом методе почва становится безжизненной и её естественное плодородие в течение
короткого времени разрушается.
Одно из великих преимуществ метода м-ра Фукуока заключается в том, что рис можно
выращивать не затопляя поля на весь вегетационный период. Мало кто может даже
представить себе такую возможность. Однако это возможно, и м-р Фукуока утверждает, что
при таком способе рис растёт лучше. Его рис имеет крепкий стебель и глубокую корневую
систему. Старый сорт, богатого клейковиной риса, который он выращивает, даёт 250–300
зёрен на одно растение.

Использование мульчи увеличивает способность почвы удерживать воду. Во многих
местах натуральный метод может полностью снять проблему ирригации. Таким образом, рис
и другие высокоурожайные культуры можно выращивать в таких районах, которые ранее
считались непригодными для них. Крутые склоны и другие неудобные земли можно
окультурить, не опасаясь эрозии. Натуральный метод позволяет вернуть плодородие почвам,
испорченным неправильной обработкой или химикатами.
Болезни растений и вредные насекомые можно найти и на полях м-ра Фукуока, но они
никогда не вызывают существенных повреждений культурных растений. Повреждаются
только самые слабые растения. М‑р Фукуока утверждает, что лучший способ держать под
контролем болезни и вредителей – это создать для растений здоровую среду.
Плодовые деревья в саду м-ра Фукуока не подрезаны для более удобного сбора плодов,
их кроны растут свободно и принимают свою естественную форму. Овощи и травы
выращивают в саду с минимумом почвенной обработки. Весной семена капусты, редиса,
сои, горчицы, репы, садового лопуха, моркови и других овощей смешивают вместе и
разбрасывают между деревьями, чтобы они проросли до начала долгих весенних дождей.
Такой способ посева, очевидно, годится не для всех условий. Он хорошо работает в Японии с
её влажным климатом и обильными весенними дождями. В саду м-ра Фукуока почва
глинистая. Рыхлый поверхностный слой богат органическим веществом и хорошо
удерживает воду. Этот слой образовался в результате деятельности растительного покрова из
сорняков и клевера, которые росли в саду постоянно в течение многих лет.
Когда проростки овощных растений ещё молоды и слабы, сорняки должны быть
скошены, но когда овощи достаточно разовьются, их оставляют расти вместе с естественным
растительным покровом. Некоторые овощи остаются неубранными, их семена попадают в
почву и через одно-два поколения они возвращаются к свойствам своих выносливых и слегка
горьковатых на вкус диких предков. Многие из этих овощей растут совершенно без всякого
ухода. Однажды вскоре после того, как я пришёл на ферму м-ра Фукуока, я проходил через
отдалённую часть сада и неожиданно споткнулся обо что-то твёрдое в высокой траве.
Наклонившись, чтобы лучше рассмотреть, я нашёл огурец и рядом с ним в траве угнездилась
тыква.
В течение многих лет м-р Фукуока писал о своём методе в книгах и журналах и давал
интервью по радио и телевидению, но почти никто не следовал его примеру. В это время
японское общество предопределённо развивалось в прямо противоположном направлении.
После Второй Мировой Войны американцы принесли в Японию методы современного
химического земледелия. Благодаря им японский фермер смог получать почти такой же
урожай, как и при традиционном методе, но при затратах времени и труда почти вдвое
меньших. Казалось, что осуществилась давнишняя мечта, и в течение одного поколения
почти все фермеры переключились на химическое земледелие.
В течение столетий японский фермер поддерживал высокое содержание органического
вещества в почве путём чередования культур, путём внесения компоста и навоза и путём
выращивания покровных культур. Но когда эта практика была отвергнута и вместо этого
стали использовать быстро действующие химические удобрения, гумус почвы за время
жизни одного поколения разрушился. Почвенная структура также разрушилась, растения
стали слабыми и полностью зависимыми от химических удобрений. Чтобы компенсировать
снижение затрат труда человека и сельскохозяйственных животных, новая система
хищнически использовала резервы почвенного плодородия.

В течение последних сорока лет м-р Фукуока со скорбью наблюдал за деградацией и
земли, и японского общества. Японцы прямолинейно следовали американской модели
экономики и индустриального развития. Произошли сдвиги в размещении населения, так как
фермеры из деревни мигрировали в растущие индустриальные центры. Деревня, где м-р
Фукуока родился и где его семья жила в течение, вероятно, 1400 лет или больше, теперь
оказалась на границе растущих пригородов города Матсуяма. Национальное шоссе с его
мусором из бутылок саке и прочей ерунды пролегло через рисовые поля м-ра Фукуока. Хотя
он не идентифицирует свою философию с каким-то определённым религиозным
направлением или организацией, терминология м-ра Фукуока и его методика обучения
выдают сильное влияние Дзен-буддизма и таоизма. Иногда он цитирует Библию или
использует идеи иудео-христинской философии и теологии, чтобы проиллюстрировать свои
высказывания или вызвать дискуссию.
М‑р Фукуока считает, что натуральное хозяйство возникает из душевного здоровья
личности. Он предполагает, что оздоровление страны и очищение человеческого духа – это
один и тот же процесс и он предлагает такой способ жизни и такой способ земледелия,
которые могут способствовать этому процессу.
Было бы наивно думать, что при его жизни и в современных условиях, м-р Фукуока
сможет полностью реализовать свою идею на практике. Даже через 30 лет работы его
техника находится в стадии разработки. Его великий вклад в сокровищницу человеческого
духа заключается том, что он продемонстрировал, как повседневный процесс становления
духовного здоровья может вызвать благотворное, преображение всего мира.
Сегодня всеобщее осознание опасности долговременного использования химического
метода снова вызвало интерес к альтернативным методам земледелия. М‑р Фукуока занял
положение лидирующего агитатора за сельскохозяйственную революцию в Японии. Со
времени публикации «Революции одной соломинки» в октябре 1975 года интерес к
натуральному земледелию быстро распространился среди населения Японии.
В течение полутора лет, когда я работал у м-ра Фукуока, я часто возвращался на мою
ферму в Киото. Там каждый хотел попробовать новый метод и постепенно всё больше и
больше нашей земли переводилось на путь натурального хозяйства. Кроме риса и ржи в
традиционном севообороте, мы выращивали также пшеницу, гречиху, кукурузу, картофель и
сою по методу м-ра Фукуока. Чтобы посадить кукурузу и другие пропашные культуры,
которые медленно прорастают, мы проделывали в почве отверстия палкой или куском
бамбука и бросали семена в каждое углубление. Тем же способом мы сажали сою между
растениями кукурузы или семена сои покрывали оболочкой из глины и разбрасывали по
полю. Затем мы скашивали растительный покров из сорняков и покрывали поле соломой.
Клевер снова отрастал, но только после того, как кукуруза и соя становилась крепкими и
хорошо развитыми растениями.
М‑р Фукуока помогал нам советами, но мы должны были сами освоить метод путём
проб и ошибок и приспособить его к нашим различным культурам и местным условиям. Мы
знали с самого начала, что потребуется не один год и для земли и для наших душ, чтобы
измениться и встать на путь натурального земледелия. Это превращение стало длительным
процессом.
Ларри Корн

Глава I Посмотрите на эти поля зерновых
Я верю, что революция может начаться с одной соломинки. На вид эта рисовая
соломинка может показаться лёгкой и незначительной. Вряд ли кто-нибудь поверит, что она
способна начать революцию. Но я пришёл к понимаю веса и силы этой соломинки. Для меня
эта революция совершенно реальна.
Посмотрите на эти поля ржи и ячменя. Их зреющее зерно даёт урожай около 58 ц с
гектара. Я думаю, что это высшая отметка урожайности в префектуре Эхиме. И если это
лучший урожай в префектуре Эхиме, это может также быть высший урожай во всей стране,
поскольку это один из ведущих сельскохозяйственных районов во всей Японии. И тем не
менее, эти поля не были вспаханы в течение 25 лет.
При осеннем посеве я просто разбрасываю семена ржи и ячменя по поверхности поля в
то время, как на нём ещё растёт рис. Через несколько недель я убираю рис и рисовую солому
разбрасываю по поверхности земли.
То же самое для риса. Озимые зерновые скашивают приблизительно 20 мая. За две
недели до того, как зерно полностью созреет, я разбрасываю семена риса по полям, занятым
рожью и ячменём. После уборки и обмолота озимых зерновых, я раскидываю по полям
ячменную и рисовую солому.
Я думаю, что использование одного и того же метода для посева риса и озимых
зерновых – уникальная особенность этой системы земледелия. Если вы пройдёте к
следующему полю, позвольте мне обратить ваше внимание на то, что рис здесь был посеян
прошлой осенью одновременно с озимыми зерновыми. Так что на этом поле все посевы года
была закончены к Новому году. Это ещё один способ облегчения труда.
Вы можете также заметить, что на этом поле растут белый клевер и сорняки. Семена
клевера были высеяны между растениями риса в начале октября незадолго до посева ржи и
ячменя. О посеве сорняков я не беспокоился – они очень легко обсеменяются сами.
Таким образом, очерёдность посевов на этом поле следующая: в начале октября семена
клевера разбрасывают по растениям риса, затем в середине октября следует посев озимых
зерновых. В начале ноября рис убирают, затем высевают семена риса следующего года и
поверхность поля покрывают соломой. Рожь и ячмень, которые вы видите перед собой, были
выращены таким способом.
На поле площадью одна десятая гектара один или два человека могут сделать всю
работу по выращиванию риса и озимых зерновых за несколько дней. Вряд ли может
существовать более простой способ возделывания зерновых.
Этот метод совершенно противоположен современной сельскохозяйственной
технологии. Он вышвыривает в окно всё научное знание и ноу-хау традиционного
земледелия. Этот способ земледелия, не использующий ни машин, ни готовых удобрений, ни
химических средств защиты, позволяет получать урожай равный или более высокий, чем на
средней японской ферме. Доказательство этого созревает прямо перед вашими глазами.
Совсем ничего
Недавно люди спросили меня, почему я много лет назад начал заниматься этим методом
земледелия. До сих пор я никогда не обсуждал это ни с кем. Вы могли бы сказать, что просто
не было повода говорить об этом. Это был просто, как вы сказали бы, шок, вспышка, одно
маленькое переживание, которое стало отправной точкой.

Это прозрение полностью изменило мою жизнь. В этом новом видении нет ничего
конкретного, но суть его можно приблизительно описать так: «Человечество не знает совсем
ничего. Ничто не имеет внутренней ценности и всякое действие – это тщетное,
бессмысленное усилие». Это может показаться абсурдным, но это единственный способ
выразить словами мою мысль.
Эта мысль возникла в моей голове внезапно, когда я был ещё совсем молод. Я не знал,
было ли правильно или нет это интуитивное понимание того, что все человеческие знания и
усилия ничего не стоят, но если я сомневался и пытался отогнать эту мысль, то внутри себя я
не мог найти ничего, что бы противопоставить ей. Только твёрдая уверенность, что это так
горела во мне.
Обычно думают, что нет ничего более великолепного, чем человеческий разум, что
человеческие существа – это вершина творения и что их созидания и свершения, отражённые
в культуре и истории, выглядят потрясающе. Это распространённая точка зрения.
Поскольку то, что я думал, было отрицанием этого распространённого воззрения, я был
не в состоянии объяснить кому-нибудь свой взгляд на вещи. Постепенно я решил придать
моим мыслям форму, претворить их в практическую деятельность и, таким образом,
определить, было ли моё понимание правильно или ошибочно. Посвятить свою жизнь
работе на ферме, выращиванию риса и озимых зерновых – это было направление, которому я
решил следовать.
Что же это было за переживание так изменившее мою жизнь?
Сорок лет назад, когда мне было 25 лет, я работал в Таможенном Управлении Иокогамы
в отделе инспекции растений. Моя основная обязанность заключалась в том, чтобы
проверять ввозимые и вывозимые растения на наличие насекомых – носителей болезней.
Мне повезло, так как я имел много свободного времени, которое я проводил в лаборатории,
занимаясь исследованиями по моей специальности, фитопатологии. Эта лаборатория
находилась недалеко от парка Ямате и стояла на высоком обрыве над гаванью Иокогамы.
Прямо перед зданием была расположена католическая церковь, а к востоку от неё – школа
для девушек. Это было очень спокойное место, которое создавало прекрасные условия для
занятий научной работой.
Научным работником лаборатории патологии был Эйичи Куросава. Я изучал
фитопатологию у Макото Окера, преподавателя Высшей Сельскохозяйственной Школы в
Гифу, и под руководством Суехико Игата из сельскохозяйственного Исследовательского
Центра префектуры Окаяма.
Мне очень повезло, что я работал с профессором Куросава. Хотя он остался мало
известным в Академическом мире, это был человек, который изолировал и вырастил
культуру гриба, вызывающую болезнь риса «бакане». Он стал первым, кто экстрагировал из
культуры гриба гормон роста растений гиббереллин. Этот гормон, поглощённый в
небольшом количестве молодым растением риса, даёт удивительный эффект, вызывая
ненормальный сильный рост растений в высоту. В больших количествах этот гормон даёт
противоположный эффект, задерживая рост растений. Никто в Японии не обратил особого
внимания на это открытие, но за рубежом оно стало предметом активных исследований.
Вскоре после этого в США стали использовать гиббереллин для получения бессемянного
винограда.
Я уважал Куросава-сан (форма обращения в Японии, одинаковая для мужчин и женщин)
как своего собственного отца и под его руководством я создал препаративный микроскоп и

посвятил себя исследованию болезней, вызывающих гниль стволов, ветвей и плодов
американских и японских сортов цитрусовых.
В микроскоп я наблюдал культуру грибов, скрещивание различных видов и образование
новых болезнетворных видов. Я был увлечён моей работой. Поскольку мои занятия
требовали глубокой постоянной концентрации, то бывали моменты, когда я буквально падал
без сознания от усталости во время работы в лаборатории.
Это было также время расцвета юности и я не всё своё время проводил, закрывшись в
лаборатории. Я жил в портовом районе Иокогамы, не лучшее место, чтобы шататься по
улицам и приятно проводить время.
В это время произошёл следующий эпизод. Погружённый в себя с фотоаппаратом в
руках, я прогуливался по причалу и вдруг увидел красивую женщину. Я подумал, что она
может послужить прекрасным объектом для фотографии и попросил её позировать мне. Я
помог ей подняться на палубу иностранного парохода, стоявшего здесь на якоре, и попросил
её принять одну позу, потом другую и сделал несколько снимков. Она попросила прислать ей
фотографии, когда они будут готовы. Когда я спросил, куда их прислать, она просто
сказала:«В Офуна» и ушла, не назвав своего имени.
Когда я проявил плёнку, я показал другу отпечатки и спросил, узнаёт ли он, кто это. Он
ахнул и сказал: «Это Миеко Такамине, известная кинозвезда». Я немедленно отослал ей в
город Офуна десять увеличенных отпечатков. Вскоре фотографии с автографами были
возращены мне по почте. Но одной фотографии среди них не было. Думая об этом позже, я
понял, что это был снимок, сделанный крупным планом в профиль и, очевидно, на нём были
заметны морщинки на её лице. Я был доволен и чувствовал себя так, будто мне удалось на
мгновение заглянуть в тайну женской психики.
Хотя я был неуклюж и неловок, я часто ходил в танцевальный зал в районе Нанкингаи.
Однажды я увидел там популярную певицу Норико Авайя и пригласил её на танец. Я никогда
не забуду этого танца, потому что я был совершенно ошеломлён её телом, таким огромным,
что я не смог обнять её рукой за талию. Так или иначе я был очень занятый, очень удачливый
молодой человек, дни которого проходили в постоянном изумлении перед миром природы,
открывающемся мне через объектив микроскопа, поражая сходством этого микромира с
большим миром бесконечной Вселенной. По вечерам, влюблённый или нет, я флиртовал с
девушками и наслаждался жизнью. Я думаю, что эта бесцельная жизнь и переутомление от
напряжённой работы привели в конце концов к повторяющимся обморокам во время работы.
Затем я заболел острой пневмонией и был помещён в палату на последнем этаже
Полицейского Госпиталя, где мне сделали пневмоторакс.
Была зима и сквозь разбитое окно врывался ветер и разносил снег по всей комнате. Под
одеялом было тепло, но моё лицо было холодно как лёд. Медсестра измеряла мне
температуру и тут же уходила. Поскольку моя комната была на отшибе, никто ко мне не
заглядывал. Мне казалось, что я был брошен на милость холода, и внезапно я погрузился в
мир одиночества. Я ощутил себя один на один со страхом смерти. Когда я думаю об этом
теперь, этот страх кажется беспричинным, но в то время это было очень сильное чувство.
В конце концов, я был выписан из госпиталя, но я не мог выбраться из состояния
депрессии. Во что я верил до сих пор? Я ни о чём не задумывался и был доволен, но какова
была природа этого благодушия? Я был в смятении от своих размышлений о природе жизни и
смерти. Я не мог спать, не мог заниматься своей работой. В еженощных блужданиях по
кручам недалеко от гавани я не мог найти облегчения.

Однажды ночью, когда я как обычно бесцельно бродил, я упал без сил в полном
изнеможении на вершине холма, с которого открывался вид на гавань и задремал,
прислонившись к стволу большого дерева. Я лежал там, не бодрствуя и не засыпая до
рассвета. Я даже могу припомнить, что это было утро 15 мая. В полусне я наблюдал как
гавань светлеет, и, видя восход солнца, я в то же время как бы и не видел его.
Когда внизу подул лёгкий бриз, утренний туман внезапно исчез. Как раз в этот момент
появилась ночная цапля, издала резкий крик и улетела прочь. Я мог слышать удары её
крыльев. В это мгновение все мои сомнения и мрачный туман моего смятения исчезли. Всё,
что было моим твёрдым убеждением, всё, чему я раньше доверял, было унесено ветром. Я
чувствовал, что я понял только одну вещь. Без участия моего разума слова сами пришли ко
мне: «В этом мире совсем ничего нет». Я чувствовал, что я ничего не понял (ничего не
понять в этом смысле означает осознание незначительности интеллектуального знания).
Я мог видеть, что все концепции, которые я разделял, все представления о самом
существовании, были пустыми выдумками. Мой дух стал светлым и ясным. Я дико плясал от
радости. Я мог слышать щебетание маленьких птичек на деревьях и видеть далёкие волны с
бликами восходящего солнца. Листва деревьев колыхалась надо мной зелёная и блестящая. Я
чувствовал, что это был настоящий рай на земле. Всё, что владело мной, всё смятение
испарилось как сон, и что-то одно, что можно назвать «истинной природой» открылось мне.
Я думаю, можно смело сказать, что после переживания того утра моя жизнь полностью
изменилась.
Несмотря на перемену, я остался в своей основе средним, неумным человеком, и это так
и сохранилось без изменений с тех пор и до настоящего времени. Глядя со стороны, в моей
ежедневной жизни нельзя было найти ничего экстраординарного. Но уверенность, что я
знаю эту одну вещь с тех пор не покидала меня. Я провёл тридцать лет, сорок лет, проверяя
не ошибся ли я, всё время осмысливая пройденный путь, но ни разу я не нашёл
доказательств, противоречащих моему убеждению.
То, что это прозрение само по себе имеет огромное значение, не означает, что и я
приобрёл кукую-то особую значительность. Я остался простым человеком, старым вороном,
так сказать. Для случайного наблюдателя я могу показаться или скромным или
высокомерным. Я повторяю молодым людям в моём саду снова и снова, чтобы они не
пытались подражать мне, и меня, действительно, сердит, когда кто-то из них не принимает
всерьёз этого совета. Вместо этого, я прошу, чтобы они просто жили в природе и выполняли
свою дневную работу. Нет, во мне нет ничего особенного, но то, что мне удалось понять – в
высшей степени важно.

Возвращение в деревню
В один из дней после этого случая я сделал отчёт о своей работе и тут же подал
заявление об уходе. Мой начальник и друзья были удивлены. Они не знали, что с этим делать.
Они устроили мне прощальный вечер в ресторане над набережной, но атмосфера была
несколько необычная. Молодой человек, который до сегодняшнего дня хорошо ладил со
всеми, который не казался неудовлетворённым своей работой, а наоборот, был всем сердцем
предан своим исследованиям, вдруг внезапно объявляет, что он бросает всё и уходит. А я был
счастлив и смеялся.
В это время я всем говорил следующее: «На этой стороне – набережная. На другой
стороне – пирс № 4. Если вы представите себе, что на этой стороне – жизнь, тогда на другой
стороне – смерть. Если вы хотите избавиться от мысли о смерти, то вы должны избавиться
также от мысли, что на этой стороне – жизнь. Жизнь и смерть едины».
Когда я говорил это, каждый становился ещё более обеспокоен моим состоянием. «Что
он говорит? Он, наверное, сошёл с ума», – должно быть, думали они. Они провожали меня с
печальными лицами. Только я один шагал бодро, в хорошем настроении.
В это время сосед по комнате был особенно сильно обеспокоен моим поведением. Он
предложил мне немного отдохнуть, возможно, на полуострове Босо. Одним словом, я ушёл. Я
уехал бы в любое место, если бы кто-то пригласил меня. Я сел в автобус и ехал много миль,
глядя на поля с рисовыми чеками и маленькие деревушки вдоль дороги. На одной остановке
я увидел маленький указатель, на котором было написано «Утопия». Я вышел из автобуса и
пошёл искать её.
На побережье была маленькая гостиница. Поднявшись на утёс, я нашёл место с
прекрасным видом. Я остановился в гостинице и проводил дни, валяясь в полудрёме в
высокой траве высоко над морем. Это продолжалось, может быть, несколько дней, неделю,
месяц, но, во всяком случае, я оставался там некоторое время. Дни проходили и моя радость
тускнела, и я начал осмысливать, что же всё-таки случилось. Вы могли бы сказать, что,
наконец, пришёл в себя.
Я поехал в Токио и оставался там некоторое время, проводя дни в прогулках по парку,
разговаривая на улицах с людьми, а спал, где придётся. Мой друг беспокоился обо мне и
приехал посмотреть, как я живу. «Разве ты не живёшь в мире снов, в мире иллюзий?» «Нет, –
ответил я, – это вы живёте в мире снов». Когда мой друг обернулся, чтобы сказать «До
свидания», я ответил ему что-то вроде: «Не говори «До свидания», прощаться, так
прощаться». Мой друг, кажется, потерял всякую надежду.
Я покинул Токио, пересёк район Консаи (Осака, Кобе, Киото) и, двигаясь на юг,
добрался до Кюсю. Я наслаждался, кочуя с места на место вместе с ветром. Я испытывал
многих людей моим открытием, что всё бессмысленно и не имеет значения, что всё
возвращается в ничто. Но это было слишком много или слишком мало, чтобы быть понятым
в нашем мире, занятом своей повседневной жизнью. Никакой связи с этим миром не было. Я
мог только мысленно представлять себе эту «концепцию бесполезности» как великое благо
для мира и особенно для современного мира, который так быстро двигался в
противоположном направлении. Я намеревался распространить свою идею по всей стране.
Но результат был таков, что всюду, где бы я ни появлялся, меня рассматривали только как
эксцентричного молодого человека. Тогда я вернулся на ферму моего отца в деревню.

Мой отец выращивал в это время мандарины, и я поселился в хижине на горе и стал
жить очень простой, примитивной жизнью. Я думал, что если здесь я смогу на реальном
примере выращивания мандаринов и зерновых продемонстрировать своё понимание жизни,
мир признает мою правоту. Разве не лучший путь, вместо сотни объяснений, практически
претворить свою философию в жизнь? С этой мысли начался мой метод земледелия,
который условно можно назвать «ничего-не делание» (этим выражением м-р Фукуока
привлекает внимание к сравнительной лёгкости своего метода. Этот метод земледелия
требует напряжённой работы, особенно во время уборки, но всё же значительно меньше, чем
другие методы). Это был 1938 год, 13‑ый год правления нашего императора.
Я обосновался на горе и всё шло хорошо, пока мой отец не доверил мне обильно
плодоносящие деревья в саду. Он уже подрезал крону деревьев придав им форму «чашки для
сакэ», так что с них было легко собирать плоды. Когда я оставил их в этом состоянии без
ухода, то в результате ветки переплелись, насекомые атаковали деревья и весь сад в короткое
время пришёл в жалкое состояние.
Моё убеждение состояло в том, что культурные растения должны расти сами по себе и
не должны быть выращиваемы. Я действовал в уверенности, что всё должно быть
предоставлено своему естественному развитию, но я обнаружил, что если вы примените на
практике эту идею без необходимой подготовки, то довольно долго ваши дела будут идти
неважно. Это просто бесхозяйственность, а не «натуральное хозяйство». Мой отец был
потрясён. Он сказал, что я должен дисциплинировать себя, может быть, устроиться где-то на
работу и вернуться обратно, когда я снова возьму себя в руки. В это время мой отец был
старостой деревни, и другим членам деревенской общины было трудно понять его
эксцентричного сына, который явно не мог наладить свои отношения с миром людей,
живущих на холмах. Кроме того, мне не нравилась перспектива военной службы, и поскольку
война становилась всё более ожесточённой, я решил исполнить желание моего отца и
устроиться на работу.
В это время специалистов было немного. Опытная Станция префектуры Коти слышала
обо мне и мне предложили пост главного научного работника Службы контроля болезней и
вредителей. Я пользовался расположением префектуры Коти почти восемь лет. В Опытном
Центре я стал инспектором в отделе научного земледелия и погрузился в исследования по
увеличению производства продуктов питания в военное время. Но в действительности в
течение этих восьми лет я обдумывал взаимоотношения между научным и натуральным
земледелием. Химическое земледелие, которое использует плоды человеческого интеллекта,
признано самым прогрессивным. Вопрос, который всегда вертелся у меня в голове, был
такой: может или нет натуральное земледелие противостоять современной науке?
Когда война окончилась, я почувствовал свежий ветер свободы и со вздохом облегчения
вернулся в мою деревню, чтобы заново приняться за земледелие.

Путь к методу «ничего-не-делания»
В течение 30 лет я жил только моим хозяйством и имел мало контактов с людьми за
пределами моей собственной общины. В течение этих лет я прямиком двигался к созданию
метода земледелия «ничего-не-делания».
Обычно способ разработки метода заключается в том, что задают вопрос: «А что, если
попробовать это?» или «А что, если попробовать то?», то есть испытывают различные виды
агротехники один за другим. Такова современная сельскохозяйственная наука и
единственный её результат заключается в том, что она делает фермера ещё более занятым.
Мой способ прямо противоположен. Я стремлюсь к приятному, естественному способу
ведения сельского хозяйства (хозяйство ведётся так просто, как это возможно в естественной
среде и во взаимодействии с ней, в отличие от современной тенденции применять всё более
сложную технику, чтобы полностью переделать природу в угоду человеку), цель которого
сделать работу легче, а не труднее. «А что, если не делать этого? А что, если не делать
того?» – это мой способ мышления. В конце концов, я пришёл к заключению, что нет
необходимости пахать землю, нет необходимости вносить удобрения, нет необходимости
делать компост, нет необходимости использовать инсектициды. Когда вы додумываетесь до
этого, то остаётся немного таких агротехнических приёмов, которые действительно
необходимы.
Причина, по которой постоянное совершенствование агротехники кажется
необходимым, заключается в том, что естественный баланс уже так сильно нарушен этой
самой агротехникой, что земля становится зависимой от неё.
Эту причинно-следственную связь можно применить не только к сельскому хозяйству,
но также и к другим аспектам человеческой деятельности. Доктора и медицина становятся
необходимы, когда люди создают нездоровую среду. Формальное школьное обучение не
имеет внутренней ценности, но становится необходимым, когда человечество создаёт
условия, при которых человек должен получить «образование», чтобы жить.
Перед концом войны, когда я пытался в цитрусовом саду приобрести опыт натурального
ведения хозяйства, я не делал обрезки деревьев и предоставил им расти, как они хотят. В
результате ветки переплелись между собой, деревья подверглись нападению насекомых и
почти 0,8 га мандаринового сада пришли в негодность и погибли. С этого времени вопрос
«Что же такое натуральный метод?» не выходил у меня из головы. В процессе поиска ответа
я погубил ещё 400 деревьев. Но, наконец, я почувствовал, что я могу с уверенностью сказать:
«Вот натуральный метод».
В той степени, в какой деревья отклоняются от своей естественной формы и становятся
необходимы обрезка и уничтожение насекомых, в той же степени человеческое общество
отдаляется от жизни природы и становится необходимым школьное образование. В природе
формальное школьное обучение не имеет применения.
В воспитании детей многие родители делают ту же ошибку, которую я делал в саду на
первых порах. Например, обучение детей музыке также не нужно, как обрезка плодовых
деревьев. Детское ухо само ловит музыку. Бормотание ручья, лягушечье кваканье на берегу
реки, шелест листьев в лесу – все эти естественные звуки – это музыка, настоящая музыка.
Но когда врываются различные раздражающие шумы и сбивают с толку, детское чистое
восприятие музыки исчезает. Если продолжать в том же роде, то ребёнок будет неспособен

услышать песню в зове птицы или звуке ветра. И вот поэтому музыкальное воспитание
считается благотворным для детского развития.
Ребёнок, выросший с неиспорченным чистым слухом, возможно, не сумеет сыграть
популярные мелодии на скрипке или пианино, но я не думаю, что это имеет какое-то
отношение к способности слышать истинную музыку или петь. Когда сердце полно песней,
о таком ребёнке можно сказать, что он музыкально одарён.
Почти каждый думает, что «природа» – это хорошая вещь, но мало кто может постигнуть
разницу между тем, что свойственно и что несвойственно природе.
Если одну единственную новую почку срезать с фруктового дерева, это может вызвать
такие нарушения, которые будет невозможно исправить. Если дереву дают расти свободно в
естественной для него форме, то ветви отходят от ствола в определённой
последовательности, так что все листья получают солнечный свет одинаково. Если этот
порядок нарушен, ветви приходят в конфликт друг с другом, перекрывают одна другую,
сплетаются, и листья засыхают в тех местах, куда солнце не может проникнуть. Развивается
повреждение насекомыми. Если не сделать обрезку, то на следующий год появится ещё
больше засохших ветвей.
Вмешательство людей нарушает естественный ход вещей, а когда повреждения не
восстанавливаются и отрицательные эффекты накапливаются, начинают изо всех сил
трудиться, чтобы исправить их. Если это исправление оказывается успешным, они
рассматривают принятые меры как великолепное достижение. Люди повторяют это снова и
снова. Это как если бы глупец бездумно разбил черепицы свой крыши. А потом, обнаружив,
что потолок начал гнить от дождей, он поднимается на крышу и исправляет повреждение,
радуясь, что он нашёл прекрасное решение проблемы.
То же самое происходит с учёным. Он сгибается день и ночь над книгами, переутомляя
свои глаза и становясь близоруким, и если вы поинтересуетесь, над чем же он работал всё
это время, окажется, что он изобретал очки, чтобы исправить близорукость.

Возвращение к источнику
Опираясь на длинную ручку своей косы, я делаю перерыв в своей работе в саду и
смотрю на гору и на деревню внизу. Я удивляюсь, что философские системы сменяли одна
другую быстрее, чем происходила смена времён года.
Путь, которому я следовал, это натуральное земледелие, большинству людей кажущееся
странным, вначале интерпретировали как реакцию, направленную против интенсивного и
бесконтрольного развития науки, но всё, что я делал, работая здесь в деревне, – это попытка
показать, что человечество ничего не знает. Поскольку мир движется с бешеной энергией в
противоположном направлении, может показаться, что я просто отстал от времени, но я
твёрдо знаю, что путь, которым я следую, самый разумный.
В течение последних нескольких лет число людей, интересующихся натуральным
земледелием, значительно выросло. Кажется, что предел научного развития уже достигнут,
начинают появляться опасения в правильности выбранного пути и настаёт время
переоценок. То, что раньше считалось примитивным и отсталым, теперь неожиданно
видится как далеко опередившее современную науку. На первый взгляд это может показаться
странным, но я совсем не нахожу это странным.
Недавно я обсуждал это с профессором Инума из Университета в Киото. Тысячу лет
назад в Японии практиковалось земледелие без вспашки, и это продолжалось до начала Эры
Токугава 300–400 лет назад, когда было введено неглубокое рыхление почвы. Глубокая
вспашка пришла в Японию вместе с Западной сельскохозяйственной наукой. Я говорил, что
под давлением будущих проблем следующее поколение будет вынуждено вернуться к
беспахотному земледелию.
Выращивание культур на невспаханном поле может показаться на первый взгляд
возращением к примитивному земледелию, но в течение нескольких лет этот метод
проверялся в Университетских лабораториях и сельскохозяйственных опытных Центрах по
всей стране. И было показано, что он является самым простым, эффективным и
современным методом по сравнению со всеми другими. Хотя этот метод отрицает
современную науку, теперь он оказался на переднем крае в развитии современного сельского
хозяйства.
Я опубликовал описание этого «беспахотного с прямым высевом севооборота озимых
зерновых и риса» в сельскохозяйственном журнале 20 лет назад. С тех пор оно часто
появлялось в печати и было неоднократно представлено публике по радио и в телевизионных
программах, но никто не обращал на него внимания.
Теперь внезапно началась совсем другая история. Можно сказать, что натуральное
земледелие стало страстным увлечением. Журналисты, профессора, рядовые исследователи
толпятся, чтобы посетить мои поля и хижины на горе. Различные люди смотрят на это с
различных точек зрения, делают свои собственные умозаключения и уезжают. Одним это
кажется примитивным, другим – отсталым, а кто-то считает это вершиной
сельскохозяйственных достижений и даже более того – приветствует как прорыв в будущее.
Обычно люди озабочены только одним: является ли этот тип земледелия предвестником
будущего или возрождением прошлого. Немногие способны правильно понять, что
натуральное земледелие возникло из неподвижного и неизменного центра развития
сельского хозяйства. В той степени, в какой люди отдаляются от природы, они всё дальше и

дальше отдаляются от центра. В то же время центростремительная сила проявляется в том,
что возникает желание вернуться к природе. Но если люди просто случайно попадают в то
или иное течение, двигаясь направо или налево в зависимости от условий, то результатом
будет только большая активность. Неподвижная точка первоисточника, лежащая вне области
относительности, остаётся незамеченной ими. Я думаю, что даже движение «за возвращение
к природе», против загрязнения среды, как бы ни были они достойны поощрена, не
направлены на истинное решение проблемы, если они являются только реакцией на
гипертехнизацию настоящего века.
Природа не меняется, хотя пути познания природы неизбежно меняются от одной эпохи
к другой. Независимо от эпохи, натуральное земледелие существовало всегда, как
родниковый колодец, откуда берёт начало сельское хозяйство.

Почему натуральное земледелие не получило
широкого распространения?
В течение последних двадцати или тридцати лет натуральный метод выращивания риса
и озимых зерновых был испытан в широком диапазоне климатических и природных условий.
Почти в каждой префектуре Японии были проведены испытания с целью сравнить урожай
при «Прямом посеве без вспашки» и урожай риса и озимых зерновых, выращенных
традиционным способом гребней и борозд со вспашкой. Эти испытания не дали
доказательств, отрицающих универсальность натурального земледелия для различных
условий.
Итак, напрашивается вопрос, почему эта правда не получила широкого
распространения. Я думаю, одна из причин заключается в том, что мир стал так
специализирован, что люди потеряли способность охватить что-либо во всей полноте.
Например, эксперт по защите от насекомых-вредителей исследовательского Центра
префектуры Коти пришёл узнать, почему на моих полях так мало рисовой цикадки, несмотря
на то, что я не использую инсектициды. В результате установления баланса между
насекомыми и их естественными врагами, скоростью размножения пауков и некоторыми
другими факторами рисовая цикадка стала встречаться на моих полях так же редко, как на
полях Центра, которые бесчисленное количество раз были опрыснуты различными
смертельными химикатами. Профессор был также удивлён, обнаружив, что в то время, как
вредоносные насекомые на моих полях встречаются редко, их естественные враги на моих
полях гораздо более многочисленны, чем на полях, обработанных инсектицидами. Потом он
наконец понял что поля поддерживаются в таком состоянии благодаря естественному
балансу, установившемуся между различными сообществами насекомых. Он признал, что
если освоить мой метод, то проблема гибели культуры от рисовой цикадки может быть
решена. Потом он сел в свой автомобиль и уехал в Коти.
Но если вы спросите, были ли у меня специалисты исследовательского Центра,
занимающиеся вопросами почвенного плодородия или растениеводства, ответ будет: нет, они
не были. Но если вы предложите на конференции или собрании, чтобы мой метод или,
скорее, немётод был испытан в широком масштабе, я думаю, что префектура или опытная
станция скорее всего ответит: «Извините, для этого ещё не пришло время. Мы должны
сначала исследовать метод со всех возможных точек зрения прежде, чем дать окончательное
одобрение». Пройдут годы, прежде чем будет дано заключение.
Такие вещи происходят всё время. Специалисты и научные работники со всей Японии
приезжали на эту ферму. Глядя на поля с точки зрения своей собственной специальности,
каждый из этих исследователей находил их, по меньшей мере, удовлетворительными, если не
замечательными. Но в течение пяти или шести лет со времени визита профессора опытной
станции в префектуре Коти произошло мало перемен.
В этом году сельскохозяйственный факультет Университета в Кинки создал бригаду по
программе натурального земледелия, в составе которой студенты нескольких различных
факультетов приедут сюда для проведения исследований. Такой подход может стать шагом
вперёд, но я чувствую, что за этим может последовать два шага в обратном направлении.
Самозванные эксперты часто делают следующие замечания: «Основная идея метода

правильная, но не будет ли удобнее убирать урожай машиной?» или «Не будет ли урожай
более высоким, если вы используете удобрения или пестициды в некоторых случаях?»
Всегда найдутся те, кто пытается смешать натуральное и научное земледелие. Но такой
способ мышления полностью упускает главное. Фермер, который идёт на компромиссы, не
имеет более права критиковать науку на фундаментальном уровне.
Натуральное земледелие – тонкое дело и оно означает возвращение к источнику
земледелия. Каждый шаг в противоположном от источника направлении может только сбить
с пути.

Человечество не знает природы
Я надеялся, что должно придти такое время, когда учёные, политики, люди искусства,
философы, религиозные деятели и те, кто работает в полях, соберутся здесь, осмотрят эти
поля и вместе обсудят всё это. Я думаю, что это должно случиться, если люди научатся
смотреть на вещи, выйдя за пределы своей специальности.
Учёные думают, что они могут понять природу. Они стоят на этой точке зрения.
Поскольку они убеждены, что они могут понять природу, то им предоставлено право
исследовать её и поставить на службу человеку. Но, по моему мнению, понимание природы
лежит за пределами возможности человеческого разума.
Я часто говорю молодым людям в хижинах на горе, тем, кто пришёл сюда помогать и
изучать натуральное земледелие, что каждый может видеть деревья на горе. Они могут
видеть зелень листьев, они могут видеть рисовые растения. Они думают, они знают, что
такое зелень. Соприкасаясь с природой с утра до вечера, они начинают думать, что они
знают природу. Но когда они думают, что они начинают понимать природу, можно сказать
уверенно, что они на ложном пути.
Почему невозможно познать природу? То, что понимают под природой – это только идея
природы, возникающая в сознании каждого отдельного человека. Истинную природу видят
дети. Они видят без размышления, непосредственно и ясно. Если известны даже названия
растений, например мандариновое дерево из семейства цитрусовых, сосна из семейства
сосновых, в этих названиях природа не отражена в её истинных формах.
Объект, который рассматривают изолированно от целого, – не реальная вещь.
Специалисты из различных областей науки собираются вместе и рассматривают побег
риса. Специалист по защите от вредителей видит только повреждения, вызванные
насекомыми; специалист по питанию растений интересуется только энергией роста. Это
неизбежно при сегодняшнем положении вещей.
Приведу пример. Я сказал джентльмену с Опытной станции, когда он исследовал
взаимоотношения между рисовой цикадкой и пауками на моих полях: «Профессор,
поскольку вы исследуете пауков, вы интересуетесь только одним видом среди многих
естественных врагов рисовой цикадки. В этом году пауки появились в очень большом
количестве, но в прошлом году были жабы. Перед этим преобладали лягушки. Существуют
бесчисленные вариации».
Для специализированного исследователя невозможно понять роль отдельного хищника в
сложности взаимоотношений разных видов насекомых. Есть сезоны, когда популяция
цикадки немногочисленна, потому что много пауков. В другое время, когда выпадает много
дождей, тогда лягушки уничтожают пауков, или когда выпадает мало дождей, тогда ни
лягушки, ни цикадки не появляются совсем.
Метод контроля насекомых, который игнорирует взаимодействие между самими
насекомыми, поистине бесполезен. При изучении пауков и цикадки необходимо учитывать
взаимоотношения между лягушками и пауками. Это значит, что не обойтись без профессора
по лягушкам. Эксперты по паукам и цикадке, эксперт по рису и эксперт по водному режиму
должны будут также присоединиться к собранию специалистов по борьбе с вредителями
риса.
Кроме того, на этих полях существует четыре или пять различных видов пауков. Я

вспоминаю, как несколько лет назад кто-то прибежал ко мне домой рано утром, чтобы
спросить, покрыл ли я свои поля шёлковой сетью или чем-то похожим на неё. Я не мог
представить, о чём он говорит, поэтому поспешил посмотреть, в чём дело.
Мы как раз закончили уборку риса, и за ночь стерня риса и низкие травы полностью
покрылись паутиной как шёлковой сетью. Колыхаясь и сверкая в утреннем тумане, она
представляла волшебное зрелище. Чудо заключается в том, что, когда это случается, а это
бывает чрезвычайно редко, оно продолжается только день или два. Если вы пристально
вглядитесь, вы увидите несколько пауков на квадратный дюйм (5,25 см2). Они так плотно
покрывают поле, что между ними почти не остаётся пространства. На гектар их должно
быть много тысяч или много миллионов! Если вы придёте посмотреть на поле через два-три
дня, вы увидите, что нити паутины длиной несколько метров, оторвались и развеваются по
ветру с прицепившимися к каждой нити пятью или шестью пауками. Это похоже на пух
одуванчика или семена сосны, разносимые ветром. Молодые пауки прицепляются к нитям и
таким образом парят в воздухе.
Это зрелище представляет удивительную драму природы. Видя это, вы понимаете, что
поэты и художники также должны будут присоединиться к собранию специалистов по
борьбе с вредителями.
Когда поле обрабатывают химикатами, всё это мгновенно разрушается. Я однажды
подумал, что не было бы ничего плохого, если на полях разбрасывать древесную золу (м-р
Фукуока делает компост из древесной золы и других домашних органических отходов. Он
применяет его в своём маленьком саду при кухне). Результат был бы удивительным. Через
два или три дня поле было бы совершенно чисто от пауков. Зола вызывает распад нитей
паутины. Как много тысяч пауков пали бы жертвой одной единственной горсти этой как
будто бы безвредной золы? Применение инсектицидов приводит не просто к уничтожению
цикадки вместе с её естественными хищниками. Оно оказывает влияние на многие важные
процессы природы.
Феномен этого великого скопления пауков, которые появились на рисовом поле осенью
и исчезли через один день, ещё не понят до конца. Никто не знает, откуда они приходят, как
переживают зиму и куда они деваются, когда исчезают.
Итак, использование химикатов – это проблема не только для энтомологов. Философы,
религиозные деятели, художники и поэты должны также помочь решить, допустимо или нет
использовать химикаты в земледелии и каковы могут быть результаты использования даже
органического удобрения.
Мы предполагаем собрать около 58 ц риса и 58 ц озимых зерновых с гектара этой земли.
Если урожай достигнет 78 ц, как это бывало в некоторые годы, это будет самый высокий
урожай в Японии. Поскольку прогрессивная технология не применялась при выращивании
этих зерновых, полученный нами результат можно рассматривать как опровержение
положений современной науки. Каждый, кто приедет и увидит эти поля и воспримет их
немое свидетельство, почувствует глубокое недоверие к утверждению, что человечество
знает природу. И поневоле перед ним встанет вопрос, может ли природа быть познана ввиду
ограниченности человеческого понимания вообще.
Ирония заключается в том, что наука служит только для того, чтобы показать, как
ничтожны человеческие знания.

Глава II Четыре принципа натурального земледелия
Пройдите неспеша по этим полям. Стрекозы и мотыльки суетятся в воздухе. Пчёлы
перелетают с цветка на цветок. Раздвиньте листья и вы увидите насекомых, пауков, лягушек,
ящериц и многих других мелких животных, снующих в прохладной тени. Кроты и дождевые
черви роются под поверхностью почвы.
Это сбалансированная экосистема рисового поля. Сообщества насекомых и растений
находятся здесь в стабильных взаимоотношениях. Нет ничего необычного в том, что болезни
растений, распространённые в этом районе, оставляют нетронутыми культуры на этих
полях.
А теперь давайте бросим взгляд на поле соседа. Здесь все сорняки уничтожены
гербицидами и культивацией. Почвенные животные и насекомые уничтожены ядами.
Почвенное органическое вещество и микроорганизмы начисто выжжены химическими
удобрениями. Летом вы увидите фермеров, работающих на полях в противогазах и длинных
резиновых перчатках. Рисовые поля, которые обрабатывались непрерывно в течение 1500 лет,
теперь опустошены и заброшены за время жизни одного поколения благодаря новой
земледельческой практике.
Четыре принципа
Первый – отказ от рыхления, то есть от вспашки, или переворачивания почвы. В течение
столетий фермер был уверен, что вспашка необходима для выращивания культур. Однако,
принцип «отказ от рыхления» является фундаментальным для натурального земледелия.
Почва рыхлит сама себя естественно благодаря проникновению корней растений и
активности микроорганизмов, мелких животных и земляных червей.
Второй – отказ от химических удобрений или приготовленного компоста (для удобрения
м-р Фукуока выращивает белый клевер, как бобовую покровную культуру; возвращает на
поля обмолоченную солому и добавляет немного птичьего навоза). Люди нарушают
естественную жизнь природы и затем как ни стараются, не могут залечить нанесённые раны.
Их неосторожная фермерская практика приводит к вымыванию из почвы необходимых
питательных веществ, а в результате – ежегодное истощение земли. Оставленная в покое,
почва поддерживает своё плодородие естественным путём» согласно с упорядоченным
циклом растений и животных.
Третий – отказ от прополки путём вспашки или обработки гербицидами. Сорняки
играют свою роль в создании почвенного плодородия и сбалансированного биологического
сообщества. Основной принцип таков: сорняки надо сдерживать, но не уничтожать.
Соломенная мульча, покров из белого клевера, подсеянного под культурные растения, и
временное затопление обеспечивают эффективный контроль сорняков на моих полях.
Четвёртый – отказ от химических средств защиты (м-р Фукуока выращивает зерновые
культуры без применения каких-либо химикатов. Некоторые плодовые деревья он иногда
обрабатывает эмульсией машинного масла для снижения численности насекомых. Он не
использует долгодействующие пестициды широкого спектра действия и не имеет
«программы» по пестицидам). С тех пор, как в результате неестественной практики вспашки
и удобрения культурные растения стали ослабленными, болезни и дисбаланс насекомых
стали громадными проблемами в сельском хозяйстве. Природа, оставленная нетронутой,
находится в совершенном равновесии. Вредоносные насекомые и болезни растений всегда

есть, но в природе они не распространяются в такой степени, которая требует применения
химикатов. Разумный подход к защите от болезней и вредителей – это выращивание сильных
растений в здоровой среде.

Культивация (вспашка, рыхление)
После вспашки естественная среда почвы изменяется до неузнаваемости. Последствия
этих действий преследуют фермера как кошмар в течение многих поколений. Например,
когда поднимают целину, очень сильные сорняки, такие как росичка и щавели, иногда
начинают доминировать на поле. Если эти сорняки укоренятся на поле, фермер окажется
перед лицом почти невыполнимой задачи – ежегодной прополки. Очень часто землю просто
забрасывают.
Сталкиваясь с подобными проблемами, фермер может найти только один разумный
выход – прекратить неестественную практику, которая является причиной возникновения
этих проблем. Фермер несёт также ответственность за исправление причинённого им вреда.
Вспашка почвы должна быть прекращена. Если проводить такие осторожные мероприятия
как разбрасывание соломы и посев белого клевера вместо того, чтобы применяя химикаты и
машины, вести войну на уничтожение, тогда среда начнёт постепенно возвращаться к
восстановлению естественного равновесия и даже трудноискоренимые сорняки могут быть
взяты под контроль.

Удобрения
В беседе с экспертами по почвенному плодородию я спрашивал: «Если поле не
обрабатывать и предоставить самому себе, плодородие почвы увеличится или будет
снижаться?» Они обычно размышляли некоторое время и затем обычно говорили что-то
вроде: «Ну, дайте подумать… Оно снизится. А может быть, не снизится. Если вы вспомните,
что при выращивании риса долгое время на одном и том же месте без удобрений, урожай
устанавливается на уровне около 22,6 ц/га. Почва становится ни обогащённой, ни
истощённой.
Эти специалисты имеют ввиду культивируемое, затопляемое поле. Если поле
предоставить само себе, плодородие увеличится. Органические остатки растений и
животных накапливаются и разлагаются на поверхности бактериями и грибами. С дождевой
водой питательные вещества проникают глубоко в почву, чтобы стать пищей для
микроорганизмов, дождевых червей и других мелких животных. Корни растений достигают
нижних слоёв почвы и возвращают питательные вещества обратно на поверхность.
Если вы хотите получить наглядное представление о естественном плодородии земли,
совершите когда-нибудь прогулку в дикую горную местность и посмотрите на гигантские
деревья, которые растут без удобрений и без вспашки. Плодородие нетронутой природы выше
всякого воображения.
Срубите естественный лесной покров, посадите японскую красную сосну и кедры и
через несколько поколений почва станет истощённой и открытой для эрозии. С другой
стороны, возьмите безлесные горы с глинистой красной почвой и посадите сосну или кедр с
почвенным покровом из клевера и люцерны. Когда зелёное удобрение (зелёное удобрение –
это покровные культуры, такие как клевер, вика, люцерна, которые улучшают и удобряют
почву) обогатит и разрыхлит почву, под покровом деревьев вырастут сорняки и кусты и
начнётся цикл обогащения и регенерации.
При выращивании сельскохозяйственных культур использование готовых удобрений
также не обязательно. Большей частью постоянный покров из зелёного удобрения и возврат
всей соломы и мякины в почву будет достаточным. Чтобы добавить навоз животных,
ускоряющий разложение соломы, я выпускаю в поле уток. Если выпустить в поле утят, когда
молодые ростки риса только появляются, то утки будут расти вместе с рисом. 10 уток
обеспечат количество навоза, необходимое на 0,1 га и помогут также сдерживать рост
сорняков.
Я делал это в течение многих лет, пока строительство национальной шоссейной дороги
не помешало уткам переходить дорогу и возвращаться обратно домой. Теперь для ускорения
разложения соломы я использую куриный помёт. В других районах утки и другие мелкие
пастбищные животные всё ещё могут быть использованы для удобрения полей навозом.
Внесение слишком большого количества удобрений может привести к неприятным
последствиям. Однажды я взял в аренду 0,5 га сразу после пересадки риса. Я спустил с
полей всю воду и не вносил никаких химических удобрений, используя только небольшое
количество куриного помёта. На четырёх полях растения развивались нормально, но на
пятом, что бы я ни делал, растения риса росли слишком густо и заражались бактериальными
болезнями. Когда я спросил владельца, в чём дело, он сказал, что в течение зимы
использовал поле для хранения куриного помёта.

Используя солому, зелёное удобрение и немного куриного помёта, можно получать
высокие урожаи совсем без компоста и коммерческих удобрений. В течение нескольких
десятилетий я не торопясь наблюдаю, как реагирует земля на натуральный способ
возделывания. И пока я наблюдаю, я получаю небывало высокие урожаи овощей, цитрусовых,
риса, озимых зерновых как подарок, так сказать, от естественного плодородия земли.

Как справиться с проблемой сорняков
Есть несколько ключевых моментов, которые надо запомнить для правильного
отношения к сорнякам.
Как только вы прекращаете вспашку, число сорняков резко падает. Видовой состав
сорняков на данном поле также изменяется.
Если семена очередной культуры высеяны, когда предшествующая культура ещё зреет в
поле, эти семена прорастут раньше сорняков. Озимые сорняки отрастают только после
уборки риса, но к этому времени озимые зерновые уже пошли в рост. Яровые сорняки
отрастают сразу после уборки ячменя или ржи, но к этому времени проростки риса уже
набрали силу. Распределяя сроки посевов таким образом, чтобы не было интервала между
следующими друг за другом культурами, мы даём зерновым большое преимущество перед
сорняками.
Если сразу после уборки всё поле покрыто соломой, прорастание сорняков
прекращается. Белый клевер, посеянный вместе с зерновыми как покровная культура, также
помогает держать сорняки под контролем.
Обычный способ борьбы с сорняками – это вспашка почвы. Но когда вы рыхлите почву,
семена, которые лежат на большой глубине и никогда бы сами не проросли, стимулируются
и получают шанс тронуться в рост. Кроме того, быстро прорастающие и быстро растущие
виды в этих условиях получают преимущество. Одним словом, можно сказать, что фермер,
который пытается бороться с сорняками путём вспашки почвы, почти буквально сеет семена
собственной неудачи.

Защита от «вредителей»
Надо признать, что есть ещё некоторые люди, которые думают, что если не использовать
пестициды, то их фруктовые деревья и полевые культуры будут уничтожены вредителями
прямо на их глазах. На самом деле, благодаря использованию пестицидов люди неизбежно
создают условия, при которых этот ни на чём не основанный страх может стать
реальностью.
Недавно японские красные сосны были сильно повреждены в результате вспышки
коревого долгоносика. Лесники используют теперь вертолёты, пытаясь остановить
распространение вредителя путём авиаобработки лесов. Я не отрицаю, что этот способ
эффективен на короткий промежуток времени, но я знаю, что есть другой путь.
Заражение долгоносиком, согласно последним исследованиям, происходит не прямым
путём, а является следствием активности нематод – посредников. Нематоды размножаются
внутри ствола, блокируют транспорт воды и питательных веществ и постепенно вызывают
засыхание и гибель сосны. Первопричина этого явления ещё не совсем ясна. Известно, что
нематоды питаются грибами внутри ствола дерева. Почему эти грибы начинают так
стремительно распространяться внутри дерева? Начинают ли грибы размножаться уже после
того, как нематоды появились? Или нематоды появляются потому, что уже есть грибы? Всё
сводится к вопросу, кто появился первым – грибы или нематоды?
Кроме того, есть ещё другой микроорганизм, о котором очень мало известно и который
всегда сопровождает грибы, и есть вирус токсичный для грибов. Эффект следует за
эффектом. Единственная вещь. которую можно сказать с уверенностью, это то, что сосны
засыхают в больших количествах.
Люди не знают, какова истинная причина заболевания сосен, не могут они также знать
основных последствий применения своих «лекарств». Если ситуация включает в себя
неизвестные факторы, то непродуманные меры только сеют семена для следующей большой
катастрофы. Нет, я не могу радоваться тому, что повреждение долгоносиком снижено
благодаря химической обработке. Использование химикатов – это наиболее инертный способ
разрешения проблемы такого типа и приведёт только к большим проблемам в будущем.
Эти четыре принципа натурального земледелия (без вспашки, без химического
удобрения или приготовления компоста, без прополки с помощью вспашки или гербицидов,
без зависимости от применения химикатов) соответствуют естественному порядку вещей и
ведут к восстановлению природного плодородия. Все мои пробы и ошибки направляются
этой основной мыслью. Это сердцевина моего метода выращивания овощей, зерновых и
цитрусовых.

Культурные растения среди сорняков
Много различных видов сорняков растёт на этих полях вместе с зерновыми и клевером.
Рисовая солома, разбросанная по полю последней осенью, уже разложилась и превратилась
в гумус. Урожай будет около 58 ц с гектара.
Вчера, когда профессор Кавасе, ведущий авторитет по пастбищным травам, и профессор
Хирое, изучающий древние растения, увидели на моём поле равномерное распределение
растений ячменя и зелёного удобрения, они назвали это чудом искусства. Местный фермер,
который ожидал увидеть мои поля полностью заросшими сорняками, был удивлён тем, что
ячмень так энергично растёт среди множества других растений. Технические эксперты
также приехали сюда, увидели сорняки, увидели водяной кресс и клевер, растущие повсюду,
и ушли, в удивлении качая головами.
Двадцать лет тому назад, когда я пропагандировал использование постоянного
клеверного покрова в плодовом саду, по всей стране на полях или в садах нельзя было найти
ни травинки. Видя мой сад, люди приходят к пониманию, что плодовые деревья могут
достаточно хорошо расти среди сорняков и трав. Сегодня сады, заросшие травами, стали
обычными в Японии, а сады без травяного покрова встречаются редко.
То же самое с полями зерновых. Рис., ячмень и рожь можно успешно выращивать на
полях весь год покрытых клевером и сорняками.
Давайте рассмотрим в главных деталях годичный распорядок посева и сбора урожая на
этих полях. В начале октября перед уборкой семена белого клевера и быстро растущих сортов
озимых зерновых разбрасываются среди зреющих побегов риса (белого клевера высевают
около 4,5 кг/га, озимых зерновых – 28,4–57,2 кг/га. Для неопытного фермера или для полей с
бедной уплотнённой почвой более безопасно высевать вначале больше семян. Когда почва
постепенно улучшится благодаря разложению соломы и зелёному удобрению и когда фермер
лучше познакомится с методом прямого высева без вспашки, то количество семян можно
уменьшить). Клевер и ячмень или рожь прорастают и дают побеги высотой 2–5 см к тому
времени, когда рис пора убирать. Во время уборки проростки топчутся ногами рабочих, но
восстанавливаются затем очень быстро. Когда обмолот закончен, рисовая солома
разбрасывается по полю.
Когда рис высевают осенью и оставляют непокрытыми семена, они часто поедаются
мышами и птицами, иногда загнивают. Поэтому я заключаю семена риса перед высевом в
маленькие глиняные капсулы.
Семена помещают на противень или в корзину и встряхивают круговыми движениями.
Сверху семена припудривают тонко размолотой глиной и время от времени смачивают мелко
распылённой водой. Таким образом, вокруг каждого семени образуются маленькие глиняные
капсулы около 1,2 см в диаметре.
Есть и другой метод приготовления капсул. Сначала семена риса замачивают в воде на
несколько часов. Семена вынимают из воды и смешивают с влажной глиной, перемешивая
руками или ногами. Затем глину пропускают через проволочную сетку (для цыплят), чтобы
разделить её на маленькие комочки. Комочки должны быть слега подсушены день или два до
тех пор, пока они не будут легко скатываться в ладонях в капсулы. В идеале в одной капсуле
должно быть одно семя. В один день возможно сделать достаточно капсул, чтобы засеять 1–2
гектара.

В зависимости от условий я иногда заключаю в капсулы перед высевом семена других
зерновых и овощей.
С середины ноября до середины декабря – хорошее время, чтобы разбросать капсулы с
семенами риса между молодыми растениями ячменя и риса, но их можно разбросать также и
весной (рис высевают в количестве 20–40 кг/га. Ближе к концу апреля м-р Фукуока
проверяет прорастание высеянных осенью семян и разбрасывает, если надо, больше капсул).
Тонкий слой куриного помёта распределяется по поверхности поля, чтобы ускорить
разложение соломы. На этом посевы этого года закончены.
В мае убирают озимые зерновые. После обмолота вся солома разбрасывается по полю.
Затем поле заливают водой на неделю или 10 дней. Это ослабляет сорняки и клевер и
даёт возможность рису прорасти через солому.
В июне и июле для растений достаточно одной дождевой воды, в августе через поля раз
в неделю пропускают свежую воду, но не дают ей застаиваться на поверхности почвы. А
скоро и уборка урожая.
Таков годовой цикл возделывания риса и озимых зерновых в натуральном земледелии.
Посев и уборка так тесно совпадают с естественными процессами в природе, что их можно
скорее воспринимать как естественные процессы, чем как сельскохозяйственную
технологию.
У одного фермера посев и разбрасывание соломы на площади 0,1 га занимает всего 1–2
часа. За исключением уборочных работ, для ухода за посевами озимых зерновых достаточно
одного человека, а два или три человека могут выполнять всю работу, необходимую для
выращивания риса, используя только традиционные японские орудия труда. Возможно, не
существует более лёгкого и простого способа выращивания зерновых. Он не включает
никаких операций, кроме разбрасывания семян и раскидывания соломы, но чтобы достичь
такой простоты, мне понадобилось более тридцати лет.
Этот способ земледелия разработан в соответствии с природными условиями японских
островов, но я думаю, что натуральное земледелие может быть введено также и в других
районах и для выращивания и других местных культур. В тех местах, где вода не так
доступна, например в горных районах, можно выращивать рис или другие зерновые, такие
как гречиха, сорго или просо. Вместо белого клевера в качестве покровной культуры можно
попробовать другие виды клевера, люцерну, вику или люпин. Натуральное земледелие
принимает ту единственно возможную форму, которая соответствует уникальным условиям
каждой отдельной местности.
При переходе к этому виду земледелия на первых порах может быть необходимо делать
небольшую прополку, вносить компост и производить обрезку деревьев, но каждый год эти
мероприятия должны постепенно уменьшатся. В конечном счёте, наиболее важный фактор –
это не техника выращивания, а скорее, состояние сознания фермера.

Земледелие и солома
Разбрасывание соломы можно считать довольно важным мероприятием, но для моего
метода выращивания риса и озимых зерновых – это одно из основных мероприятий. Оно
определяет всё – плодородие, прорастание, засорённость, защиту от воробьёв, водный
режим. И в практике, и в теории использование соломы в земледелии – решающий фактор.
Но не так просто убедить в этом людей.
Разбрасывание нерезаной соломы
Испытательный центр префектуры Окаяма теперь испытывает метод прямого посева
риса на 80 % своих полей. Когда я предложил им разбрасывать нерезаную солому, они,
очевидно, сочли этот способ неверным и провели эксперимент, нарезав её механическим
резаком. Когда я приехал несколько лет назад посмотреть эти испытания, я увидел, что поля
разделены на три части. На первой используют нарезанную солому, на второй используют
ненарезанную солому и на третьей – совсем не используют солому. Это как раз то, что я сам
делал в течение долгого времени, и поскольку нерезаная солома работала лучше всего, я
рекомендовал использовать нерезаную.
М‑р Фуджии, преподаватель из Ясукской Высшей Сельскохозяйственной Школы в
префектуре Симане, хотел попробовать прямой посев и приехал посмотреть мою ферму. Я
предложил ему разбрасывать по полю нерезаную солому. Он вернулся на следующий год и
рассказал, что опыт не удался. Выслушав внимательно его сообщение, я понял, что он
раскладывал рисовую солому равномерно и аккуратно, как садовую мульчу. Если делать так,
то семена ячменя совсем не прорастут. То же самое с соломой ржи и ячменя. Если её
раскладывать очень ровным слоем, побегам риса будет трудно пробиться через него. Лучше
всего просто раскладывать солому вокруг так, как если бы она падала естественно.
Рисовая солома работает хорошо как мульча для озимых зерновых, а солома озимых
зерновых работает как самая лучшая мульча для риса. Я хочу, чтобы это было хорошо понято.
Если разбросать по полям свежую рисовую солому, то молодые растения риса могут быть
заражены некоторыми специфическими для этой культуры болезнями, возбудители которых
находятся в свежей соломе. Эти болезни риса не распространяются на озимые зерновые.
Свежая рисовая солома безопасна для других зерновых, так же как гречишная солома и
солома других видов зерновых может быть использована для риса и гречихи. По той же
причине свежая солома озимых зерновых, таких как пшеница, рожь и ячмень, не может быть
использована как мульча для других озимых зерновых, так как служит источником болезней.
Вся солома и мякина, которые остаются после обмолота урожая, должны быть
возвращены на поле.
Солома обогащает почву
Солома поддерживает почвенную структуру и обогащает почву, так что химическое
удобрение становится ненужным. Но этот эффект проявляется только при условии

выполнения принципа «отказ от вспашки». Мои поля, может быть, единственные в Японии,
которые не были вспаханы в течение более 20 лет, и тем не менее, качество почвы
улучшалось с каждым сезоном. Я мог бы сказать, что верхний слой, обогащённый гумусом,
за эти годы увеличился до толщины более 10 см. Это, в основном, результат того, что в почву
возвращалось всё, выросшее на поле, за исключением зерна.
Нет необходимости делать компост
Нет необходимости делать компост. Я не говорю, что вам совсем не нужен компост,
нужно только избегать трудоёмких операций при его изготовлении. Если солому оставить
лежать на поверхности поля весной или осенью и покрыть её тонким слоем куриного или
утиного помёта, то за шесть месяцев она полностью разложится.
Чтобы приготовить компост обычным способом, фермер работает как сумасшедший на
жарком солнце, нарезая солому, добавляя воду и глину, перемешивая компостную кучу и
перевозя её в поле. Он проходит через все эти испытания так как думает: «Это самый лучший
способ». Я бы с большим удовольствием видел людей, разбрасывающих по своим полям
солому и мякину или древесную стружку.
Путешествуя по железной дороге Токайдо в Западной Японии, я заметил, что фермеры
используют не всю солому. Они следуют указаниям экспертов, которые рекомендуют
определённые нормы внесения соломы на единицу площади. Но почему эксперты не скажут,
что надо вернуть на поле всю солому? Глядя из окна вагона, я мог видеть фермеров, которые
срезали и разбрасывали по полю около половины всего количества соломы, а остальное
оставили на обочине поля гнить под дождём.
Если бы все фермеры Японии начали возвращать на поля всю солому, то результатом
было бы громадное количество компоста, возвращённого земле.
Прорастание
В течение столетий фермеры с величайшей заботливостью готовили грады для семян,
чтобы вырастить крепкие, здоровые проростки риса. Маленькие гряды содержались в таком
порядке, как если бы это был семейный алтарь. Почву рыхлили, добавляли в неё песок и золу
от сожжённой рисовой мякины, и молились, чтобы проростки хорошо росли. Поэтому
вполне понятно, что многие крестьяне в округе думали, что я сошёл с ума, когда я стал
разбрасывать семена риса по ещё не убранным растениям озимых зерновых с сорняками и
раскиданными всюду клочьями разлагающейся соломы. Конечно, семена хорошо прорастают,
если они высеяны в хорошо подготовленную рыхлую почву. Но если идут дожди и поле
превращается в топь, вы не сможете пройти по нему и посев приходится отложить. С этой
точки зрения метод без вспашки более надёжен, но зато возникают проблемы с мелкими
животными: кротами, мышами, сверчками, слизнями, которые любят есть семена. Эту
проблему решают глиняные капсулы, защищающие семена.
Обычный метод посева озимых зерновых – это посеять семена и затем покрыть их
почвой. Если семена окажутся заделанными слишком глубоко, они загнивают. Я обычно
бросаю семена в маленькие отверстия в почве или в борозды, не прикрывая их почвой. Но

вначале с обоими этими методами у меня было много неудач.
Позже я стал более ленив и вместо проделывания борозд и просверливания отверстий в
почве, я стал заключать семена в глиняные капсулы и разбрасывать их прямо по поверхности
земли. Семена лучше всего прорастают на поверхности, где они имеют достаточно
кислорода. Я обнаружил, что если эти капсулы покрыть соломой, семена прорастают хорошо
и не загнивают даже в очень дождливые годы.
Солома помогает справиться с сорняками и воробьями.
В идеальном случае с 0,1 га получают около 4 ц ячменной соломы. Если всю эту солому
разбросать по полю, поверхность будет почти полностью покрыта. Это помогает держать
под контролем даже такие трудноискоренимые сорняки как росичка, которая представляет
наиболее сложную проблему при методе прямого высева и без вспашки.
Воробьи причинили мне много неприятностей. Прямой посев не даёт результата, если
нет реального способа защиты от птиц. И есть много районов, где метод прямого высева
распространяется медленно именно по этой причине. Многие из вас могли столкнуться с
проблемой воробьёв, и вы знаете, что я имею в виду.
Я помню времена, когда эти птицы следовали прямо за мной и подбирали все семена,
которые я посеял даже ещё до того, как я успевал засеять всё поле. Я пытался использовать
пугала и сети и подвески из грохочущих консервных банок, но ничто не помогало достаточно
хорошо. А если один из этих методов начинал хорошо работать, то его эффективность
снижалась через год или два. Мой собственный опыт показал, что проблема воробьёв может
быть решена наиболее эффективно путём высева семян в то время, когда предшествующая
культура ещё в поле, так что семена спрятаны под травой и клевером, а после уборки
культуры почва покрывается соломой.
Я делал много ошибок, когда экспериментировал в течение многих лет, и я пережил
самые разнообразные неудачи. Я, может быть, лучше, чем кто-либо другой в Японии знаю,
какие могут быть ошибки при возделывании сельскохозяйственных культур. Когда я в первый
раз получил положительный результат при выращивании риса и озимых зерновых методом
без вспашки, я чувствовал такую же радость, какую должен был чувствовать Колумб, когда
он открыл Америку.
Выращивание риса в сухом поле
В начале августа рис на полях соседей уже вырос по пояс, а на моих полях растения
почти в два раза ниже. Люди, которые приезжают сюда к концу июля, всегда настроены
скептически и спрашивают:
«Фукуока-сан, этот рис вырастет потом нормальным?» «Конечно, – отвечаю я, – не надо
беспокоиться».
Я не стараюсь получать высокие, быстро растущие растения с большими листьями.
Наоборот, я стараюсь, чтобы растения были как можно компактнее. Поддерживайте
небольшие размеры метёлки, не перегружайте растение и дайте ему приобрести его
естественную форму рисового растения. На затопляемых полях растения риса обычно
достигают высоты 0,9–1,2 м и имеют большие листья. Создаётся впечатление, что они могут
дать много зерна. Но на самом деле это только крепкий облиственный побег. Он производит
много крахмала, но эффективность его низка, и так много энергии расходуется на

вегетативный рост, что на зерно остаётся не так уж много. Например, если высокие
переросшие растения дают урожай соломы 9 ц, то урожай зерна будет 4,5–5,4 ц (на 0,1 га).
Маленькие растения риса, как те, что растут на моих полях, на 9 ц соломы дают 9 ц риса. В
хороший год урожай риса с моих растений достигает 10,8 ц зерна на 9 ц соломы, что на 20 %
больше, чем вес соломы.
Рисовые растения, выращенные на сухом поле не будут очень высокими. Солнце
освещает их равномерно, достигая основания растения и нижних листьев. 6,2 см2 площади
листьев достаточно для питания 6 зёрен риса. Три или четыре небольших листа более, чем
достаточно, чтобы выкормить 100 зёрен риса в метёлке. Я сею немного густо и в результате
имею около 250–399 плодоносящих побегов (20–25 растений) на 0.84 м. Если вы имеете
много побегов и не будете стараться вырастить большие растения, вы получите большой
урожай без труда. Это правило верно также для пшеницы, ржи, гречихи, проса, овса и других
зерновых.
При традиционном способе выращивания риса слой воды в несколько сантиметров
поддерживается в чеках в течение всего вегетационного периода. Фермеры выращивали рис в
воде в течение стольких столетий, что большинство людей думают, что другого способа не
может быть. Культивируемые сорта влаголюбивого риса растут относительно хорошо на
затопленных полях, но такой способ выращивания не очень благоприятен для растений. Рис.
растёт лучше всего, когда содержание воды в почве составляет 60–80 % от
водоудерживающей способности. Если поле не затоплено, растения развивают более
сильные корни и очень устойчивы к болезням и вредителям,
Основная причина выращивания риса на затопленных полях – это необходимость
подавлять сорняки, создавая условия среды, в которых может выжить только ограниченное
число видов сорняков. Но те, которые выживают, приходится вырывать руками или подрезать
мотыгой. Согласно традиционному методу эта кропотливая и трудная, надрывающая спину
работа должна быть проведена несколько раз в каждый сезон.
В июне, в период муссонов я держу воду на поле приблизительно в течение одной
недели. Только очень немногие из неустойчивых к затоплению видов сорняков могут
пережить даже такой короткий период без кислорода, и клевер также страдает и становится
жёлтым. Идея заключается не в том, чтобы убить клевер, а в том, чтобы ослабить его и тем
самым дать возможность проросткам риса укрепиться. Когда вода спущена, клевер
оправляется и разрастается, покрывая поверхность поля под покровом растений риса. После
этого я почти ничего не делаю для поддержания водного режима. В первую половину сезона
я совсем не орошаю поля. Даже в годы с очень небольшим количеством осенних дождей,
почва под слоем соломы и зелёного удобрения остаётся влажной. В августе я иногда
пропускаю воду на поля, но ни когда не даю ей застаиваться.
Если вы покажете фермеру растение риса с моего поля, он немедленно поймёт, что оно
выглядит так, как должно выглядеть растение риса и что это его идеальная форма. Он
поймёт, что растение выращено без пересадки проростков, без затопления и без применения
химических удобрений. Любой фермер может сказать это как нечто само собой
разумеющееся, глядя на общую форму растения, форму корней и длину междоузлии на
главном стебле. Если вы понимаете, что такое идеальная форма, то ваша задача – вырастить
растение такой формы в специфических условиях вашего собственного поля.
Я не согласен с идеей профессора Матсусима, что четвёртый лист от верхушки растения
должен быть самым длинным. В некоторых случаях вы получаете лучшие результаты, если

самый длинный второй или третий лист. Если рост задерживается, когда растение ещё
молодое, то верхний лист или второй лист вырастают самими длинными и всё же вы
получаете большой урожай.
Теория профессора Матсусимы создана на основании экспериментов с ослабленными
растениями риса, выращенными с применением удобрений на посевных грядках и затем
пересаженных. Мой рис, наоборот, был выращен в соответствии с естественным жизненным
циклом рисового растения так, как если бы он рос диким. Я терпеливо жду, когда растение
разовьётся и созреет в соответствии со своим собственным темпом развития.
В последние годы я испытывал старый сорт богатого клейковиной риса с юга. Каждое
семя, посеянное осенью, даёт в среднем 12 побегов и 250 зёрен на метёлку. Я думаю, что с
этим сортом я в один прекрасный день смогу собрать урожай близкий к теоретически
возможному при данной солнечной энергии, которую получает моё поле. На некоторых
участках моих полей урожай в 73,5 ц с гектара риса этого сорта уже стал реальностью.
С точки зрения сомневающегося специалиста мой метод выращивания риса можно
считать ненадёжным и дающим неустойчивый результат. «Если эксперимент продлится
дольше, то неизбежно возникнут определённые проблемы», – могут сказать они. Но я
выращивал этим способом рис свыше 20 лет. Урожаи продолжают повышаться, и почва с
каждым годом становится плодороднее.

Плодовые деревья
На склонах холмов недалеко от моего дома я выращиваю также несколько сортов
цитрусовых. После войны, когда я впервые стал заниматься фермерством, я начал с 0,7 га
цитрусового сада и 0,15 га рисовых полей, но теперь одни цитрусовые сады занимают
площадь 5 га. Я пришёл на эту землю и взял себе окружающие заброшенные склоны холмов.
Затем я руками расчистил их.
Сосновые деревья на некоторых склонах были вырублены за несколько лет до меня, и
всё, что я сделал, – это выкопал ямы вдоль намеченных рядов и посадил саженцы цитрусовых
деревьев. Между тем, на вырубках стали появляться новые побеги и через некоторое время
начали буйно разрастаться японская пампасная трава, императа цилиндрическая и
папоротник орляк. Саженцы цитрусовых деревьев затерялись в путанице диких растений.
Большую часть побегов сосны я срезал, но часть оставил расти для защиты от ветра.
Затем я скосил заросли растений и трав, покрывающих поверхность почвы, и посадил
клевер. Через б или 7 лет цитрусовые деревья наконец принесли плоды. Я удалил часть
земли на склонах, чтобы образовать террасы, и теперь мой сад стал немного отличаться от
любого другого сада.
Конечно, я придерживался принципов «отказ от вспашки», «отказ от использования
химических удобрений» и «отказ от использования инсектицидов и гербицидов». Я
наблюдал интересное явление: вначале, когда саженцы росли под покровом отрастающих
лесных деревьев, не было никаких признаков повреждения насекомыми типа восточной
цитрусовой щитовки. Когда заросли трав и побеги деревьев были скошены, участок стал
менее диким и больше походим на сад. Только тогда появились вредители.
Лучше всего дать садовому дереву с самого начала расти свободно в соответствии с его
естественной формой. Тогда дерево будет плодоносить каждый год и не будет необходимости
обрезки. Цитрусовое дерево имеет тот же тип роста, что кедр и сосна, а именно – один
центральный прямой ствол с ветвями, отходящими от него в очередном порядке. Конечно,
разные сорта цитрусовых не имеют точно одинаковый размер и форму роста. Сорта Хассаку
и Шеддок вырастают очень высокими, мандариновое дерево зимний Иншу – очень
невысокое и приземистое, ранние сорта мандаринов Сатсума характеризуются небольшими
размерами деревьев. Но все они имеют один центральный ствол.

Не убивайте естественных хищников.
Я думаю, всякий знает, что поскольку наиболее распространённые вредители
цитрусовых садов – различные виды щитовок – имеют естественных врагов, нет
необходимости применять инсектициды, чтобы держать их под контролем. Одно время в
Японии использовался инсектицид фузол. Естественные хищники были полностью
уничтожены и, возникшие в результате этого проблемы, до сих пор сохраняются во многих
префектурах. Я думаю, что на основе этого печального опыта многие фермеры осознали, что
нежелательно уничтожать хищных насекомых, так как результатом является ещё большее
повреждение садов.
Если же ко времени появления клещей и в середине лета развести в 200–400 раз
машинное масло, относительно безвредное для хищников, и слегка опрыснуть им сады и
после этой обработки предоставить сообществу насекомых вернуться к своему
естественному равновесию, то обычно проблема вредителей будет решаться дальше уже без
нашего участия. Этот способ не работает, если в июне или июле уже были применены
фосфорорганические пестициды, поскольку хищники уже убиты этими обработками.
Я не пропагандирую применение так называемых безвредных «органических»
опрыскиваний таких как настойка чеснока с солью или эмульсия машинного масла, я также
не в восторге от применения чужих завезённых видов хищников для контроля вредных
насекомых. Деревья ослабляются и подвергаются атакам насекомых в той степени, в которой
они отклоняются от своей природной формы. Если деревья выращиваются неестественным
для них образом, и в таком состоянии будут оставлены без ухода, то в результате ветви
образуют беспорядочные переплетения и будут повреждены насекомыми. Раньше я уже
рассказывал, каким образом я уничтожил несколько гектаров цитрусовых деревьев. Но если
деревья постепенно исправлять, то они в какой-то степени вернутся к своей естественной
форме. Деревья станут более сильными и меры борьбы с насекомыми станут не нужны. Если
дерево заботливо посажено и с самого начала предоставлено свободе следовать своей
природе, то нет необходимости ни в какой обрезке и опрыскивании. Если саженцы деревьев
подверглись обрезке и их корневая система была повреждена в питомнике до того, как они
были пересажены в сад, в этом случае делать обрезку необходимо.
Чтобы улучшить садовую почву, я пытался сажать разные виды деревьев. Среди них была
акация Моришима. Это дерево растёт круглый год, образуя всё время новые почки. Тли,
которые кормятся на этих почках, начинаются размножаться в громадных количествах.
Божьи коровки питаются этими тлями и скоро тоже начинают увеличиваться в числе. После
того, как божьи коровки уничтожили всех тлей на акации, они поднимаются на цитрусовые
деревья и начинают поедать других насекомых, в том числе клещей, цитрусовых щитовок,
австралийских желобчатых червецов.
Выращивание плодов без обрезки, удобрения или химических обработок возможно
только в естественной среде.

Почва плодового сада
До сих пор я не говорил о том, что улучшение почвы – это главная забота при
возделывании сада. Если вы используете химические удобрения, деревья вырастают
большими, но почва год от года истощается. Химические удобрения лишают почву её
жизненных сил. Если удобрения используют в течение жизни одного поколения, за это время
почва значительно ухудшается.
Нет более мудрого направления в земледелии, чем курс на улучшение почвы. Двадцать
лет назад склоны этой горы представляли голую красную глину, такую твёрдую, что
невозможно было воткнуть в неё лопату. Большая часть земли в этой местности была такая
же. Крестьяне выращивали картофель, пока почва не истощалась, а затем забрасывали поля.
Поэтому, прежде чем выращивать здесь цитрусовые и овощи, я постарался восстановить
здесь плодородие почвы.
Давайте поговорим о том, каким образом я восстановил плодородие этих голых склонов
горы. После войны поощрялась глубокая вспашка цитрусовых садов и выкапывание траншей
для внесения органического вещества. Когда я вернулся из Испытательного Центра, я
пытался делать это в моём собственном саду. Через несколько лет я пришёл к заключению,
что этот метод не только физически изнурителен, но и совершенно бесполезен с точки
зрения улучшения почвы.
Сначала я закапывал солому и папоротники, которые приносил с горы. Переноска
тяжестей в 38 кг и более была тяжёлым испытанием, но через 2 или 3 года такой работы я не
мог набрать даже горсти гумуса. Траншеи, которые я делал, чтобы закопать органический
материал, осели и превратились в открытые ямы.
Затем я пытался закапывать древесные стволы. Кажется, что солома самое хорошее
средство для улучшения почвы, но если судить по образующемуся количеству гумуса,
древесина лучше. Всё это хорошо, пока есть деревья, которые можно срубить. Но тем, у кого
нет деревьев поблизости, можно выращивать деревья прямо в саду, а не везти их издалека.
В моём саду есть сосны и кедры, несколько грушевых деревьев, хурма, мушмула,
японские вишни и много других видов местных плодовых деревьев, растущих среди
цитрусовых. Одно из наиболее интересных деревьев, хотя и не местного происхождения, это
акация Моришима. Это то самое дерево, которое я упоминал ранее, рассказывая о божьих
коровках и защите хищных насекомых, У него твердая древесина, цветы привлекают пчёл и
листья идут на корм для скота. Акация Моришима помогает предотвращать повреждение
сада насекомыми, защищает сад от ветра, а бактерии ризобиум, живущие на корнях,
обогащают почву азотом.
Это дерево было интродуцировано в Японии из Австралии несколько лет назад. Оно
растёт быстрее, чем любое другое дерево, которое я когда-либо видел. За несколько месяцев
оно образует глубокие корни, а через 6–7 лет оно достигает высоты телефонного столба.
Добавим к этому, что это дерево – фиксатор азота. Поэтому если посадить 6–7 деревьев на
0,1 га, то улучшение почвы захватывает даже глубокие почвенные горизонты, и нет
необходимости гнуть спину и таскать брёвна с гор.
Для улучшения поверхностного слоя почвы я посеял на голом грунте смесь белого
клевера и люцерны. Прошло несколько лет, прежде чем они смогли укрепиться, но в конце
концов они разрослись и покрыли склоны холмов. Я посадил также японскую редьку

(дайкон). Корни этого мощного растения проникают глубоко в почву, внося в неё
органическое вещество и проделывая каналы для циркуляции воды и воздуха. Она
воспроизводится очень легко и раз посеяв её, вы можете перестать заботиться о ней.
Когда почва становится богаче, начинают появляться сорняки. Через 7 и 8 лет клевер
почти исчезает в зарослях сорняков, поэтому я разбрасываю немного больше семян клевера
после того, как в конце лета скашиваю сорняки. В результате образования плотного
растительного покрова из клевера и сорняков через 25 лет поверхностный слой садовой
почвы, который раньше был голой глиной, стал рыхлым тёмно-окрашенным и обогатился
дождевыми червями и органическим веществом.
С помощью зелёного удобрения, обогащающего верхний слой почвы, и корней акации
Моришима, улучшающими глубокие почвенные слои, вы вполне можете обойтись без
удобрений, а также отпадёт необходимость рыхления почвы между плодовыми деревьями. С
высокими деревьями для защиты от ветра, цитрусовыми деревьями в середине и зелёным
удобрением внизу сад может позаботиться о себе, и уход за ним значительно облегчается.

Выращивание овощей как диких растений
Давайте поговорим теперь о выращивании овощей. Некоторые семьи для снабжения
своей кухни овощами используют огород на заднем дворе, другие выращивают овощи на
открытых, неиспользуемых землях.
Об огороде на заднем дворе достаточно сказать, что вы получите хорошие овощи, если
посадите их в соответствующее время в почву, удобренную органическим компостом и
навозом. Метод выращивания овощей для собственного стола в старой Японии хорошо
гармонировал с естественным образом жизни. Дети играли на заднем дворе между
плодовыми деревьями. Свиньи поедали отбросы с кухни и копались в земле в поисках
корней. Собаки лаяли и играли во дворе, и фермер сеял семена овощей в богатую почву.
Гусеницы и насекомые росли вместе с овощами, куры поедали гусениц и откладывали яйца
для питания детей.
Типичная японская семья выращивала овощи таким образом не далее, как двадцать лет
тому назад.
Заболевания растений предотвращали тем, что выращивали традиционные культуры в
соответствующее для них время, поддерживая здоровье почвы путём возвращения в неё всех
органических остатков и путём чередования культур. Вредных насекомых собирали руками и
их поедали куры. На юге Шикоку был распространён вид кур, которые поедали гусениц и
насекомых, не выкапывая корни и не повреждая растений.
Некоторые люди могут скептически относиться к использованию навоза животных и
человеческих отходов, думая что это примитивно или грязно. Сегодня люди хотят «чистые»
овощи, поэтому фермеры выращивают их в теплицах совсем без почвы. Гравийная культура,
песчаная культура и гидропоника становятся всё более популярными. Овощи выращивают на
химическом питании и на свету, который фильтруется через виниловое покрытие. Довольно
странно, что люди считают эти химически выращенные овощи «чистыми» и безопасными
для питания. Продукты, выращенные на почве, сбалансированной деятельностью червей,
микроорганизмов и разложившимся навозом животных, – самая чистая и самая богатая
пища.
При выращивании овощей «полудиким» способом на пустующих участках, берегах рек
или заброшенных землях, моя идея заключается в том, чтобы разбросать семена и дать
овощам расти вместе с сорняками. Я выращиваю мои овощи на склонах горы между
цитрусовыми деревьями.
Очень важно знать правильное время посадки. Для весенних овощей правильное время,
когда зимние сорняки отмирают, а летние ещё не проросли (этот метод выращивания овощей
разработан м-ром Фукуока путём опытов и экспериментов в соответствии с местными
условиями. Там, где он живёт, весной всегда идут дожди и климат достаточно тёплый, чтобы
выращивать овощи круглый год. За многие годы он пришёл к знанию того, какие овощи
можно выращивать вместе с сорняками и какой уход требует каждый вид овощей. В
большинстве районов Северной Америки специфический метод м-ра Фукуока для
выращивания овощей неприменим. Каждый фермер, который хочет выращивать овощи
полудиким способом, должен разработать свой метод, пригодный для его почвы и местных
видов овощей). Для осеннего посева семена разбрасывают, когда летние травы отмирают, а
зимние сорняки ещё не появились.

Лучше всего подождать, пока начнутся дожди, которые продолжаются несколько дней.
Скосите сорняки и высевайте семена овощей. Нет необходимости покрывать их почвой, так
как вы покроете их как мульчей скошенными сорняками и таким образом спрячете их от
птиц и кур, пока они не начнут прорастать. Обычно сорняки надо скашивать два или три
раза, чтобы дать возможность проросткам овощей укрепиться, но иногда достаточно одного
раза.
Если сорняки и клевер образуют не очень плотный покров, вы можете просто
разбросать семена овощей. Куры съедят часть семян, но большая часть прорастёт. Если вы
сажаете в ряды или в борозды, то жуки и другие насекомые могут уничтожить значительную
часть семян, так как они склонны двигаться по прямой линии. Мой опыт свидетельствует о
том, что лучше всего разбрасывать семена тут и там без определённого порядка.
Овощи, выращенные таким образом, получаются лучше, чем думает большинство людей.
Если они дадут побеги раньше, чем сорняки, то позднее они не дадут сорнякам задавить их.
Есть некоторые виды овощей, например шпинат и морковь, которые прорастают медленно.
Если вы замочите семена на 1–2 дня, а затем заключите их в глиняные капсулы, то и для
этих овощей проблема будет решена.
Если сделать более загущённый посев, то японская редька, репа и различные листовые
осенние овощи будут в состоянии успешно конкурировать с зимними и ранне-весенними
сорняками. Небольшое количество этих овощей оставляют неубранными и они
обсеменяются и возобновляются сами год за годом. Они имеют неповторимый аромат и
представляют собой очень интересное блюдо.
Это удивительное зрелище – вид многих знакомых овощей пышно разрастающихся на
склоне горы. Японская редька и репа растут наполовину в почве и наполовину над
поверхностью почвы. Морковь и садовый лопух (молодые корни и побеги садового лопуха
пригодны в пищу [БСЭ]) часто вырастают короткими и толстыми с большим количеством
корешков и, я думаю, что их терпкий слегка горьковатый вкус напоминает их дикого
предшественника. Чеснок, японский перламутровый лук и китайские виды лука, будучи раз
посаженными затем сами возобновляются год за годом.
Бобовые лучше всего сажать весной. Вигна китайская и фасоль легко выращиваются и
дают высокие урожаи. При выращивании гороха, фасоли адзуки, сои очень важно обеспечить
их быстрое прорастание. При недостатке влаги они прорастают медленно, и вы должны всё
время защищать их от птиц и насекомых.
Томаты и баклажаны недостаточно сильны, чтобы конкурировать с сорняками на стадии
проростков и поэтому их надо начинать выращивать на специальной грядке для рассады и
затем пересаживать. Дайте томатам стелиться по земле, не привязывая их к опорам. Тогда из
узлов на главном стебле образуются корни и вырастут новые плодоносящие побеги. Что
касается огурцов, то предпочтение следует отдать стелющимся видам. Вы должны
заботиться о молодых растениях, периодически подрезая сорняки, но позже растения
становятся достаточно сильными, чтобы сорняки их не подавляли. Воткните в землю
бамбуковые шесты или ветки деревьев и огурцы будут оплетать их и подниматься по ним.
Благодаря этому плоды не будут лежать на земле и гнить. Этот метод выращивания огурцов
годится также для дынь и тыкв.
Картофель и таро (тропическое многолетнее растение семейства арендных. Крупные
клубни употребляют в пищу) очень сильные растения. Однажды посаженные, они будут сами
возобновляться на одном и том же месте каждый год и никогда не будут задавлены

сорняками. Во время уборки оставьте в почве несколько растений. Если почва очень плотная,
то сначала посадите японскую редьку. Она своими разросшимися корнями разрыхлит и
смягчит почву и через несколько сезонов на этом месте можно будет выращивать картофель.
Я обнаружил, что белый клевер сдерживает разрастание сорняков. Он образует плотный
покров и подавляет даже такие сильные сорняки как полынь и росичка. Если клевер посеять
вместе с семенами овощей, он будет играть роль живой мульчи, обогащая почву и
поддерживая в ней влажность и хорошую аэрацию.
Чтобы клевер не мешал овощам важно выбрать правильное время для его посева.
Наиболее благоприятное время – конец лета или осень. Развившиеся в течение холодных
месяцев корни, весной дадут клеверу преимущество перед однолетними травами. Клевер
развивается хорошо, если его посеять ранней весной. Семена клевера можно разбрасывать
произвольно или высевать рядами, расположенными на расстоянии 30 см друг от друга.
После того, как клевер укоренится, у вас не будет необходимости сеять его в течение 5–6
последующих лет. Главная цель выращивания овощей этим полудиким способом – это
выращивание овощей в наиболее естественных условиях на участках земли, которые не
годятся ни для чего другого. Если вы попытаетесь улучшить технику выращивания или
получить более высокий урожай, вас ждёт неудача. В большинстве случаев причиной неудачи
будут вредители или болезни. Если различные виды трав и овощей будут вперемешку
высажены среди естественной растительности, повреждение насекомыми и болезнями будет
минимальным и не будет необходимости применять ядохимикаты или руками собирать
насекомых.
Для того, чтобы успешно применять этот метод, важно хорошо знать годичный цикл и
особенности развития сорняков и трав. Виды и состояние сорняков могут рассказать вам о
свойствах почвы и недостатке тех или иных питательных веществ,
В моём саду я выращиваю садовый лопух, капусту, томаты, морковь, горчицу, бобы и
многие другие травы и овощи этим полудиких способом.

Можно ли отказаться от химикатов?
Рисоводство в Японии сегодня стоит на перепутье. Фермеры и специалисты находятся в
смущении, не зная какой путь им выбрать: продолжать выращивать рассаду или перейти к
прямому посеву, и если выбрать последнее, то пахать почву или нет. В течение последних 20
лет я говорил, что метод прямого высева без вспашки постепенно докажет свои
преимущества. Скорость, с которой прямой высев уже распространился в префектуре
Окаяма, хорошее доказательство этому.
Однако есть люди, которые говорят, что поворот к нехимическому сельскому хозяйству
немыслим с точки зрения снабжения продовольствием всей страны. Они говорят, что
химические обработки необходимы для защиты риса от его трёх основных болезней: гниль
стебля, пирикуляриоз и пятнистость листьев. Но если фермер перестанет использовать
слабые, «улучшенные» сорта риса, перестанет вносить в почву слишком много азота и
уменьшит количество воды для орошения, так чтобы у риса развилась сильная корневая
система, эти болезни исчезнут и опрыскивание химическими препаратами станет не
нужным.
Вначале красная глинистая почва на моих полях была непригодна для выращивания
риса. Растения болели бурой пятнистостью листьев. Но по мере того, как повышалось
плодородие почвы, бурая пятнистость появлялась всё реже. Со временем она исчезла совсем.
С вредителями наблюдается аналогичная ситуация. Самое главное – не убивать
естественных хищников. Проблема вредителей возникает тогда, когда поля постоянно
находятся под водой или орошаются застойной или загрязнённой водой. Наибольший вред
причиняет осенняя и весенняя рисовая цикадка, численность которой можно значительно
снизить, убрав воду с полей.
Весенняя рисовая цикадка, зимующая в сорняках, может стать носителем вируса. Если
это происходит, то в результате можно потерять 10–20 % урожая от вирусных болезней риса.
Однако, если химикаты не применяются, то на полях будет много пауков и они избавят вас от
этой заботы. Пауки чувствительны к малейшему вмешательству человека, и это всегда надо
иметь в виду.
Большинство людей думает, что если отказаться от химических удобрений и
инсектицидов, то урожаи значительно снизятся. Эксперты по насекомым-вредителям
установили, что в первый год после прекращения применения инсектицидов потери
составили около 5 %. Вероятно, не будет большой ошибкой считать, что ещё 5 % потерь
будут следствием отказа от химических удобрений.
Таким образом, если прекратить затопление рисовых полей и отказаться от химических
удобрений и пестицидов, рекомендуемых Сельскохозяйственной Кооперацией, потери за
первый год составят в среднем около 10 %. Восстановительные силы природы значительно
превосходят наше воображение и после некоторого снижения урожаи начнут повышаться и
постепенно превзойдут первоначальный уровень.
Когда я работал на Опытной станции в Коти, я проводил опыты по предотвращению
распространения стеблевого сверлильщика. Эти насекомые проникают в стебель риса и
кормятся там, в результате чего побег белеет и засыхает. Метод определения повреждения
прост: вы подсчитываете число побелевших побегов. На 100 растений может быть от 10 до
20 белых стеблей. При сильных повреждениях, когда кажется, что вся культура погибла, на

самом деле число погибших растений составляет 30 %.
Чтобы определить степень повреждения риса стеблевым сверлильщиком, одно поле
риса было обработано инсектицидами, другое оставлено необработанным. Когда подсчитали
результаты, то оказалось, что необработанное поле с большим количеством засохших стеблей
дало больший урожай. Вначале я сам этому не мог поверить и думал, что это ошибка
эксперимента. Но цифры оказались точными, поэтому я продолжал исследование.
Оказалось, что нападая на более слабые растения, стеблевой сверлильщик производит
как бы прореживание. Благодаря гибели части стеблей для оставшихся растений остаётся
больше пространства. Солнечный свет может проникать до нижних листьев. В результате
оставшиеся растения риса вырастают более сильными, дают большее число плодоносящих
побегов и дают больше зерна на одну метёлку. Если густота побегов очень высока, и
насекомые не могут уничтожить излишек, растения выглядят вполне здоровыми, но во
многих случаях дают более низкий урожай.
Читая доклады Опытных станций, вы можете найти в них результаты испытаний
практически всех химических препаратов, имеющихся в списке. Но обычно мало кому
известно, что в доклады попадает только половина этих результатов. Конечно, это не значит,
что данные намеренно скрывают, но если результаты публикуются химическими
компаниями в качестве рекламы препаратов, это то же самое, как если бы отрицательные
результаты были скрыты. Результаты, которые свидетельствуют о более низком урожае, как в
случае со стеблевым сверлильщиком, рассматриваются как противоречивые и исключаются.
Конечно, есть случаи, когда уничтожение насекомых приводит к увеличению урожая, но есть
и другие случаи, когда урожай снижается. В последних случаях доклады о них редко
появляются в печати.
Из всех сельскохозяйственных химикатов фермеры охотнее всего применяют гербициды
и довольно трудно убедить их отказаться от этого. С древних времён фермер страдал от того,
что может быть названо «битвой с сорняками». Вспашка, междурядная культивация,
ритуальная пересадка риса – всё это направлено главным образом на подавление сорняков.
До появления гербицидов фермер должен был каждый сезон прошагать много миль по
затопленным полям вдоль рядов риса, удаляя сорняки мотыгой или руками. Легко понять,
почему гербициды были восприняты как божья милость. Используя солому и клевер и
временное затопление почвы, я нашёл простой путь держать сорняки под контролем без
тяжёлого труда прополки и без применения химикатов.

Ограничения научного метода
Прежде, чем исследователи становятся исследователями, они должны стать
философами. Они должны понять, какова цель существования человечества и что
человечество должно создать. Врачи должны прежде всего определить на фундаментальном
уровне, что человеческому существу необходимо для жизни.
С целью применения моих философских теорий к земледелию я проводил
эксперименты, выращивая культуры различными способами, в основе которых всегда лежала
идея создания метода близкого к природе. Я делал это путём исключения не необходимых
агрономических приёмов.
Современная сельскохозяйственная наука исходит из других представлений. Во-первых,
исследования проводятся не целенаправленно. Во-вторых, каждый исследователь видит
только одну часть бесконечного множества определяющих урожай естественных факторов,
которые изменяются в зависимости от места и от времени.
На одном и том же участке земли фермер должен каждый год по-разному выращивать
свои культуры в соответствии с изменениями погоды, популяции почвенных насекомых,
почвенных условий и многих других естественных факторов. Природа находится в
постоянном движении, условия одного года никогда не повторяются.
Современная наука искусственно раздробила природу на мельчайшие частички, и
проводимые учёными исследования не соответствуют ни закону природы, ни практическому
опыту. Результаты отбираются согласно запросам исследователя, но без учёта потребностей
фермера. Большой ошибкой было бы думать, что полученные таким путём умозаключения
могут быть использованы на фермерских полях с постоянным успехом.
Недавно профессор Тсуно из университета в Эхиме написал многословную книгу о
взаимосвязях между метаболизмом растений и урожаем. Этот профессор часто приходит на
моё поле, выкапывает яму, чтобы проверить состояние почвы, приводит студентов, чтобы
измерить угол падения солнечных лучей и затенение и прочую чепуху и уносит в свою
лабораторию образцы растений для анализа. Я часто спрашиваю его: «Собираетесь ли вы
попробовать метод прямого сева без вспашки?» Он, смеясь, отвечает: «Нет, применение – это
ваше дело. Я собираюсь продолжать заниматься исследованиями».
Вот как обстоит дело. Вы изучаете функцию метаболизма растений и способность
растений поглощать питательные вещества из почвы, пишете книгу и получаете степень
доктора сельскохозяйственных наук, но не интересуетесь, имеет ли какое-нибудь отношение
к урожаю ваша теория ассимиляции.
Даже если вы можете определить, как влияет метаболизм верхнего листа на
продуктивность верхнего листа при средней температуре +29 °С, есть места, где температура
иная. И если в Экиме в этом году температура +29 °С, то на следующий год температура
может быть только +24 °С. Будет ошибкой утверждать, что стимуляция метаболизма
увеличит образование крахмала и обеспечит более высокий урожай. География и топография
участка, свойства почвы, её структура, дренаж, солнечное освещение, сообщества насекомых,
сорт семян, используемых вами, способ обработки почвы – одним словом, бесконечное
разнообразие факторов должны быть приняты во внимание. Метод научного эксперимента,
учитывающий все взаимосвязанные факторы, просто невозможен.
В наши дни вы слышите много разговоров об успехах «Движения за здоровый рис» и

«Зелёной революции». Поскольку методы, применяемые сторонниками этих движений,
зависят от слабых, «улучшенных» сортов риса, они вынуждены использовать химикаты и
инсектициды восемь-десять раз в течение вегетационного периода. В короткое время в почве
начисто уничтожаются микроорганизмы и сжигается органическое вещество. Жизнь почвы
разрушается и питание растений становится зависимым от питательных веществ,
добавляемых в почву в виде минеральных удобрений.
Кажется, что дела фермера идут лучше, когда он применяет «научную» методику, но это
не значит, что наука может помочь улучшить естественное плодородие почвы, это значит, что
наука может только помочь восстановить разрушенное людьми естественное плодородие
почвы. Разбрасывая солому, выращивая клевер и возвращая в почву все органические остатки,
мы приводим землю в такое состояние, что она обладает всеми питательными веществами,
необходимыми для выращивания год за годом риса и озимых зерновых на одном и том же
поле.
Благодаря натуральному земледелию, поля, которые уже испорчены вспашкой или
применением химикатов, могут быть эффективно восстановлены.

Глава III С точки зрения фермера
В настоящее время в Японии очень большое внимание, и это оправдано, привлекает к
себе ухудшение состояния среды и в результате этого загрязнение продуктов питания.
Горожане организовали бойкоты и большие демонстрации протеста против равнодушия
политических и промышленных лидеров к этим вопросам. Но вся эта активность при
современном уровне духовного развития является пустыми усилиями. Говорить об очищении
от особо сильных загрязнений – это всё равно, что лечить симптомы болезни, в то время как
коренная причина болезни продолжает усугубляться.
Например, два года назад Исследовательский Центр сельскохозяйственного
менеджмента совместно с Советом Органического земледелия и Нада Кооперацией
организовали конференцию с целью обсуждения проблемы загрязнения среды.
Председателем на конференции был м-р Теруо Ичираку, который возглавляет Японскую
Ассоциацию Органического земледелия и является также одной из наиболее влиятельных
фигур в Государственной Сельскохозяйственной Кооперации. Рекомендациям этого
учреждения относительно того, какие выращивать сорта различных культур, какое
количество удобрений использовать и какие следует применять химикаты, следует почти
каждый японский фермер.
Поскольку в конференции принимало участие такое количество влиятельных лиц, я
следил за ней с надеждой, что могут быть приняты и претворены в жизнь далеко идущие
решения. С точки зрения рекламы проблемы загрязнения продуктов питания эта
конференция может быть признана успешной. Но, как и на других собраниях, дискуссия
выродилась в серию высоко технических докладов научных работников и персональных
сообщений об ужасах пищевого загрязнения. Ни один из выступающих не пожелал перевести
обсуждение проблемы на её фундаментальный уровень.
При обсуждении ртутного отравления тунца например, представитель Управления
рыболовства говорил о том, насколько действительно пугающей стала эта проблема. В то
время ртутное загрязнение обсуждалось каждый день по радио и в газетах и поэтому каждый
внимательно слушал его выступление.
Оратор сказал, что содержание ртути в телах тунцов, выловленных даже около
Антарктики или около Северного полюса, было необыкновенно высоким. Однако, когда
лабораторный образец рыбы, взятый несколько столетий назад, был проанализирован, то эта
рыба, вопреки ожиданиям, также содержала ртуть. В его заключении было высказано
предположение, что поглощение ртути жизненно важно для рыбы.
Люди, присутствующие на докладе, переглядывались в полном недоумении.
Предполагалось, что цель встречи – определить, что делать с отходами, которые уже
загрязнили среду, и принять меры, чтобы исправить положение. Вместо этого представитель
Управления рыболовства говорит, что ртуть необходима для выживания тунца. Вот что я
имею в виду, когда говорю, что люди не могут постигнуть коренных причин загрязнения и
видят эту проблему только в узкой и неглубокой перспективе.
Я встал и предложил предпринять совместные действия для выработки конкретного
плана работы по проблеме загрязнения. Не лучше ли прямо сказать о прекращении
использования химикатов, которые являются причиной загрязнения? Рис., например, может
быть прекрасно выращен без химикатов, так же как и цитрусовые, и не составляет большого

труда выращивать и овощи таким же способом. Я сказал, что это может быть сделано и что я
делал это на моей ферме в течение многих лет, но пока правительство продолжает поощрять
использование химикатов, оно не даст возможности «чистому» земледелию
продемонстрировать свои возможности.
На встрече присутствовали члены Управления рыболовства, так же как и работники
Министерства сельского хозяйства и лесоводства и Сельскохозяйственной Кооперации. Если
бы они и председатель конференции м-р Ишираку действительно хотели сдвинуть дело с
мёртвой точки и предложили бы фермерам по всей стране попытаться выращивать рис без
химикатов, то тогда могли бы произойти решительные изменения.
Однако для этого было одно существенное препятствие. Если бы фермеры начали
выращивать культуры без ядохимикатов, удобрений и машин, гигантские химические
компании стали бы ненужными и Государственное Управление Сельскохозяйственной
Кооперации рухнуло бы.
Чтобы поставить вопрос ребром, я сказал, что кооперативы и те, кто определяет
государственную политику зависят от инвестиций капитала в удобрения и
сельскохозяйственную технику, которые являются основой их могущества. Отказ от машин и
удобрений вызовет полное изменение экономических и социальных структур. Поэтому я не
вижу возможности для м-ра Ишираку, Кооперации или правительственных служащих
выступать в защиту мероприятий, способствующих очищению от загрязнения.
Когда я выступил таким образом, председатель сказал: «М‑р Фукуока, своими
замечаниями вы мешаете работе конференции», и этим заткнул мне рот. Да, вот, что
случилось тогда.

Простые средства для решения сложной проблемы
Таким образом, стало ясно, что правительственные организации не собираются
остановить загрязнение среды. Другая трудность заключается в том, что все аспекты
проблемы загрязнения продуктов питания должны быть собраны вместе и разрешены
одновременно. Проблема не может быть решена людьми, которые занимаются только той
или другой её частью. До тех пор, пока сознание каждого человека не будет фундаментально
трансформировано, загрязнение среды не уменьшится.
Например, фермер думает, что он не имеет никакого отношения к проблеме
Внутреннего Моря (Маленькое море между островами Хонсю, Кюсю и Шикоку). Он думает,
что это дело служащих Управления рыболовства – следить за рыбой, а дело Совета по
Окружающей среде – заботиться об уменьшении загрязнения океана. Проблема заключена
также в таком способе мышления.
Наиболее часто используемые химические удобрения – сульфат аммония, мочевина,
суперфосфат и другие применяются в больших количествах, но только небольшая их часть
поглощается растениями в поле. Остальное смывается в ручьи и реки и постепенно
выносится ими во Внутреннее Море. Азотистые соединения становятся пищей для
водорослей и планктона, которые размножаются в громадных количествах, что служит
причиной появления «красного прилива». Конечно, промышленные выбросы ртути и других
отходов тоже вносят свой вклад в загрязнение, но загрязнение воды в Японии в наибольшей
степени обусловлено сельскохозяйственными химикатами.
Таким образом, фермеры должны нести основную долю ответственности за появление
«красного прилива». Фермер, который применяет химикаты на своём поле, корпорации,
производящие эти химикаты; сельскохозяйственные чиновники, которые верят в пользу
химикатов и согласно этому составляют технические рекомендации – если все эти люди не
осмыслят проблему достаточно глубоко, то вопрос загрязнения воды не будет никогда
разрешён.
Теперь только те, кто непосредственно сам пострадал, становятся активными в
попытках разрешить проблему загрязнения, как это было в случае с местными рыбаками,
которые выступили против больших нефтяных компаний после утечки нефти около
Мицушима. Другая попытка решения проблемы принадлежит одному профессору, который
предложил прорыть канал через остров Шикоку, чтобы относительно чистая вода Тихого
океана потекла во Внутреннее Море. Такие попытки делаются время от времени, но
радикальное решение проблемы никогда не будет достигнуто таким путём.
Фактически, что бы мы ни делали, ситуация только ухудшается. Чем тщательнее
разрабатывают контрмеры, тем сложнее становятся проблемы. Предположим, через Шикоку
будет проложена труба и вода будет закачиваться из Тихого океана и переливаться во
Внутреннее Море. Но откуда придёт электроэнергия, необходимая предприятию для
производства стальных труб и откуда возьмётся энергия для закачивания воды? Для этого
необходимы атомные электростанции. Для строительства этих энергетических систем
должен быть завезён бетон и все необходимые материалы, а также построен центр уранового
процессинга. Если дело пойдёт таким путём, оно только посеет семена второго и третьего
поколения проблем загрязнения, ещё более сложных, чем предыдущие.
Это похоже на случай с жадным фермером, который слишком широко открывает вход в

оросительный канал, чтобы побольше воды устремилось на его рисовые поля. В результате
образуется ударная волна, которая размывает край канала. После этого необходимы
восстановительные работы. Стенки оросительного канала выравнивают и русло расширяют.
Увеличившийся объём потока воды только увеличивает потенциальную опасность размыва, и
в следующий раз кромка снопа не выдерживает и ещё большие усилия требуют для
устранения повреждения.
Если принимают решение работать над симптомами проблемы, то при этом обычно
подразумевают, что исправительные меры смогут устранить саму проблему. Но это редко
получается. Кажется, что инженеры не в состоянии усвоить это. Принимаемые контрмеры
основаны на слишком узком представлении о том, что надо исправлять. Как меры, так и
контрмеры основаны на ограниченном понимании научной истины и суждений. Правильное
решение никогда не придёт таким путём (Словами «ограниченная научная истина» м-р
Фукуока характеризует то представление о мире, которое сконструировано человеческим
интеллектом. Он считает это восприятие ограниченным рамками субъективных
представлений).
Моё скромное решение вопроса путём разбрасывания соломы и выращивания клевера
не создаёт нового загрязнения. Оно эффективно, потому что оно уничтожает источник
проблемы. Пока вера современного человечества в возможность решения этой проблемы с
помощью больших технологических разработок не будет отвергнута, загрязнение среды будет
только увеличиваться.

Плоды трудных времён
Потребители тоже считают, что они не имеют никакого отношения к причинам
сельскохозяйственного загрязнения. Многие из них хотят иметь продукты питания, не
обработанные химикатами. Но химически обработанные продукты продаются, главным
образом, как ответ на запросы потребителей. Потребитель требует большой лоснящийся
плод без пятен и правильной формы. Чтобы удовлетворить эти желания, быстро вводятся в
употребление новые химикаты, которых не было ещё 5–6 лет тому назад.
Как мы зашли в такой тупик? Люди говорят, что для них не имеет значения, имеет ли
огурец прямую или искривлённую форму, и что фрукты не обязательно должны иметь
красивую внешность. Но посмотрите на оптовый рынок в Токио, если вы хотите узнать, как
цены реагируют на предпочтения потребителей. Если фрукты выглядят у вас немного лучше,
чем у других продавцов, вы получаете премию 10–20 центов за килограмм. Если фрукты
классифицируют как мелкие, средние и крупные, цена может увеличиваться в два или три
раза за каждый размер.
Согласие потребителей платить высокую цену за продукты, выращенные не в сезон,
также вносит свой вклад в увеличение использования искусственных методов выращивания
и химикатов. В последний год мандарины из Уншу, выращенные в теплицах для летней
продажи (обычно фрукты созревают поздно осенью), имели цену а 10–20 раз более высокую,
чем сезонные мандарины. Таким образом, если вы вложите несколько тысяч долларов в
закупку оборудования, купите необходимое топливо и поработаете свыше всякой нормы, вы
можете получить прибыль.
Выращивание овощей и фруктов вне сезона становится всё более и более популярным.
Чтобы получить мандарины на один месяц раньше, люди в городе кажется были бы
счастливы заплатить за все расходы фермера на труд и оборудование. Но если вы спросите,
нужно ли человеку иметь эти фрукты на месяц раньше, то правдивый ответ будет – нет,
совсем не нужно, и деньги – это не единственная цена, которую мы платим за это
потворство своим желаниям.
Кроме того, окрашивающие вещества, которые не использовались несколько лет назад,
теперь используются. Благодаря этим химикатам плоды приобретают зрелую окраску на одну
неделю раньше. В зависимости от того, проданы ли фрукты на неделю раньше или после 10
октября, цена в два раза повышается или падает, поэтому фермер применяет ускоряющие
созревание химикаты и после уборки помещает плоды в камеру для обработки газом,
ускоряющим созревание.
Но если плоды сняты рано, они не очень сладкие, и поэтому используются
искусственные сладкие добавки. Обычно думают, что химические подслащиватели были
запрещены, но искусственный подслащиватель, которым опрыскивают цитрусовые деревья,
не был объявлен вне закона. Вопрос заключается в том, попадает ли он в категорию
«сельскохозяйственные химикаты». Во всяком случае, почти каждый фермер использует его.
Затем фрукты забирает кооперативный сортировочный центр. Чтобы разделить плоды
по размеру на крупные и мелкие, их заставляют прокатиться несколько сотен метров по
длинному конвейеру. Неизбежны удары и повреждения. Чем крупнее сортирующий центр,
тем больше плоды бьются и мнутся. После мытья водой мандарины опрыскивают
защитными и окрашивающими веществами. Наконец, как завершающий мазок их

обрабатывают парафином и полируют до блеска.
Таким образом, начиная со времени, предшествующего уборке, и до того момента, когда
плоды попадают на прилавок их обрабатывают химикатами пять или шесть раз. Здесь не
упомянуты ещё химические удобрения и опрыскивания плодовых деревьев пестицидами. И
всё это потому, что покупатель хочет покупать фрукты немножко более привлекательные.
Именно эта маленькая грань предпочтения ставит фермера в действительно затруднительное
положение.
Все эти меры принимаются не потому, что фермеру нравится это делать, и не потому,
что чиновники Министерства Сельского хозяйства получают удовольствие, заставляя
фермера трудиться сверх меры, этого требует общепринятая шкала ценностей.
Когда я сорок лет назад работал в Иокогаме на таможне, лимоны и апельсины
обрабатывались именно таким образом. Я был категорически против введения этой системы,
но все мои усилия не могли помешать её внедрению в практику.
Если одно фермерское хозяйство или кооператив осваивают новый приём, как,
например вощение мандаринов, как знак особого внимания и заботы о внешнем виде, то они
получают более высокую прибыль. Это замечают другие сельскохозяйственные кооперативы
и скоро они также осваивают новый приём. Плоды, не покрытые воском, больше не могут
получить такую же высокую цену. Через два или три года вощение плодов распространяется
по всей стране. Затем благодаря конкуренции цены снижаются и всё, что остаётся фермеру –
это бремя тяжёлой работы и дополнительные расходы на материалы и оборудование. Теперь
он должен покрывать плоды воском.
В результате, конечно, страдает покупатель. Продукты, которые фактически не являются
свежими, могут быть проданы, потому что они выглядят свежими. С биологической точки
зрения плоды в слегка подвядшем состоянии снижают своё дыхание и расход энергии до
возможно низкого уровня. Это похоже на состояние человека в медитации: его метаболизм,
дыхание и расход энергии снижается до необычно низкого уровня. Даже если он голодает,
энергия в его теле будет сохранена. Точно также, когда мандарин сморщивается при
подсыхании или когда овощи подвядают, они переходят в состояние, которое сохраняет их
пищевые качества в течение возможно долгого времени,
Не стоит пытаться сохранить только вид свежести, что происходит, когда продавец снова
и снова сбрызгивает водой свои овощи. Хотя при этом овощи производят впечатление
свежих, их вкус и питательная ценность снижаются.
Все сельскохозяйственные кооперативы и коллективные центры сортировки были
объединены и расширены для такой ненужной деятельности, как придание продуктам
товарного вида. Это называется модернизацией. Продукты упаковывают, отправляют в
громадную систему доставки, откуда они поступают к потребителю.
Короче говоря, пока существует извращённое представление о ценностях, согласно
которому размеры и внешний вид имеют большее значение, чем качество, проблема
загрязнения продуктов питания не будет решена.

Торговля натуральными продуктами
За последние несколько лет я передал 2,2–2,9 тонн риса в магазины натуральных
продуктов в различных частях страны. Я отгрузил 400 15‑ти килограммовых коробок с
мандаринами в кооператив в Токийском округе Сугинами. Председатель кооператива хотел
продавать чистую продукцию, и это явилось основой для нашего соглашения. Первый год
прошёл довольно успешно, хотя были некоторые жалобы. Размер плодов сильно варьировал,
поверхность была не совсем чистой, кожица в некоторых местах сморщилась и так далее. Я
отгружал плоды в простых немаркированных коробках, и некоторые люди подозревали без
всяких оснований, что это фрукты второго сорта. Теперь я упаковываю плоды в коробки с
надписью «натуральные мандарины».
Поскольку натуральные продукты можно производить с наименьшими затратами и
усилиями, я считаю, что они должны продаваться по самой низкой цене. В последний год в
Токийском округе мои фрукты были самыми дешёвыми. По мнению многих владельцев
магазинов, они обладали самым изысканным вкусом. Конечно, было бы лучше всего, если бы
фрукты можно было бы продавать прямо на месте, исключив тем самым время и затраты на
перевозку. Но даже и так цена была правильной, фрукты не содержали химикатов и имели
хороший вкус. В этом году меня просили отгрузить в 2–3 раза больше мандаринов, чем
раньше.
Теперь возникает вопрос, как далеко может распространиться прямая продажа
натуральных продуктов. В этом отношении у меня есть одна надежда. В последнее время
производители химически обработанных продуктов попали в очень тесные экономические
тиски, и это сделало более привлекательным для них выращивание натуральных продуктов.
Несмотря на то, что среднему фермеру приходится тяжело трудиться, применяя химикаты,
ускорители созревания, покрывая плоды воском и так далее, он может продать свои плоды
только за такую цену, которая едва покрывает расходы. Фермер, продукция которого немного
более низкого качества, закончит год совсем без всякой прибыли.
Поскольку в последние несколько лет цены резко упали, сельскохозяйственные
кооперативы и сортировочные центры стали очень требовательны, принимая фрукты только
очень высокого качества. Худшие по качеству фрукты не могут быть проданы сортировочным
центрам. После целого дня работы в саду: сначала сбор мандаринов, затем раскладка их в
ящики и перевозка в помещение для сортировки, фермер должен ещё трудиться до 11–12
часов ночи, отбирая плоды только совершенной формы и размера (не попавшие в этот разряд
фрукты продают приблизительно за полцены частным компаниям для изготовления сока).
Хорошие плоды составляют в среднем от 25 до 50 % от общего урожая и ещё какая-то их
часть бракуется кооперативом. Если прибыль составляет хотя бы 2–3 цента на фунт, это
считается очень хорошо. Бедный фермер тяжело трудится целыми днями и всё же существует
на грани разорения.
Выращивание фруктов без химикатов, удобрений и вспашки почвы требует меньше
затрат и, следовательно, фермер получает более высокую прибыль. Фрукты поступают в
продажу практически без сортировки. Я только упаковываю фрукты в коробки, отсылаю их
на рынок и рано ложусь спать.
Мои соседи – фермеры поняли, что они тяжело трудятся только для того, чтобы
закончить год, не имея ничего в карманах. Они постепенно приходят к выводу, что нет ничего

странного в выращивании натуральных продуктов питания. Внутренне они уже готовы к
переходу к земледелию без химикатов. Но пока натуральные продукты распределяются
локально, среднего фермера будет беспокоить отсутствие рынка для продажи его продукции.
Что касается потребителей, то среди них распространено мнение, что натуральные
продукты должны быть дороги. Если они недороги, люди подозревают, что продукты не
натуральные. Один лавочник сказал мне, что никто не будет покупать натуральные продукты,
пока они не будут стоить дорого.
Я всё же думаю, что натуральные продукты должны продаваться дешевле. Несколько лет
назад меня попросили прислать мёд, собранный в цитрусовом саду, и яйца от кур, живущих
на горе, в магазин натуральных продуктов в Токио. Когда я обнаружил, что торговец продаёт
их по высоким ценам, я был взбешён. Я знаю также, что торговец обманывал своих
покупателей, смешивая мой рис с другим рисом, чтобы увеличить вес, и этот рис продаёт
покупателям по неправильной цене. Я немедленно прекратил отгрузку продуктов в этот
магазин.
Если за натуральные продукты берут высокую цену, это значит, что торговец получает
избыточную прибыль. Кроме того, если натуральные продукты дороги, они становятся
предметом роскоши и только богатые люди могут себе позволить питаться ими.
Если натуральные продукты должны стать широко известны, они должны быть
доступны на местах по умеренным ценам. Покупатели должны привыкнуть к мысли, что
низкая цена не означает, что продукты не натуральные.

Коммерческое земледелие потерпит поражение
Когда впервые появилась концепция коммерческого земледелия, я был против неё.
Коммерческое земледелие в Японии не доходно для фермера. Среди коммерсантов
существует правило, что за качество продукта начисляется дополнительная плата. Но в
японском земледелии всё это не так прямо связано. Удобрение, корма, оборудование,
химикаты закупаются по ценам, установленным за рубежом, и поэтому трудно сказать,
какова будет истинная цена этих импортированных товаров для фермеров. Это полностью
зависит от коммерсантов. И поскольку продажная цена продукта строго фиксирована, то
получается, что доходы фермера отданы на милость тех сил, которые находятся вне его
контроля.
Обычно коммерческое земледелие находится в неустойчивом положении. Фермер был
бы в гораздо лучшем положении, если бы выращивал продукты только для удовлетворения
своих потребностей, не думая о зарабатывании денег. Если вы посадили одно зерно риса, вы
получаете больше тысячи зёрен. Один ряд репы даст вам корнеплодов на всю зиму. Если вы
будете следовать этому образу мыслей, у вас будет достаточно еды, более, чем достаточно,
без всякой борьбы. Но если вы вместо этого решите делать деньги, вы оказываетесь в бешено
мчащемся экспрессе прибыли и мчитесь вместе с ним.
В последнее время я часто думал о белых леггорнах. Благодаря улучшению породы
белый леггорн откладывает яйца 200 дней в году, поэтому выращивание их для получения
прибыли – относительно хороший бизнес. Для коммерческого выращивания эти куры
содержатся в маленьких клетках, составленных длинными рядами, напоминающими камеры
для заключённых. И за всю их жизнь их ноги ни разу не ступают по земле. Среди них часто
распространяются болезни, поэтому птицы накачаны антибиотиками и в корма вводят
витамины и гормоны.
Говорят, что местные куры, которые сохранились здесь с древних времён, коричневые и
чёрные «шамо» и «чабо» откладывают яйца в два раза меньше. В результате этих кур в
Японии больше не держат. Я выпустил двух кур и петуха свободно бегать по склонам горы, а
через год их стало двадцать четыре. Птицы отложили мало яиц только потому, что они были
заняты выращиванием цыплят.
В первый год леггорны дают более высокую яйценоскость, чем местные породы, но
через год леггорн истощается и выбраковывается. Кроме того, леггорн даёт много яиц, так
как он выкормлен искусственно обогащённой пищей, импортируемой из других стран и
покупаемой у коммерсантов, а местные птицы породы шамо свободно бегают между
деревьями, питаются семенами и насекомыми и дают отборные натуральные яйца.
Если вы думаете, что коммерческие овощи произвела природа, вы очень сильно
ошибаетесь. Эти овощи представляют собой водянистый концентрат азота, фосфора и калия,
полученный с небольшой помощью семян. И они имеют соответствующий вкус. И
коммерческие куриные яйца (вы можете называть их яйцами, если вам это нравится) – это
ничего более, как смесь синтетической пищи, химикатов и гормонов. Это не природный
продукт, но сделанная людьми синтетика в форме яйца. Фермера, который производит овощи
и яйца такого рода, я называю промышленником.
Теперь, когда речь зашла о промышленном производстве, вам придётся заняться
некоторыми мысленными подсчётами, если вы хотите иметь прибыль. Если коммерческий

фермер не умеет считать деньги, он похож на коммерсанта, который не умеет обращаться со
счетами. Другие люди считают его глупцом и его прибыль присваивают себе политики и
торговцы.
В старые времена были воины, земледельцы, ремесленники и торговцы. Сельское
хозяйство было ближе к источнику жизни, чем торговля или промышленность, и
земледельца почитали как «виночерпия богов». Он всегда был в состоянии свести концы с
концами, и всегда имел достаточно еды.
Но теперь все находятся в волнении по поводу делания денег. Выращивают ультрамодные продукты: виноград, томаты, дыни. Цветы и фрукты выращивают в любой сезон в
теплицах. Распространилось разведение рыбы и выращивание крупного рогатого скота,
поскольку это даст высокую прибыль.
Такой порядок вещей ясно показывает, что происходит, когда земледелие включается в
погоню за прибылью. Колебания цен громадны. Есть прибыли, но есть также и потери.
Катастрофа неминуема. Японское сельское хозяйство потеряло чувство своего
направления и стало нестабильным. Оно отошло от основных принципов сельского
хозяйства и стало бизнесом.

Исследования для чьей пользы?
Когда я впервые начал прямой посев риса и озимых зерновых, я собирался убирать
урожай вручную серпом и поэтому считал, что более удобно сеять семена правильными
рядами. После многих попыток, всё время попадая впросак как дилетант, я сделал
приспособление для посева. Думая, что это приспособление может быть использовано
другими фермерами, я принёс его показать человеку с Испытательного Центра. Он сказал
мне, что поскольку мы живём в век больших машин, моё «хитроумное» приспособление не
может его интересовать,
Затем я пошёл к производителю сельскохозяйственного оборудования. Там мне сказали,
что такое простое орудие, независимо от того, как много вы их сделаете, не может быть
продано дороже, чем 3,5 доллара за штуку, «Если мы будем делать такие безделушки,
фермеры могут начать думать, что им не нужны тракторы, которые мы продаём за тысячи
долларов». Он сказал, что в наше время идея заключается в том, чтобы быстро изобрести
машину для посадки риса, продавать её как можно дольше, затем придумать что-нибудь
новое. Вместо маленьких тракторов, они хотели перейти к большим моделям, и моё
предложение было для них шагом назад. Чтобы удовлетворять требованиям времени,
ресурсы тратятся на дальнейшие бесполезные исследования, и до сегодняшнего дня мой
патент остаётся лежать на полке.
То же самое с удобрениями и химикатам!. Вместо того, чтобы создавать удобрения с
учётом потребностей фермера, удобрение делают для производства чего-нибудь нового,
чтобы делать деньги. После того, как специалисты уходят с работы в Испытательных
Центрах, они переводятся на работу прямо в большие химические компании. Недавно я
разговаривал с м-ром Асада, чиновником из Министерства Сельского хозяйства и
Лесоводства, и он рассказал мне интересную историю. Узнав, что баклажаны, поступившие
в продажу зимой, лишены витаминов, а огурцы – вкуса, он исследовал это явление и нашёл
причину: часть солнечных лучей не может проникнуть через виниловое и стеклянное
покрытие, под которым выращивают овощи. Его исследования говорят в пользу помещения
осветительной системы внутрь теплицы
Здесь возникает основной вопрос: необходимо ли человеку есть зимой баклажаны и
огурцы. Единственная причина, по которой эти овощи выращиваются зимой, это то, что они
могут быть проданы за хорошую цену. Кто-то разрабатывает систему их выращивания, а через
некоторое время обнаруживается, что эти овощи не имеют никакой питательной ценности.
Специалист думает, что если питательные вещества потеряны, то надо найти способ
предотвратить эту потерю. Поскольку считают, что причина в системе освещения, он
начинает исследовать свет. Он думает, что всё будет в порядке, если он сможет выращивать
тепличные баклажаны, содержащие витамины. Мне говорили, что есть специалисты,
которые всю свою жизнь посвятили исследованиям такого рода.
Конечно, если такие громадные усилия и ресурсы направлены на производство
тепличных баклажан и объявлено, что эти овощи обладают высокой питательной ценностью,
то они имеют более высокую цену и хорошо продаются. «Если это прибыльно, и если вы
можете продать это, то в этом не может быть ничего плохого».
Как бы люди ни старались, они не могут улучшить качества выращенных натуральным
способом овощей и фруктов. Продукты, выращенные неестественным способом,

удовлетворяют только мимолётные желания людей, но ослабляют тело и изменяют химию
тела так, что оно начинает зависеть от такой пищи. Когда это происходит, то становятся
необходимыми витаминные добавки и лекарства. Эта ситуация создаёт только трудности для
фермера и страдания для потребителей.

Что такое пища человека?
Однажды кто-то с телевидения пришёл ко мне и попросил сказать что-нибудь о вкусе
натуральной пищи. Мы поговорили, а потом я предложил ему сравнить яйца, отложенные
курами в курятнике, с яйцами от кур, свободно живущих в саду. Он увидел, что желток у яиц
от кур типичного птицеводческого хозяйства, мягкий и водянистый и имеет светло-жёлтую
окраску. А желток у яиц тех кур, которые жили на горе, плотный, упругий и светлооранжевого цвета. Когда старик, который держит ресторан «суши» в городе, попробовал одно
из натуральных яиц, он сказал: «Вот это настоящее яйцо, такое же, как в старые дни» и он
наслаждался этим яйцом, как будто это было какое-то редкое сокровище.
В мандариновом саду наверху среди сорняков и клевера растёт много разных овощей.
Репа, лопух, огурцы и тыквы, арахис, морковь, картофель, съедобные хризантемы, лук,
листовая горчица, капуста, различные виды бобов и многие другие травы и овощи растут все
вместе. Мы обсуждали, имеют ли эти овощи, выращенные как полудикие растения, лучший
вкус, чем те, которые выращены на домашнем огороде или в полях с помощью химических
удобрений. Когда мы сравнивали их, вкус был совершенно различен, и мы пришли к выводу,
что «дикие» овощи имеют более «богатый» вкус.
Я сказал репортёру, что когда овощи выращивают на вспаханном поле с использованием
химических удобрений, в почву вносят азот, фосфор и калий. Но когда овощи растут среди
естественной растительности на почве, богатой естественным органическим веществом,
они получают более сбалансированное питание. Большое разнообразие сорняков и трав
означает, что в почве содержатся разнообразные доступные овощным растениям элементы и
микроэлементы питания. Растения, выращенные на такой сбалансированной почве,
обладают более тонким вкусом.
Съедобные травы и дикие овощи, растения, выросшие на склоне горы или на лугу,
обладают высокой питательной ценностью и могут быть использованы как лекарства. Пища
и лекарства – это не совсем две разные вещи: это две стороны одной медали. Химически
выращенные овощи можно съесть как пищу, но их нельзя использовать как лекарства.
Если вы собираете и съедаете семь весенних трав (водный кресс, пастушья сумка, дикая
репа, сушеница, мокричник, дикая редька, пикульник), ваша душа смягчается. А если вы
съедаете побеги папоротника-орляка, осмунды и пастушьей сумки, вы станете спокойны.
Пастушья сумка – лучшее средство, чтобы успокоить нетерпение и тревогу. Говорят, что если
маленькие дети едят пастушью сумку, почки ивы или насекомых, живущих на деревьях, они
становятся спокойнее и меньше плачут. И поэтому в прежние времена детям давали их есть.
Дайкон (японская редька) имеет в качестве предка растение, называемое «назуна» (пастушья
сумка), и слово «назуна» родственно слову «нагома», что значит «быть смягчённым».
Дайкон – это растение, «которое смягчает нрав».
Говоря о естественной пище, часто забывают насекомых. Во время войны, когда я
работал в Опытном центре, мне было поручено изучить какие насекомые в Южной Азии
могут служить пищей. Когда я занялся исследованием этого вопроса, я был поражён,
обнаружив, что почти каждое насекомое съедобно.
Например, никто не думает, что вши и блохи могут иметь какое-то употребление. Но
вши, размолотые и съеденные с зерном, являются средством от эпилепсии, а блохи –
лекарство от обморожения. Все личинки насекомых довольно съедобны, но они должны

быть съедены живыми. Изучая старые тексты, я нашёл истории о «деликатесах»,
приготовленных из личинок, и вкус обычного шелковичного червя был назван «изысканным
сверх всякого сравнения». Даже бабочки, если вы стряхнёте с них пыльцу, очень вкусны.
Так с точки зрения вкуса или здоровья, многие вещи, которые люди считают
омерзительными, на самом деле довольно вкусны и полезны.
Овощи, которые биологически наиболее близки своим диким предкам, имеют самый
лучший вкус и пищевую ценность. Например, в семействе лилейных, которые включают
чеснок, китайский лук, лук-порей, лук на перо, перламутровый лук, «нира», репчатый лук,
«нира» и лук-порей наиболее питательны, используются как лекарство и как тонизирующее
средство. Для большинства людей, однако, более привычны лук на перо и репчатый лук,
которые считаются самыми вкусными. По каким-то причинам современные люди любят
вкус овощей, который не похож на вкус их диких предков.
Те же вкусовые предпочтения наблюдаются в отношении животной пищи. Дикие птицы
гораздо полезнее, чем домашние куры и утки, и всё же домашние птицы, выращенные в
условиях очень далёких от естественных, считаются вкусными и продаются по высоким
ценам. Козье молоко намного питательнее, чем коровье, но именно на коровье молоко
установился наибольший спрос. Продукты питания, которые получены с помощью
химикатов или в совершенно неестественных условиях, нарушают химический баланс тела.
Чем более разбалансировано тело, тем более сильной становится потребность в
неестественной пище. Эта ситуация опасна для здоровья.
Говорить, что питание человека – это только вопрос вкуса – значит обманывать себя,
потому что спрос на неестественную или экзотическую пищу создаёт трудности и для
фермера и для рыбака. Чем больше желаний, тем больше требуется работы, чтобы их
удовлетворить. Некоторые рыбы, как например популярный тунец и желтохвост ловятся
только в отдалённых водах, но сардины, морской лещ, камбала и другие мелкие рыбы в
изобилии водятся во Внутреннем Море. С точки зрения питания, животные, обитающие в
пресной воде рек и ручьёв, как сазан, прудовая улитка, речной рак, болотные крабы и так
далее полезнее для здоровья, чем животные солёных вод. Среди морских животных на
первом месте по питательности стоят мелководные обитатели, за ними – рыбы
глубоководных морей. Пища, которая рядом, для человека полезнее, а те продукты, которые
добываются с большим трудом, оказываются наименее ценными. Короче говоря, та пища,
которая всегда под руками, всегда полезна. Если фермер, который живёт в деревне, ест
только ту пищу, которая может быть выращена или собрана на месте, он не ошибётся. А ктото, подобно молодым людям, живущим в хижинах в моём саду, сочтёт самым простым есть
неполированный рис, неотшлифованный ячмень, просо и гречиху вместе с
соответствующими сезону полудикими овощами. Это будет самая лучшая пища. Она будет
иметь хороший вкус и будет полезна для тела. Если собирать 5,85 ц риса и 5,85 ц озимых
зерновых с поля размером 0,1 га, то такое поле может снабжать питанием 5–10 человек,
причём каждый человек будет затрачивать на выращивание этого урожая меньше одного часа
работы в день. А если поле будет отведено под пастбище или если зерно скармливать скоту,
то на обработку 0,1 га потребуется всего 1 человек. Мясо становится предметом роскоши,
если на его производство требуется земля, которая могла бы непосредственно производить
пищу для потребления человека (хотя большая часть мяса в Северной Америке производится
путём скармливания скоту зерна полевых культур: пшеницы, ячменя, кукурузы и сои, есть
также большие площади земли, которые можно использовать только как пастбища или

сенокосы. В Японии такой земли почти нет. Почти всё мясо должно ввозиться из-за рубежа).
Это нужно сказать ясно и определено. Каждый человек должен серьёзно взвесить,
сколько затруднений он создаёт, отдавая предпочтение пище, получаемой такой дорогой
ценой.
Мясо и другие импортируемые продукты – это предметы роскоши, так как они требуют
больше энергии и ресурсов, чем традиционные овощи и зерно, производимые на месте. Из
этого следует, что люди, которые ограничивают себя простой местной пищей, должны
меньше работать и им требуется меньше земли, чем тем, у кого разгорается аппетит на
деликатесы.
Если люди будут продолжать есть мясо и импортировать продукты питания ещё в
течение 10 лет, то можно с уверенностью сказать, что в Японии начнётся пищевой кризис. В
течение 30 лет произойдёт очень сильное снижение производства продуктов. Откуда-то в
страну проникла абсурдная идея, что замена в питании риса хлебом означает повышение
уровня жизни японского народа. На самом деле это не так. Неполированный рис и овощи
могут показаться грубой пищей, но это наиболее питательная пища, и она делает человека
способным жить просто и естественно.
Если у нас разразится пищевой кризис, он разразится не от недостатка естественных
производительных сил, а от непомерных человеческих желаний.

Милосердный конец ячменя
Сорок лет назад в результате усиления политической вражды между США и Японией
прекратился ввоз пшеницы из Америки. По всей стране возникло движение за выращивание
пшеницы своими силами. Американские сорта пшеницы требуют длинного вегетационного
периода и в Японии созревание зерна приходится на середину сезона дождей. После того,
как фермер потратит столько труда для выращивания культуры, зерно часто сгнивает во
время уборки. Эти сорта доказали свою непригодность и высокую чувствительность к
болезням, поэтому фермеры не хотели выращивать пшеницу. Размолотая и поджаренная
традиционным способом, она имела такой отвратительный вкус, что её невозможно было
есть.
Традиционные сорта ржи и ячменя могут быть убраны в мае, до начала дождей, поэтому
они более надёжны. Возделывание пшеницы было навязано фермерам без всякого смысла.
Хотя все смеялись и говорили, что нет ничего хуже, чем выращивание пшеницы, но тем не
менее фермеры терпеливо следовали государственной политике.
После войны американская пшеница снова импортировалась в больших количествах,
что вызвало падение цен на пшеницу, выращенную в Японии. Это добавило ко многим
другим ещё одну причину для прекращения выращивания пшеницы. «Покончим с пшеницей!
Покончим с пшеницей!» – таков был лозунг, пропагандируемый по всей стране
государственными сельскохозяйственными лидерами, и фермеры с радостью следовали ему.
В то же самое время из-за низких цен на импортируемую пшеницу государство поощряло
фермеров прекратить выращивание традиционных озимых зерновых ржи и ячменя. Эта
политика привела к тому, что поля в Японии оставляли под паром на всю зиму.
Приблизительно десять лет назад я был выбран представлять префектуру Эхиме в
телевизионном конкурсе «Лучший фермер года». В это время член отборочного комитета
спросил меня: «М‑р Фукуока, почему вы не прекратили выращивание ржи и ячменя?»
Я ответил: «Рожь и ячмень легко выращивать, и чередуя их с посевами риса вы можете
получить с японских полей самое большое количество калорий. Вот почему я не прекратил
их выращивать».
Мне дали понять, что тот, кто упрямо идёт против воли Министерства Сельского
хозяйства, не может быть назван «Лучшим фермером», и тогда я сказал: «Если это может
помешать получить звание «Лучшего фермера», лучше я останусь без этого звания». Один из
членов отборочной комиссии позже сказал мне: «Если бы я бросил университет и стал
фермером, я, возможно, вёл бы хозяйство, как вы, и выращивал рис летом, а рожь и ячмень
зимой, каждый год, как до войны».
Вскоре после этого эпизода я участвовал в телевизионной программе в дискуссии с
различными университетскими профессорами и меня снова спросили: «Почему вы не
прекратили выращивать рожь и ячмень?» Я снова очень чётко объяснил, что я не мог этого
сделать по любой из дюжины веских причин. Приблизительно в это время лозунг о
прекращении возделывания озимых зерновых призывал к «Милосердному концу». Это
значило, что практика выращивания озимых зерновых и риса должна тихо скончаться. Но
«милосердный конец» – это слишком нежный термин. Министерство Сельского хозяйства,
действительно, хотело похоронить его в могиле. Когда мне стало ясно, что основная цель
программы – ускорить конец выращивания озимых зерновых, оставив их, так сказать,

«умирать на обочине дороги», я взорвался от возмущения.
Сорок лет назад был призыв выращивать пшеницу, выращивать чуждую зерновую
культуру, бесполезную и неудобную в наших условиях. Затем было сказано, что японские
сорта ржи и ячменя не имеют такой высокой питательной ценности, как американская
пшеница. И фермеры с сожалением прекратили выращивание этих традиционных культур.
Поскольку жизненный стандарт рос очень быстро, люди начали есть мясо, есть яйца, пить
молоко и вместо риса стали есть хлеб. Кукуруза, соя и пшеница ввозились во всё больших
количествах. Американская пшеница была дешёвой, поэтому выращивание местных
культур – ржи и ячменя – было прекращено. Японское сельское хозяйство освоило те
приёмы, которые заставляли фермеров часть времени работать в городе, чтобы они могли
покупать те продукты, которые им было велено не выращивать.
А теперь возникло новое представление о недостатке пищевых ресурсов. Снова начали
пропагандировать самообеспечение производством ржи и ячменя. Говорят, что это будет
даже субсидироваться. Но нельзя выращивать традиционные озимые зерновые пару лет, а
затем снова отказаться от них. Должна быть выработана здоровая сельскохозяйственная
политика. Поскольку Министерство Сельского хозяйства не имеет ясного представления о
том, что нужно выращивать в первую очередь, и поскольку оно не понимает, какая связь
существует между тем, что выращивается в полях, и питанием, устойчивая
сельскохозяйственная политика остаётся недосягаемой.
Если работники Министерства пошли бы в горы и на луга, собрали бы семь весенних
трав и семь осенних трав (китайский колокольчик, маранта (кудзу), посконник, валериана,
леспедеца, смолёвка и японская пампасная трава) и попробовали их на вкус, они бы узнали,
каков источник питания человека. Если бы они захотели узнать больше, они увидели бы, что
вы можете очень хорошо жить, питаясь традиционными домашними культурами: рисом,
ячменём, рожью, гречихой и овощами и они могли бы просто решить, что это всё, что нужно
выращивать японскому сельскому хозяйству. Если это всё, что должен выращивать фермер, то
земледелие становится очень лёгким делом.
До сих пор среди современных экономистов господствует мнений, что мелкие, сами
себя обеспечивающие фермы не нужны, что это примитивный тип сельского хозяйства,
который должен быть уничтожен как можно скорее. Было установлено, что площадь каждого
поля должна быть увеличена, чтобы обеспечить переход к крупномасштабному, сельскому
хозяйству американского типа. Этот тип мышления относится не только к сельскому
хозяйству – развитие всех областей жизни идёт в том же направлении.
Цель заключается в том, чтобы в земледелии осталась только небольшая часть
населения. Сельскохозяйственные авторитеты говорят, что меньшее количество людей,
используя большие сельскохозяйственные машины, может получить больший урожай с той
же самой площади. Так понимают прогресс в сельском хозяйстве.
После войны 70–80 % населения Японии были фермерами. Это количество быстро
снизилось до 50 %, затем 30 %, 20 % и теперь эта цифра составляет около 14 %. Намерение
Министерства Сельского хозяйства состоит в том, чтобы достигнуть того же уровня, что в
Европе и Америке, оставив в сельском хозяйстве меньше 10 % населения.
По моему мнению, если бы 100 % населения были бы фермерами, это было бы
идеально. На одного человека в Японии приходится как раз 0,1 га пахотной земли. Если
каждому человеку дать по 0,1 га, это составит 0,5 га на семью из пяти человек, этого будет
более, чем достаточно, чтобы кормить семью в течение целого года. Если будет введено

натуральное земледелие, то фермер будет иметь также достаточно времени для отдыха и
социальной активности в деревенской общине. Я думаю, что это наиболее прямой путь,
чтобы превратить страну в счастливую процветающую землю.

Просто следуй природе, и всё будет хорошо
Преувеличенность желаний – это основная причина, которая привела мир к кризисному
состоянию. Быстро лучше, чем медленно; больше лучше, чем меньше – это внешние черты
так называемого «развития». Они непосредственно связаны с надвигающейся катастрофой
нашего общества. Человечество должно прекратить удовлетворять своё желание
материального обладания и личной выгоды и вместо этого начать двигаться по пути
духовного знания.
Сельское хозяйство должно перейти от больших механизированных хозяйств к
маленьким фермам, производящим только необходимое для их жизни. Материальная жизнь и
питание должны занять своё соответствующее место. Если это будет сделано, работа станет
удовольствием и духовная жизнь станет полной.
Чем более фермер разворачивает масштабы своих операций, тем больше его тело и душа
растрачиваются зря и тем дальше он уходит от духовно удовлетворяющей жизни. Жизнь
маленькой фермы может показаться примитивной, но такая жизнь делает возможным
созерцание Великого Пути (путь духовного осознания, который включает внимание и заботу
об обычной повседневной деятельности). Я думаю, что если человек сумеет глубоко
проникнуть в жизнь своего собственного окружения и мир повседневных забот, в котором он
живёт, то ему откроется величайший из миров.
В давние времена в конце года фермер проводил январь, февраль и март, охотясь на
кроликов на холмах. И хотя он был бедным крестьянином, он всё же имел эту степень
свободы. Новогодние каникулы продолжались около трёх месяцев. Постепенно свободное
время сократилось до двух месяцев, одного месяца, и теперь Новогодние праздники свелись
к трём дням.
Уменьшение Новогодних каникул говорит о том, насколько более занят стал фермер и
насколько потерял своё физическое и духовное благополучие. В современном сельском
хозяйстве фермер не имеет времени, чтобы писать стихи или сочинять песни.
Однажды, когда я чистил маленькую деревенскую часовню, я был удивлён, увидев
висящие на стенах маленькие дощечки. Счистив пыль и разглядывая смутные и выгоревшие
буквы, я смог обнаружить дюжину стихов «хайку». Даже в такой маленькой деревне, как эта,
двадцать или тридцать человек сочиняли «хайку» и дарили их как приношения. Это говорит
о том, как много свободного времени было у людей в их жизни в старые времена. Возраст
некоторых стихов должен был составлять несколько столетий. Поскольку это было так
давно, авторы этих стихов, очевидно, были бедные крестьяне, но они всё же имели время,
чтобы писать «хайку».
Теперь в этой деревне нет ни одного человека, который имел бы достаточно времени,
чтобы заниматься поэзией. В течение холодных зимних месяцев только некоторые жители
деревни могут найти время выбраться на день или два для охоты на кроликов. Во время
отдыха теперь в центре внимания находится телевизор и совсем нет времени для простых
развлечений, которые разнообразят и обогащают жизнь фермера. Вот что я имею в виду,
когда говорю, что сельская жизнь становится бедной и слабой духовно, она становится
заполненной только материальными заботами. Лао Цзы, таоистский мудрец, говорит, что
полная и упорядоченная жизнь может быть только в маленькой деревне. Бодхидхарма,
основатель философии Дзен, прожил девять лет в уединении в пещере. Заботы о делании

денег, расширении, развитии, выращивании коммерческих культур и продаже их – это не
дело фермера. Быть здесь, заботиться о маленьком поле, в совершенном обладании свободой
и полнотой каждого дня – таков должен быть самобытный уклад жизни в деревне.
Разбить жизненный опыт пополам и назвать одну сторону физической, а другую
духовной – это слишком узко и неясно. Жизнь человека не зависит от пищи. В конечном
счёте, мы не знаем, что такое пища. Было бы лучше, если бы люди перестали вообще думать
о пище. Было бы также хорошо, если бы люди перестали заботиться об открытии «истинного
смысла жизни». Мы никогда не узнаем ответа на «вечные» духовные вопросы, но было бы
хорошо «ничего не понимать». Мы были рождены и живём на земле, чтобы лицом к лицу
противостоять реальности жизни.
Жизнь – не больше, чем результат того, что ты рождён. Что бы люди ни ели, чтобы
жить – всё это ничего более, чем что-то придуманное ими. Мир существует таким образом,
что если люди откажутся от своей воли и доверятся природе, нет причин ожидать голода.
Жить здесь и теперь – это настоящая основа человеческой жизни. Когда наивное
научное знание становится основой человеческой жизни, люди начинают жить так, как будто
они зависят только от крахмала, жиров и белков, а растения – от азота, фосфора и калия.
И учёные, независимо от того, как долго они исследуют природу, независимо от того,
как далеко они продвинулись в исследованиях в конце концов приходят только к осознанию
того, как совершенна и таинственна природа. Думать, что путём исследований и открытий
человечество может создать что-то лучшее, чем природа, – это иллюзия. Я думаю, что люди
борются только за то, чтобы узнать, что они могут называть громадной непостижимостью
природы.
То же самое относится к фермеру и его работе: следуй природе, и всё будет хорошо.
Земледелие всегда считалось священной работой. Когда человечество удалилось от этого
идеала, возникло современное коммерческое земледелие. Когда фермер начал выращивать
культуры, чтобы делать деньги, он забыл настоящие источники земледелия.
Конечно, коммерсанты играют определённую роль в обществе, но прославление
коммерческой активности уводит людей в сторону от осознания истинного источника
жизни. Земледелие, как занятие, связанное с природой, тесно связано с этим источником.
Многие фермеры не знают природы, даже живя и работая в окружении природы, но мне
кажется, что земледелие представляет много возможностей для более глубокого постижения
жизни.
«Принесёт ли ветер осень и дождь, но сегодня я буду работать в полях». Это слова
старой народной песни. Они выражают правду сельского труда как способа жизни. Не имеет
значения, каким будет урожай, будет или нет достаточно пищи, радость в том, чтобы просто
сеять семена и заботливо выращивать растения.

Различные школы натурального земледелия
Я не особенно люблю слово «работа». Человеческие существа – единственные
животные, которые должны работать. Я думаю, что это наиболее бессмысленная вещь в
мире. Другие животные просто живут в жизни, но люди работают как сумасшедшие, думая,
что они должны это делать, чтобы остаться живыми. Чем больше труд, чем сложнее задача,
тем прекраснее, считают они. Было бы хорошо отказаться от такого образа мыслей и жить
лёгкой удобной жизнью, имея много свободного времени. Я думаю, что образ жизни
животных в тропиках, когда утром и вечером они выходят в поисках еды, а всё
послеполуденное время проводят в дремоте, – я думаю, что это чудесная жизнь.
Для человека такая простая жизнь была бы возможна, если бы он работал ровно
столько, сколько надо, чтобы удовлетворить его ежедневные потребности. В такой жизни
работа – это не та работа, которую люди обычно подразумевают под этим словом, но это
просто исполнение того, что должно быть сделано.
Моя цель – обеспечить движение в этом направлении. Это также цель семи или восьми
молодых людей, которые живут коммуной в хижинах на горе и помогают делать ежедневную
фермерскую работу. Эти молодые люди хотят стать фермерами, чтобы основать новые
деревни и коммуны, и попробовать вести этот образ жизни. Они пришли на мою ферму,
чтобы научиться фермерскому мастерству, которое им понадобится, чтобы выполнить этот
план.
В последнее время в стране возникло несколько коммун. Некоторые называют их
сборищем хиппи. Хорошо, их можно рассматривать и с такой точки зрения тоже. Но, живя и
работая вместе, пытаясь найти путь назад к природе, они служат моделью «нового фермера».
Они понимают, что прочно укорениться означает жить урожаем своей собственной земли.
Коммуны, которые не в состоянии обеспечить себя пищей, долго не протянут.
Многие из этих молодых людей едут в Индию или во французскую Ганди Виллидж,
живут в кибутцах в Израиле, или посещают коммуны в горах и пустынях Американского
Запада. Среди них есть такие, как группа на острове Суванозе в цепи Тохарских островов в
Южной Японии, которые пробуют новые формы семейной жизни и испытывают
возможность тесного племенного образа жизни. Я думаю, что движение этой горстки
людей – это путь к лучшим временам. Именно среди этих людей натуральное земледелие
стало быстро распространяться и набирать силу.
Кроме того, различные религиозные группы начали перенимать методы натурального
земледелия. В поисках фундаментальной основы природы человека, независимо от того,
каким путём вы идёте, вы должны начать с проблемы здоровья. Путь, ведущий к
правильному пониманию, включает проживание каждого дня просто и открыто и питание
здоровой натуральной пищей. Отсюда следует, что натуральное земледелие для многих
людей – это лучшее начало пути.
Я сам не принадлежу ни к какой религиозной группе и могу свободно обсуждать свои
взгляды с любым человеком. Я не придаю большого значения различиям между
христианами, буддистами, шинтоистами и другими религиозными группами, но мне
интересно, что моя ферма привлекает людей с глубокими религиозными убеждениями. Я
думаю, причина в том, что натуральное земледелие в отличие от других типов земледелия,
основано на философии, которая простирается значительно дальше представлений о

почвенных анализах, рН и урожае.
Не так давно один молодой человек из Парижского Центра Органического земледелия
поднялся на гору и мы целый день проговорили с ним. Я слышал о делах во Франции и знал,
что они планируют международную конференцию по органическому земледелию и, готовясь
к этой конференции, этот француз посещал органические и натуральные фермы во всём
мире. Я показал ему сад, и затем мы сидели за чашкой полынного чая и обсуждали
некоторые из моих наблюдений, сделанные за прошедшие тридцать лет.
Во-первых, я сказал, что если вы рассмотрите принципы органического земледелия,
популярного на Западе, вы обнаружите, что они мало отличаются от традиционного
Восточного земледелия, практикуемого в Китае, Корее и Японии в течение многих веков. Все
японские фермеры использовали этот тип земледелия ещё в период Эры Мейджи и Тайшо
(1868–1962 гг.) и до конца Второй Мировой войны. Это была система, в которой большую
роль играл компост и утилизация человеческих и животных отходов. Это было интенсивное
земледелие, и оно включало такие приёмы, как ротацию культур, совместные посевы разных
культур и зелёное удобрение. Поскольку площадь была ограничена, поля никогда не
оставались пустыми и чередование посевов и уборок соблюдалось с большой точностью. Все
органические остатки шли в компост и возвращались на поля. Использование компоста
официально поощрялось, и сельскохозяйственные исследования касались, главным образом,
органического удобрения и техники компостирования.
Таким образом, животные, растения и человек слились в сельском хозяйстве в единое
тело, составлявшее основу японского земледелия вплоть до нашего времени. Можно сказать,
что органическое земледелие в том виде, в каком оно существует на Западе, имеет своей
отправной точкой традиционное земледелие Востока.
Далее я сказал, что среди методов натурального земледелия можно выделить два типа:
трансцендентальное натуральное земледелие и ограниченное натуральное земледелие
относительного мира (это мир в том виде, в котором его представляет себе интеллект). Если
бы мне пришлось говорить об этом в терминах буддизма, я назвал бы эти два типа
соответственно натуральное земледелие Махаяна и Хинаяна.
Натуральное земледелие Махаяна возникает само по себе там, где существует единство
между человеком и природой. Оно приспособлено к природе, как она есть, и к разуму, как он
есть. Оно возникло из убеждения, что если человек временно откажется от своей воли и
позволит природе руководить собой, то в ответ природа обеспечит его всем необходимым.
Можно пояснить это на простой аналогии: в трансцендентном натуральном земледелии
отношения между человеком и природой можно сравнить с отношением между мужем и
женой, соединившихся в совершенном браке. Брак не подарен и не получен от кого-то,
совершенная пара создаёт сама себя.
С другой стороны, ограниченное натуральное земледелие также следует путём природы,
это осознанные попытки органического и других методов соответствовать законам природы.
Хотя земледелец по-настоящему любит природу и всерьёз полагается на неё, отношения
между ними всё же не очень прочные.
Узкий взгляд на натуральное земледелие говорит, что для фермера хорошо вносить
органическое удобрение в почву и что хорошо выращивать животных и что это лучший и
наиболее эффективный способ использования природы. С точки зрения личной практики,
это прекрасно, но если сохранится только такой подход, то дух истинного натурального
земледелия умрёт. Этот тип ограниченного натурального земледелия аналогичен школе

фехтования, известной как «Школа одного удара», которая учит побеждать благодаря
высокому мастерству и благодаря осознанному применению техники. Современное
индустриальное земледелие подобно «Школе одного удара», оно верит, что победа может
быть выиграна путём возведения мощного заграждения из ударов меча.
Чистое натуральное земледелие, наоборот, «Школа без удара», которая не стремится к
победе. Претворение в практику принципа «ничего-не-делания» – это единственная цель, к
которой должен стремиться фермер. Лао Цзы говорил о неактивной природе. Я думаю, что
если бы он был фермером, он наверняка занимался бы натуральным земледелием. Я думаю,
что путь Ганди – этот «метод без метода», действующий в состоянии сознания не
побеждающего, не протестующего, близок натуральному земледелию. Если люди поймут, что
они теряют радость и счастье в попытках овладеть ими, то они осознают также основу
натурального земледелия. Единственная истинная цель земледелия – не выращивание
растений, а культивирование и совершенствование человека (в этом параграфе м-р Фукуока
рисует различие между техническими приёмами, которые возникают спонтанно, как
выражение гармонии личности и природы, в процессе его повседневных дел, свободно от
волевого давления интеллекта).

Глава IV Заблуждения относительно пищи
Молодой человек, который три года прожил в хижине на горе, однажды сказал: «Когда
люди говорят «натуральная пища», я не знаю, что они имеют в виду».
Каждый знаком со словами «натуральная пища», но не всегда ясно понимают, что это
такое на самом деле. Многие считают, что пища, которая не содержит ни химикатов, ни
искусственных добавок – это и есть натуральная диета, а другие довольно смутно
представляют себе, что натуральное питание – это потребление пищи, которую можно найти
в природе.
Если вы спросите, входит ли в понятие натурального питания использование огня или
соли для приготовления пищи, ответы будут самые разные. Если считать, что питание
человека в примитивные времена, состоящее только из диких растений и животных, – это
натуральное питание, тогда использование огня и соли не может быть названо натуральным.
Но если на это возразить, что достижение в древние времена знания об использовании огня
и соли – это неизбежный путь естественного развития человечества, тогда пища,
приготовленная на огне и с солью, может считаться полностью натуральной. Хороша ли
пища, приготовленная по выработанному людьми способу, или нужно считать хорошей пищу
дикую, такую, какая существует в природе? Можно ли считать натуральными возделываемые
культуры? Где вы проведёте границу между натуральным и ненатуральным?
Можно сказать, что термин «натуральное питание» в Японии возник на основе учения
Саген Ишизука в Эру Мейджи. Его теория была позднее усовершенствована и переработана
м-ром Сакуразава (Джордж Осава) и м-ром Ники. Путь Питания, известный на Западе как
Макробиотический, основан на теории не-двойственности и на концепции инь-ян в книге
И-Цзин. Поскольку это означает питание неполированным рисом, то натуральное питание
обычно представляют как использование в пищу цельного зерна и овощей. Натуральное
питание, однако, не может быть сведено просто к вегетарианству, основанному на
неполированном рисе.
Так что же это такое? Причина всей этой путаницы заключается в существовании двух
видов человеческого знания: различающего и не-различающего (это разделение признают
многие восточные философы). Различающее значение исходит из аналитического, волевого
интеллекта, пытающегося организовать жизненный опыт в логическую систему. М‑р
Фукуока считает, что такой способ мышления отдаляет человека от природы. Он называет
это «ограниченностью человеческого знания и суждения». Не-различающее знание
возникает без сознательных усилий личности, когда жизненный опыт воспринимается
целиком без интерпретации его интеллектом. Различающее знание оказывает существенную
помощь при анализе практических проблем, но м-р Фукуока считает, что в конце концов оно
ведёт к слишком узкой перспективе). Люди обычно думают, что правильное восприятие
мира возможно только через различающее знание. Поэтому слово «природа» как его обычно
употребляют, означает ту сущность природы, которая познаётся различающим интеллектом.
Я отрицаю то пустое изображение природы, которое создано человеческим
интеллектом и явно отличается от самой природы, которая познаётся через опыт неразличающего понимания. Если мы искореним фальшивую концепцию природы, я думаю,
что корни мирового разлада исчезнут.
На Западе естественные науки развиваются из различающего знания, на Востоке

философия инь-ян и философия И-Цзин («Книга перемен», произведение древнекитайской
философии [прим. переводчика]) развивается из того же источника. Но научная правда
никогда не может приблизиться к абсолютной правде, и все философии – это ничего более,
как произвольные интерпретации мира. Природа, которая постигается научным знанием, –
это уже разрушенная природа. Это дух, обладающий скелетом, но не душой. Природа,
которая постигается философским знанием, – это теория, созданная человеческим
мышлением, это дух с душой, но без скелета.
Не-различающее знание может быть реализовано только путём прямой интуиции, люди
пытаются втиснуть его в привычные рамки, называя его инстинктом. На самом деле, это
знание, исходящее из неназываемого источника. Откажитесь от различающего сознания и
переступите пределы мира реальности, если вы хотите знать истинное лицо природы.
Начните с признания, что нет ни запада, ни востока, ни четырёх времён года, ни также инь и
ян.
Я сказал, что можно признать их ценность в качестве временного средства или
указателя направления, но их нельзя считать высшим достижением человеческого духа.
Научная истина и философия – это концепции относительного мира, и в этом мире они
являются носителями истины и имеют ценность. Например, для современных людей,
живущих в относительном мире, разрушающих порядок природы и вызывающих коллапс их
собственного тела и духа, система инь-ян может служить удобным и эффективным
руководством для восстановления этого порядка.
Когда я дошёл до этого, юноша спросил: «Значит, вы не только отрицаете естественные
науки, но также Восточную философию, основанную на инь-ян и И-Цзин?»
Можно сказать, что эта философия служит полезной теорией, позволяющей людям
овладеть правильной системой питания, пока не будет достигнуто истинно натуральное
питание. Но если вы поймёте, что конечная цель человечества – выйти за пределы
относительного мира, чтобы купаться в океане свободы, тогда привязанность к теории
станет для вас несчастьем. Если личность сможет войти в мир, в котором оба аспекта инь и
ян возвращаются к их первоначальному единству, то миссия этих символов на этом
закончится.
Юноша, который недавно прибыл, сказал: «Тогда, если вы становитесь натуральным
человеком, вы можете есть всё, что захотите?»
Если вы ожидаете встретить светлый мир на другом конце туннеля, темнота длится всё
дольше. Если вы больше не хотите есть что-то вкусное, вы сможете оценить истинный вкус
всего, что вы едите. Легко положить на обеденный стол простую пищу натурального
питания, но тех, кто может действительно наслаждаться таким праздником, – немного.

Мандала натуральной пищи
Моё понимание натуральной пищи такое же, как и натурального земледелия. Так же, как
натуральное земледелие подчиняется законам природы, той природы, которая познаётся неразличающим сознанием, так и натуральное питание – это способ питания, при котором
пища, собранная в диком виде или выращенная методом натурального земледелия, и рыба,
пойманная натуральными методами, добывается без преднамеренной деятельности, с
помощью не-различающего сознания.
Хотя я говорю о не-намеренной деятельности и не-методе, я признаю мудрость, которая
вырабатывается с течением времени в процессе повседневной жизни. Использование соли и
огня можно критиковать как первый шаг в отдалении человека от природы, но это просто
естественная мудрость, воспринятая первобытными людьми, и это должно быть
санкционировано как мудрость, дарованная небом.
Сельскохозяйственные культуры, которые возделывались в течение тысяч и десятков
тысяч лет и становились частью жизни людей, не являются полностью продуктом,
рождённым различающим сознанием, их можно считать натуральной пищей. Но новые
сорта, которые создавались не под влиянием естественных условий, но с помощью
агрономической науки, которая отошла очень далеко от природы, так же, как и рыба,
моллюски и домашние животные, производимые в массовом порядке, выпадают из
категории натуральной пищи.
Земледелие, рыболовство, животноводство, повседневные реальности пищи, одежды,
жилья, духовной жизни, всё это должно находиться в единстве с природой.
Я нарисовал следующие диаграммы, которые объясняют натуральное питание,
выходящее за пределы науки и философии. Первая объединяет все виды пищи, которые
наиболее доступны людям. Они разделены на несколько групп. На второй диаграмме пища
разделена на группы в соответствии с её доступностью в разные сезоны года. Эти диаграммы
составляют мандалу натуральной пищи (в восточном искусстве круговая диаграмма,
символизирующая всеобщность и целостность предмета). Из этой мандалы можно видеть,
что источники пищи на нашей земле почти неограниченны. Если человек познаёт пищу
через «не-разум» (буддистский термин, описывающий состояние в котором нет различия
между личностью и внешним миром), даже если он совсем ничего не знает об инь и ян, он
может достичь совершенной натуральной диеты.
Рыбаки и фермеры в японской деревне не очень заинтересуются логикой этой
диаграммы. Они следуют велениям природы, употребляя сезонную пищу из окружающей их
среды.
С ранней весны, когда семь трав прорастают из земли, фермер может испытать семь
вкусовых ощущений. С ними хорошо сочетается тонкий вкус прудовых улиток и морских
моллюсков.
Сезон зелени наступает в марте. Хвощ, папоротник-орляк, полынь, осмунда и другие
горные растения и конечно молодые листья хурмы, персика и побеги горного ямса – всё это
съедобно. Обладая лёгким тонким вкусом, они представляют изысканное блюдо и могут
также служить приправой. На морском берегу можно собрать морские овощи – бурые
водоросли, нори, водоросли, растущие на скалах, обладающие приятным вкусом и
имеющиеся в изобилии в весенние месяцы.

Во время появления молодых побегов бамбука каменный окунь, морской лещ и
полосатая рыба-свинья имеют свой самый лучший вкус. Сезон цветения ирисов отмечают
ленточной рыбой и макрелью «сашими». Зелёный горошек, снежный горошек, лимская
фасоль, бобы очень вкусны, если есть их прямо из стручков или варить целиком так же, как
неочищенный рис, пшеницу и ячмень.
Ближе к концу сезона дождей (на большей части Японии сезон дождей продолжается с
июня до середины июля) запасают японские сливы, а землянику и малину в это время можно
собирать в изобилии. В это время возникает естественная потребность в остром вкусе лука и
сочных фруктах, таких как мушмула, абрикосы и персики. Плоды мушмулы – не
единственная съедобная часть этого растения. Семена в размолотом виде можно
использовать для приготовления «кофе»; из листьев делают чай, который является также
прекрасным лекарственным средством. Из листьев персика и хурмы делают тоник,
способствующий долгожительству.
Под ярким летним солнцем есть дыни с мёдом в тени большого дерева – это прекрасное
времяпрепровождение. В это время созревают многие летние овощи: морковь, шпинат, редис
и огурцы. Для преодоления летней расслабленности тела полезно растительное масло или
масло сезама. Блюда, приготовленные из зерна, хорошо удовлетворяют уменьшающийся
летом аппетит. Летом часто готовят ячменную лапшу, различной формы и размера. Летом
созревает гречиха. Это растение даёт пищу, которая хорошо усваивается в этот сезон.
Ранняя осень – счастливое время, с соей и маленькими красными бобами азуки,
большим количеством фруктов и овощей и различными хлебными злаками, поспевающими в
это время. Пироги из проса украшают праздник осенней луны. Позже осенью варят кукурузу
и рис с красными бобами, грибы «матсутаке» или наслаждаются каштанами. Очень важно
отметить, что рис, вбирающий в себя солнечные лучи всё лето, долго зреет осенью. Это
значит, что главная пища, получаемая в большом количестве и богатая калориями,
обеспечена на холодные зимние месяцы.
С первыми морозами люди начинают посматривать на прилавки с рыбой.
Глубоководные рыбы, как желтохвост и тунец ловятся в этот сезон. Интересно, что японская
редька и листовые овощи, имеющиеся в это время в изобилии, хорошо идут с этими рыбами.
Новогодние праздничные блюда готовятся главным образом из той пищи, которая
замаринована и засолена специально для больших праздников. Солёный лосось, селёдка,
яйца, красный морской лещ, омар, ламинария и чёрные бобы подаются каждый год на этот
праздник в течение многих веков.
Выкапывая редьку и репу, которые оставлены в грунте под покровом почвы и снега,
доставляют приятное дополнение к меню в зимний сезон. Зерновые и различные бобовые, а
также мисо и соевый соус – основные продукты питания, которые в это время всегда под
рукой. Вместе с капустой, редькой, тыквой и бататами, запасёнными осенью, в эти месяцы
зимнего холода употребляются и другие разнообразные виды пищи. Лук-порей и дикий лукшалот хорошо сочетаются с тонким вкусом устриц и морского огурца, которые можно
собирать в это время.
В ожидании весны можно пойти поискать побеги мать-и-мачехи и съедобные листья
ползучей земляничной герани, выглядывающие из-под снега. С появлением водного кресса,
пастушьей сумки, звездчатки и других диких трав урожай натуральных весенних овощей
можно собирать прямо под окном кухни. Таким образом, придерживаясь простого питания,
собирая в каждый сезон ту пищу, которая доступна, наслаждаясь её здоровым и полезным

вкусом, местные крестьяне используют всё, что даёт природа.
Крестьяне знают тонкий вкус пищи, но им не понятен таинственный вкус природы.
Или, вернее, он им понятен, но они не могут выразить это словами.
Натуральное питание лежит прямо у нас под ногами.

Культура питания
Если спросить, зачем мы едим, то мало кто сможет ответить что-нибудь, кроме того, что
пища необходима для поддержания жизни и роста тела человека. Однако есть и более
глубокий аспект во взаимоотношениях между пищей и духовной жизнью человека. Для
животного достаточно есть, играть и спать. Для человека также очень важно наслаждаться
здоровой пищей, простой повседневной жизнью и спокойным сном.
Будда сказал: «Форма – это пустота, и пустота – это форма». Поскольку «форма» в
буддистской терминологии означает материю или «вещи», а «пустота» – это разум, то он
сказал, что материя и разум – это одно и то же. Вещи имеют различные цвета, формы и вкус,
и человеческий разум мечется из стороны в сторону, привлекаемый различными качествами
вещей. Но на самом деле, материя и разум едины.
Цвет. В мире есть семь основных цветов, но если эти семь цветов смешать, получится
белый цвет. Пропущенный через призму белый цвет распадается снова на семь цветов. Когда
человек воспринимает окружающее, «не сознавая» его, цвет исчезает. Есть не-цвет. Только
если смотреть на мир с точки зрения «семи-цветного» различающего сознания, семь цветов
появляются. Вода может подвергаться бесчисленным изменениям, но вода всегда остаётся
водой. Точно так же, хотя сознающий разум подвергается изменениям, первоначальное
неподвижное сознание не изменяется. Когда человек восхищается семью цветами, сознание
легко сбить с толку. Человек воспринимает окраску листьев, цветов и фруктов, а основа
цвета остаётся незамеченной.
Это верно также для пищи. В этом мире много натуральных продуктов, которые человек
может использовать в пищу. Сознание выделяет эту пищу и определяет, имеет ли она
хорошие или плохие качества. Люди сознательно отбирают то, что, по их мнению, им нужно.
Процесс отбора препятствует осознанию основы человеческого питания, которое небеса
прописали для каждого места и каждого сезона.
Краски природы, как соцветия гортензии, легко изменяются. Тело природы постоянно
трансформируется. На одинаковых основаниях это можно рассматривать как бесконечное
движение или как «неподвижное движение». Когда при отборе пищи действует разум,
понимание природы становится фиксированным, и трансформации природы, такие как
сезонные изменения, игнорируются.
Цель натурального питания – не создание знающих людей, которые могут дать
обоснованное объяснение и ловко разобраться в различных видах пищи, но создание
незнающих людей, которые берут пищу без сознательно делаемых различий. Это не идёт в
разрез с законами природы. С осознания «без сознания», не потерявшись в тонких различиях
форм, с восприятия окраски, бесцветной, как цвет, начинается правильное питание.
Вкус. Люди говорят: «Вы не узнаете вкус пищи, пока не попробуете её». Но даже если
вы попробуете её, вкус пищи может оказаться различным в зависимости от времени, условий
и состояния человека, который пробует,
Если вы спросите учёного, какие вещества определяют вкус, он будет пытаться
определить это, выделяя различные компоненты и определяя пропорции сладкого, кислого,
солёного, горького и острого. Но вкус не может быть определён путём анализа и даже с
помощью языка. Даже если язык воспринимает пять вкусовых ощущений, впечатления от
них собирает и интерпретирует сознание.

Натуральный человек может достигнуть правильного питания благодаря тому, что его
инстинкт находится в правильно работающем состоянии. Он удовлетворяется простой
пищей, она питательна, вкусна и полезна как повседневное лекарство. Пища и духовная
жизнь человека едины.
Современные люди потеряли свой ясный инстинкт и соответственно этому стали
неспособны собирать и наслаждаться семью весенними травами. Они ищут различные
вкусовые ощущения. Их питание становится беспорядочным, расширяется пропасть между
любимой и нелюбимой пищей, и их инстинкт становится всё более и более сбит с толку.
Тогда люди начинают применять сильные приправы к своей пище и придумывать
утончённые способы её приготовления, всё более углубляя расхождение между пищей и
духовной жизнью.
Большинство людей сегодня не знают вкуса риса. Целые зёрна очищают и подвергают
обработке, оставляя лишь безвкусный крахмал. Полированный рис теряет уникальный
аромат и вкус целого риса. Соответственно он требует приправ или соуса и подаётся вместе с
другими блюдами. Люди ошибочно считают, что низкая питательная ценность риса не имеет
значения, поскольку витаминные добавки и другая пища – мясо или рыба – восполняют
недостающие питательные элементы. Вкусная пища вкусна не сама по себе. Пища не будет
казаться невкусной, пока человек не думает, что она не вкусна. Хотя большинство людей
думают, что говядина и курятина восхитительны, для тех, кто по физическим или духовным
причинам решил не есть эту пищу, они отвратительны.
Играя или ничего не делая, дети счастливы. Взрослые с различающим сознанием сами
определяют, что им нужно для счастья, и если дети получают это, они чувствуют себя
удовлетворёнными. Пища, обладающая хорошим вкусом, необходима им не потому, что она
имеет тонкий натуральный аромат и полезна для тела, а потому, что её вкус был обусловлен
представлением, что эта пища вкусна.
Пшеничная лапша восхитительна, но чашка лапши быстрого приготовления из торгового
автомата отвратительна. Но реклама этой лапши отрицает представление о её плохом вкусе,
и для многих людей эта малопривлекательная лапша кажется почему-то вкусной.
Существует сказка о том, как обманутые лисой люди съели лошадиный навоз. В этом
нет ничего смешного. Люди в наше время едят своим сознанием, а не своим телом. Многие
люди не заботятся о том, есть ли глутамат натрия в их пище, но они ощущают вкус только
кончиком языка и поэтому их легко одурачить.
Вначале люди ели просто потому, что они были живы и потому что пища была вкусной.
Современные люди пришли к мысли, что если они не будут готовить пищу с приправами, она
будет безвкусной. Если вы не будете пытаться сделать еду вкусной, вы обнаружите, что
природа уже сама сделала это.
Прежде всего, следует жить таким образом, чтобы пища сама по себе имела хороший
вкус, но сегодня все усилия направлены на то, чтобы сделать её вкусной.
Люди стараются придать хороший вкус хлебу – и хороший хлеб исчезает. В попытках
создать богатую, роскошную еду они делают бесполезную пищу и всё же никак не могут
удовлетворить свой аппетит. Лучшие методы приготовления пищи служат тому, чтобы
сохранить её естественный тонкий аромат. Жизненная мудрость с давних времён научила
людей делать разные виды овощных заготовок: высушенных на солнце, засоленных, отрубей,
мисо, в которых сохраняется естественный вкус овощей.
Искусство приготовления пищи начинается с соли и огня. Если пища приготовлена тем,

кто понимает основы поварского дела, она сохраняет свой натуральный вкус. Если
приготовленная пища кажется странной и экзотической, если цель такого приготовления –
просто доставить наслаждение нёбу, то это фальшивое поварское мастерство.
Обычно думают, что культура – это что-то созданное, поддерживаемое и развиваемое
только человеческими усилиями. Но культура обычно возникает из содружества человека и
природы. Там, где осуществилось единство человеческого общества и природы, культура
принимает свойственную ей форму. Культура всегда была тесно связана с повседневной
жизнью и таким образом передавалась будущим поколениям и была сохранена до настоящего
времени.
То, что рождено человеческой гордостью и потребностью в наслаждении не может
считаться истинной культурой. Истинная культура рождена внутри природы, она проста,
скромна и чиста, лишившись истинной культуры, человечество исчезнет.
Когда люди отказываются от натуральной пищи и вместо этого начинают потреблять
очищенную пищу, тогда общество становится на путь к своему собственному разрушению.
Происходит это потому, что такая пища не является продуктом истинной культуры. Пища –
это жизнь, а жизнь не должна ни шагу отступать от природы.

Жизнь на одном хлебе
Нет ничего лучше, чем есть вкусную пищу, но для большинства людей еда – это только
средство укрепления здоровья, источник энергии для работы и путь к долгой жизни. Матери
часто заставляют своих детей есть, даже если им не нравится еда, только потому, что это
«полезно» для них.
Но питание не может быть отделено от ощущения вкуса. Питательная пища, полезная
для человеческого тела, возбуждает аппетит и вкусна сама по себе. Правильное питание
неотделимо от хорошего вкуса.
Не так давно дневная пища фермера в этом районе состояла из риса и ячменя с мисо и
маринованными овощами. Эта диета способствовала долгой жизни, крепкому телосложению
и хорошему здоровью. Варёные овощи и варёный рис с красными бобами подавали раз в
месяц как на праздник. Здоровое крепкое тело фермера было способно достаточно хорошо
поддерживать себя на этой диете.
Традиционное восточное питание из неполированного риса и овощей сильно отличается
от питания большинства западных обществ. Западная наука о питании считает, что если
определённое количество крахмала, жиров, белков, солей и витаминов не съедается каждый
день, это значит, что не постигнуто сбалансированное питание, без которого невозможно
сохранить хорошее здоровье. Это представление порождает мать, которая набивает
«питательной» пищей рты своего юного потомства.
Кто-то может подумать, что западная диететика с её разработанной теорией и
расчётами не оставляет сомнений относительно правильного питания. На самом деле она
создаёт гораздо больше проблем, чем разрешает. Одна из проблем состоит в том, что в
западной науке о питании не сделано никаких попыток скоординировать питание с
природными циклами. В результате создаётся диета для как бы изолированного от природы
человеческого существа. Это часто служит причиной появления страха перед природой и
чувства опасности.
Другая проблема заключается в том, что полностью забываются духовные и
эмоциональные ценности, хотя пища непосредственно связана с духовной и эмоциональной
жизнью человека. Если человеческое существо рассматривать просто как физиологический
субъект, невозможно выработать правильное понимание диеты. Когда части и обрывки
информации собирают и кое-как соединяют вместе, возникает несовершенная диета,
которая уводит нас от природы.
«Внутри одной вещи лежат все вещи, но если все вещи собрать вместе, может
возникнуть не одна вещь». Западная наука неспособна понять это правило восточной
философии. Человек может анализировать и исследовать бабочку, сколько ему хочется, но он
не может создать бабочку.
Если бы западная научная диета применялась практически в широком масштабе, как вы
думаете, какие проблемы при этом возникли бы? Высококачественная говядина, яйца,
молоко, овощи, хлеб и другая пища должны были бы быть доступны круглый год. Стали бы
необходимыми производство этих продуктов в широких масштабах и организация их
длительного хранения. В Японии введение этой диеты уже заставило фермеров производить
зимой такие летние овощи как лук, огурцы, баклажаны, томаты. Не придётся долго ждать,
когда фермеров попросят собирать хурму весной, а персики – осенью.

Бессмысленно ожидать, что здоровая сбалансированная диета может быть достигнута
просто путём создания большого запаса разнообразной пищи независимо от сезона. По
сравнению с растениями, созревшими в естественных условиях, овощи и фрукты,
выращенные вне сезона в искусственно созданных условиях, содержат мало витаминов и
солей. Не удивительно, что летние овощи, выращенные летом или зимой, не имеют того
вкуса и аромата, которыми обладают плоды, созревшие под солнцем и выращенные
органическим или натуральным методом.
Химические анализы, соблюдение соотношения питательных элементов и другие
соображения того же порядка являются основной причиной ошибки. Пища, рекомендуемая
современной наукой, далека от традиционной восточной диеты, и она подрывает здоровье
японского народа.

Общие принципы питания
В этом мире существует четыре основных класса диет.
1. Неопределённая диета, приспособленная к обычным желаниям и вкусовым
предпочтениям. Люди, следующие этой диете, неопределённо колеблются туда и сюда в
ответ на свои причуды и фантазии. Эта диета может быть названа самоудовлетворяющейся и
лёгкой.
2. Стандартное питание большинства людей, вытекающее из биологических
требований. Питательную пищу едят с целью поддержания жизни тела. Эту диету можно
назвать материалистической и научной.
3. Диета, основанная на духовных принципах и идеалистической философии.
Большинство «натуральных» диет попадает в эту категорию. Их можно назвать диетами
первичными.
4. Натуральная диета, следующая воле неба. Отрицая все человеческие знания, эта диета
может быть названа не-различающей.
Сначала люди отказываются от пустой диеты, которая служит источником
бесчисленных болезней. Затем, разочаровавшись в научной диете, которая является простой
попыткой поддержать жизнь, многие склоняются к первичной диете. Наконец, пройдя эти
первоначальные этапы, человек приходит к не-различающей диете.
Диета не-различающая. Жизнь человека поддерживается не его собственной силой.
Природа даёт рождение человеческим существам и помогает им сохранить жизнь. В этом
состоят взаимоотношения человека и природы. Пища – это дар неба. Люди не создали пищу
из природы, её дарует небо.
Когда пища, тело, сердце и сознание вступают в совершенное единство с природой, то
становится возможным соблюдение натуральной диеты. Тело, следуя своему собственному
инстинкту, свободно в выборе той еды, которая имеет хороший вкус, и отказывается от того,
что невкусно.
Невозможно предписать правила и пропорции для натуральной диеты (нельзя создать
код или систему, с помощью которой можно сознательно ответить на эти вопросы. Природа,
или само тело, служит хорошим гидом. Но это чуткое руководство остаётся неуслышанным
большинством людей из-за шума, создаваемого желаниями, и активности различающего
сознания). Эта диета определяется местными условиями, различными потребностями и
телесной конституцией каждого человека.
Первичная диета. Каждый должен знать, что природа всегда завершена и
сбалансирована в совершенной гармонии в себе самой. Натуральная пища – это нечто целое
и это целое имеет полезный и тонкий вкус.
Люди думаю, что с помощью системы инь и ян они могут объяснить происхождение
Вселенной и трансформации природы. Им кажется, что гармония человеческого тела может
быть создана и сознательно поддерживаться. Но если они пытаются проникнуть глубоко в
суть вещей (а это необходимо при изучении восточной медицины), они вступают во владения
науки, где очень трудно избежать соблазна различающего восприятия.
Увлечённый проницательностью человеческого знания, не осознавая его
ограниченности, практикующий первичную диету, в конце концов приходит к тому, что
целостность этого принципа в его сознании разбивается на отдельные объекты. Но если он

пытается схватить смысл природы с помощью всеобъемлющего и далеко проникающего
видения, то окружающие его мелочи оказываются вне его восприятия.
Типичная диета больного человека. Болезни начинаются, когда человек отдаляется от
природы. Серьёзность заболевания прямо пропорциональна степени отдаления. Если
больной человек возвращается в здоровую среду, то болезнь часто исчезает. Если разрыв с
природой становится экстремальным, число больных людей увеличивается. Тогда
усиливается желание возвращения к природе. Но в поисках возвращения к природе нет
ясного понимания, что такое природа и поэтому все попытки остаются напрасными.
Даже если человек ведёт примитивную жизнь в горах, он может всё же не понимать
истинной реальности. Если вы пытаетесь что-то делать, ваши попытки никогда не достигнут
желаемого результата.
Люди, живущие в городах, оказываются перед лицом непреодолимых трудностей при
попытках соблюдать натуральную диету. Натуральная пища просто недоступна, поскольку
фермеры прекратили выращивать её. Даже если горожане могли бы купить натуральную
пищу, их тело должно быть в состоянии усвоить такую питательную еду.
В такой ситуации, если вы попытаетесь есть здоровую пищу или достигнуть
сбалансированной инь-ян диеты, вам необходимо иметь практически сверхъестественный
разум и силу суждения. В результате возникает далёкий от природы, сложный и странный
вид «натуральной» диеты и человек только ещё больше отдаляется от природы.
Если вы заглянете в магазины «Здоровая пища», вы найдёте поразительно широкий
ассортимент свежих продуктов, упакованных продуктов, витаминов и диетических добавок.
В литературе многие типы диет представлены как натуральные, питательные и самые
лучшие для здоровья. Если кто-то скажет, что полезно варить разные продукты вместе,
найдётся кто-то другой, кто скажет, что продукты, сваренные вместе, делают людей
больными. Кто-то подчёркивает важное значение соли, другой говорит, что слишком много
соли вредно. Если кто-то избегает есть фрукты, считая, что это пища для обезьян, кто-то
другой говорит, что фрукты и овощи – самая лучшая пища, обеспечивающая долголетие и
счастье. Все эти мнения можно было бы назвать правильными, каждое для своего времени и
каждое – для своих условий. Но это только вызвало бы недоумение, а сомневающихся все эти
теории привели бы в ещё большее замешательство.
Природа находится в постоянном движении, ежесекундно изменяясь. Люди не в
состоянии уловить её истинный облик. Лицо природы непознаваемо. Попытки поймать эту
непознаваемость различными теориями, формализованными учениями похожи на попытки
поймать сачком ветер.
Человечество подобно слепому человеку, который не знает, куда он направляется. Он
нащупывает путь с помощью трости научного знания, в зависимости от инь и ян, определяя
направление.
Вот что я хочу сказать: не ешьте пищу головой, то есть откажитесь от различающего
сознания. Я надеюсь, что мандала питания, которую я нарисовал раньше, может служить
руководством, показывающим взаимоотношения разных продуктов питания между собой и
между питанием и человеческим существом. Но вы можете также выбросить эту мандалу,
взглянув на неё один раз.
Прежде всего, следует уделить внимание развитию тонкой чувствительности, которая
даст возможность телу самому выбирать пищу. Думать только о самой пище и забывать о
духовной жизни-то же самое, что посетить храм, читать сутры и забыть о Будде. Вместо того,

чтобы изучать философские теории о том, как достигнуть понимания правильного питания,
лучше прийти к теории изнутри повседневной диеты.
Врачи заботятся о больных людях, о здоровых людях заботится природа. Вместо того,
чтобы заболеть, а потом заняться натуральным питанием, чтобы выздороветь, лучше жить в
естественной среде, где человек не болеет.
Молодые люди, которые пришли, чтобы жить в хижинах на горе и живут примитивной
жизнью, едят натуральную пищу и занимаются натуральным земледелием, знают о главной
цели человека, и они живут в согласии с ней, достигая этого наиболее прямым путём.

Пища и земледелие
Эта книга о натуральном земледелии неизбежно включает некоторые соображения о
натуральной пище. Это объясняется тем, что пища и земледелие – это две стороны одного
тела. Совершенно очевидно, что если не практиковать натуральное земледелие, то
потребители не получат натуральные продукты. Но если натуральное питание не войдёт
прочно в жизнь, то фермер не будет знать, что ему следует выращивать.
Пока люди не станут натуральными людьми, не может быть ни натурального
земледелия, ни натуральной пищи. В одной хижине на горе я оставил слова, написанные на
сосновой дощечке над очагом: «Правильная пища, правильные действия, правильное
сознание» (эти слова взяты из Буддистского восьмиступенчатого Пути духовной реализации).
Люди послушно видят мир, как место, где «прогресс» вырастает из суматохи и
беспорядка. Но бесцельное и деструктивное развитие влечёт за собой беспорядок в мыслях и
приводит не меньше, чем к вырождению и кризису человечества. Если не дать себе ясного
отчёта в том, что неподвижный источник всей этой активности есть природа, то
восстановление нашего здоровья окажется невозможным.

Глава V Глупость рядится под находчивость
Осенние ночи длинные и прохладные. Можно было бы хорошо провести время, глядя на
мерцающие угли, обхватив руками тёплую чашку с чаем. Сидя у огня, я говорил, что самое
важное остаётся без внимания, что человечество невежественно, что нет ничего, за что
стоило бы бороться, и что бы ни было сделано – всё это пустые усилия. Меня спросили, как
могу я говорить всё это и в то же время продолжать такие беседы? Точно также, как если бы
я собрался написать что-нибудь, то единственное, что я мог бы написать – это, что писать
бесполезно. Мои собеседники зашли в тупик.
Я не придаю такого значения моему прошлому, чтобы писать о нём, и я не так мудр,
чтобы предсказывать будущее. Поддерживая огонь и ведя сердечную беседу о дневных делах,
разве могу я ожидать, что кто-то примет всерьёз глупое мнение старого фермера?
В верхней части сада, глядя на залив Матсуяма и широкую долину Дого, стоят несколько
маленьких глиняных хижин. Там собралась горстка людей и вместе они ведут простую
жизнь. Там нет современных удобств. Проводя мирные вечера при свете свечи или лампы,
они живут простыми потребностями: неполированный рис, овощи, простая одежда, чашка
для еды. Они пришли из разных мест, некоторое время живут здесь и потом уходят.
Среди гостей есть научные работники, студенты, учёные, фермеры, хиппи, поэты и
путешественники, молодые и старые, мужчины и женщины различных типов и
национальностей. Большинство из тех, кто остаётся долгое время, это молодые люди,
нуждающиеся в периоде самоанализа.
Моя функция состоит в том, чтобы заботиться об этом приюте на краю дороги, подавать
чай путешественникам, которые приходят и уходят. И когда они остаются, чтобы помогать в
полевых работах, я рад слушать их рассказы о событиях, происходящих в мире.
Это звучит хорошо, но на самом деле это не такая лёгкая жизнь. Я пропагандирую
земледелие «ничего-не-делания», и поэтому многие люди приходят, думая, что они найдут
утопию, где можно жить без необходимости вставать с постели. Эти люди бывают очень
удивлены. Им приходится в раннем утреннем тумане носить воду из источника, колоть дрова,
пока их руки не станут красными и не покроются мозолями, работать по колено в жидкой
глине. Многие скоро не выдерживают это и бросают.
Сегодня, когда я наблюдал за работой молодых людей в маленькой хижине, пришла
наверх молодая женщина из Фунабаши. Когда я спросил, почему она пришла, она сказала: «Я
просто пришла, вот и всё. Больше я ничего не знаю». Беспечная молодая леди имеет свои
соображения о себе. Я потом спросил: «Если вы знаете о своей неосведомлённости, то
сказать нечего. Верно? Приходя к пониманию мира через силу различения, люди теряют
представление о его смысле. Не потому ли мир в таком кризисе?»
Она ответила тихо: «Да, если вы говорите так». «Может быть вы не имеете ясного
представления о том, что такое просвещение? Какие книги вы читали до того, как пришли
сюда?» Она отрицательно покачала головой.
Люди учатся, так как они думают, что они не понимают чего-то, но учение совсем не
ставит своей целью помочь человеку понять. Они усердно учатся только для того, чтобы в
конце понять, что люди не могут знать ничего, что понимание недостижимо для них.
Обычно люди думают, что слово «непонимание» применяют в таком, например, смысле,
когда говорят, что девять вещей вы понимаете, но есть одна вещь, которую вы не понимаете.

Но, намереваясь понять десять вещей, вы на самом деле не понимаете даже одной. Если вы
знаете сто растений, вы на самом деле не «знаете» ни одного. Люди изо всех сил стараются
понять, считают, что они понимают, и умирают не зная ничего.
Молодые люди прервали свои плотничные работы, сели на траву под большим
мандариновым деревом и смотрели на белые облака в южной части неба. Люди думают, что
когда они обращают свои глаза к небу, они видят небеса. Они отделяют оранжевые плоды от
зелёных листьев и говорят, что они знают зелёный цвет листьев и оранжевый – плодов. Но с
того момента, когда человек делает различие между зелёным и оранжевым, истинный цвет
исчезает.
Люди думают, что они понимают вещи, если они близко узнают их. Но это только
поверхностное знание. Это знание астронома, который знает названия звёзд, ботаника,
который знает классификацию листьев и цветов, художника, который знает эстетику
зелёного и красного. Это не значит знать природу – землю и небо, зелёное и красное.
Астроном, ботаник и художник не более, чем восприняли впечатление и интерпретировали
его, каждый в рамках своего собственного сознания. Чем активнее их интеллект, тем более
они отдаляются от природы и тем труднее им становится вести натуральную жизнь.
Трагедия состоит в том, что в своём невежестве люди пытаются подчинить природу
свой воле. Человеческие существа могут разрушить природные формы, но они не могут
создать их. Разделение, фрагментарное и неполное понимание всегда образуют исходную
позицию человеческого знания. Неспособные воспринять природу в её целостности, люди
не могут сделать ничего лучше, чем сконструировать её неполную модель, и тем самым
обманывают себя, думая, что они создали что-то натуральное.
Всё, что человек должен сделать, чтобы познать природу, – это осознать, что он на
самом деле ничего не знает, что он не способен что-нибудь познать. Тогда можно ожидать,
что он потеряет интерес к различающему знанию. Когда он отвергнет различающее знание,
не-различающее знание само возникнет внутри него. Если он не пытается думать о знании,
если он не заботится о понимании, то придёт время, когда он поймёт. Нет другого пути,
кроме как через разрушение «эго», отбросив мысль, что человечество существует помимо
неба и земли.
«Это значит быть глупым, вместо того, чтобы быть умным». Я рассердился на молодого
парня, у которого на лице было написано мудрое самодовольство. «О чём говорит твой
взгляд? Глупость проявляется как ум. Можешь ли ты знать наверняка, глуп ты или умён, или
ты пытаешься стать умным парнем на глупый лад? Ты не можешь стать умным, не можешь
стать глупым, оставаясь в покое. Разве это не то состояние, в которым ты теперь?»
Я был сердит сам на себя за повторение снова и снова одних и тех же слов, слов,
которые никогда не могли заменить мудрость молчания, слов, которые я сам не мог понять.
Осеннее солнце висело низко над горизонтом. Сумерки наступили у подножия старого
дерева. Молча возвращались юноши в свои хижины для вечерней трапезы. Я тихо следовал за
ними, оставаясь в тени.

Кто глупец?
Говорят, что нет создания более мудрого, чем человек. Обладая этой мудростью, люди
стали единственными животными, способными вести ядерную войну.
Однажды директор магазина натуральных продуктов, который находится перед
станцией Осака, поднялся на гору и привёл с собой семь спутников, как семь богов
счастливой судьбы. В полдень, когда мы угощались импровизированным блюдом из овощей с
рисом, один из них сказал следующее: «Среди детей всегда есть беззаботный ребёнок,
который радостно смеётся, когда он писает; есть и другой, который всегда остаётся
«лошадью», когда играют в «лошадь и всадник»; есть и третий, который легко выманивает у
других детей их полдник. Прежде, чем выбрать старосту класса, учитель серьёзно говорит о
качествах, желательных для хорошего лидера, и о необходимости принимать мудрое решение.
Когда выборы проведены, самые молодые весело смеются над тем, кого избрали».
Всех это позабавило, но я не мог понять, чему они смеются. Я думал, что это
совершенно естественно. Если рассматривать вещи с точки зрения выигрыша и проигрыша,
то ребёнок, который в игре всегда оказывается в роли лошади, может считаться
проигравшим, но понятия величия и посредственности нельзя применять к детям. Учитель
думал, что вот этот ребёнок самый умный и выдающийся, но другие дети видели ложное
направление его ума, которое выражалось в притеснении других. Признавать ценности
взрослых – это значит думать, что тот, кто умён и умеет позаботиться о себе, тот
исключителен и что лучше быть исключительным. Тот, кто занимается своими
собственными делами, кто хорошо ест и хорошо спит, кто ни о чём не беспокоится – кажется
мне, ведёт наиболее удовлетворительный образ жизни. Нет никого более великого, чем тот,
кто не пытается достигнуть чего бы то ни было.
В басне Эзопа, когда лягушки просят Бога дать им царя. Бог дарит им бревно. Лягушки
насмехаются над немым бревном, а когда они просят Бога дать им более великого царя, он
посылает им журавля. В конце басни журавль заклевал всех лягушек до смерти.
Если тот, кто стоит впереди – сильная личность, тот, кто следует за ним, должен
бороться и напрягать все свои силы. Если вы пошлёте вперёд обыкновенного парня, тем, кто
идёт за ним, будет легко. Люди думают, что тот, кто силён и умён, – выдающийся человек, и
на этом основании они выбирают премьер-министра, который толкает страну вперёд, как
локомотив.
«Какой тип человека должен быть выбран на пост премьер-министра?» «Немое
бревно, – отвечаю я. – Нет ничего лучше, чем дарума-сан (популярная в Японии детская
игрушка. Это большой баллон с утяжелённым дном, сделанный в форме монаха, сидящего в
медитации). Он сидит, расслабившись в медитации, и может сидеть так много лет, не говоря
ни слова. Если вы толкнёте его, он перевернётся вверх дном, но с постоянством несопротивления он всегда возвращается в прежнее положение. Дарума-сан не сидит просто
так скучая, сложив руки и ноги. Он знает, что вы также должны держать свои руки
сложенными, и поэтому хмуро и молча глядит на людей, которые вытягивают их, чтобы чтото делать».
Если бы вы совсем ничего не делали, мир не мог бы продолжать свой бег вперёд. Что
представлял бы собой мир без развития?
Почему вы должны что-то развивать? Если рост экономики увеличится с 5 % до 10 %,

разве от этого ваше счастье увеличится? Что плохого, если скорость роста будет составлять
0 %? Разве это не означает стабильного состояния экономики? Разве может быть что-нибудь
лучше, чем жить просто и воспринимать это легко?
Люди что-то выискивают. Изучают природу и ставят её себе на службу, думая, что это
идёт на благо человечества. Результатом всего этого в настоящее время является то, что
планета загрязнена, люди находятся в смятении и мы вступаем в эпоху хаоса.
На этой ферме мы практикуем систему «ничего-не-делания» в земледелии и едим
полезные и вкусные зерновые, овощи и цитрусовые. Жить близко к источнику всех вещей – в
этом есть смысл и удовлетворение. Жизнь – это песня и поэзия.
Нагрузка на фермера слишком сильно увеличилась, когда люди начали исследовать мир
и решали, что было бы «благо», если бы мы сделали то или сделали это. Все мои
исследования идут в направлении, как не делать то или это.
Эти тридцать лет научили меня, что фермерам было бы намного лучше, если бы они
почти ничего не делали.
Чем больше люди делают, тем больше развивается общество и тем больше проблем
возникает. Возрастающее запустение природы, истощение ресурсов, приниженность и
разложение человеческого духа, – всё это вызвано тем, что человечество пытается чего-то
достигнуть. Вначале не было никакой причины для прогресса и ничего, что должно было
быть сделано. Мы дошли до такой точки, от которой нет другого пути, как создать
«движение» за то, чтобы ничего не создавать.
«

Я был рождён для того, чтобы ходить в детский сад
»
Молодой человек с небольшой сумкой через плечо не торопясь подошёл к полю, где мы
работали.
«Откуда вы?» – спросил я.
«Вон оттуда».
«Как вы сюда попали?»
«Пришёл пешком».
«Зачем вы пришли сюда?»
«Я не знаю».
Большинство из тех, кто приходит сюда не спешат назвать своё имя и рассказать о своём
прошлом. Они также не очень ясно говорят о цели своего прихода. Поскольку многие их них
не знают, зачем они пришли, просто пришли, и это только естественно.
Прежде всего, человек не знает, откуда он пришёл и куда он идёт, Сказать, что вы
рождены из чрева вашей матери и возвратитесь в землю – это чисто биологическое
объяснение, но никто, в сущности, не знает, что существует до рождения и какой мир ждёт
нас после смерти. Рождённый не зная для чего, только для того, чтобы закрыть глаза и
удалиться в бесконечное незнаемое – человек представляет собой действительно
трагическое создание.
В один из дней я нашёл циновку, оставленную группой пилигримов, которые посещали
храмы Шикоку. На ней были написаны слова:
«Изначально нет востока и запада. Десять бесконечных направлений».
И теперь, держа шляпу в руках, я снова спросил юношу, откуда он пришёл, и он сказал,
что он сын храмового священника из Каназаве. И поскольку это было глупо – весь день
читать для мёртвых священные тексты, он захотел стать фермером.
Нет ни запада, ни востока. Солнце встаёт на востоке, садится на западе, но это просто
астрономические наблюдения. Знать, что вы не понимаете ни востока, ни запада – будет
ближе к правде. Факт в том, что никто не знает, откуда приходит солнце.
Среди десятков тысяч священных текстов есть один самый любимый, в котором
расставлены все важнейшие точки, – это Главная сутра. Согласно этой сутре: «Господь Будда
провозгласил: форма – это пустота, пустота – это форма. Материя и дух – одно, но всё
пустота. Человек ни жив, ни мёртв, не рождён и не бессмертен, без старости и болезни, без
прибавления и без убавления».
В один из дней, когда мы убирали рис, я сказал юноше, который отдыхал около большой
копны соломы: «Я думал о том, что посаженные весной семена риса оживают, давая
молодые побеги, а теперь, когда ми его жнём, рис умирает. И тот факт, что этот ритуал
повторяется год за годом, означает, что жизнь на этом поле продолжается, и ежегодная
смерть – это ежегодное рождение. Вы могли бы сказать, что рис, который мы сейчас
убираем, живёт бесконечно».
Человеческие существа обычно видят жизнь и смерть в довольно короткой перспективе.
Какое значение для этой травы может иметь рождение весной и смерть осенью? Люди
думают, что жизнь – это веселье, а смерть – это печаль, но семена риса, высеянные в землю
и ожившие весной в виде молодых побегов, погибающих осенью, в самой своей сердцевине

несут полную радость жизни. Радость жизни не исчезает со смертью. Смерть – это не
больше, чем преходящее мгновение. Разве не можете вы сказать, что этот рис, обладающий
полной радостью жизни, не знает печали смерти?
Те же самые вещи, которые происходят с рисом и ячменём, непрестанно творятся в
человеческом теле. День за днём растут волосы и ногти, десятки тысяч клеток умирают,
десятки тысяч клеток рождаются, кровь в теле месяц тому назад – не та же самая кровь, что
сегодня. Если вы думаете, что ваши собственные черты повторяются в телах ваших детей и
внуков, вы могли бы сказать, что вы умираете и возрождаетесь каждый день и будете жить в
течение многих поколений после смерти.
Если участие в этом цикле может быть пережито и прочувствовано вами каждый день,
больше ничего не требуется. Но большинство людей неспособны наслаждаться
непрестанным течением и каждодневными изменениями жизни. Они привязаны к жизни и
эта привычная привязанность приносит страх смерти. Сосредоточив своё внимание только
на прошлом, которое уже ушло, или на будущем, которое ещё не пришло, они забывают, что
они живут на земле здесь и сейчас. В полном смятении они наблюдают, как их жизнь
проходит, словно сон.
«Если жизнь и смерть – реальность, неизбежны ли человеческие страдания?»
«Нет ни жизни, ни смерти».
«Как вы можете говорить это?»
Мир – это единство материи в потоке бытия, но человеческое сознание разделяет этот
феномен, создав такую двойственность как жизнь и смерть, инь и ян, бытие и пустота.
Сознание приходит к вере в абсолютную достоверность того, что воспринимают чувства, и
тогда материя превращается в объекты, которые существуют постольку, поскольку люди
воспринимают их.
Форма материального мира, концепция жизни и смерти, здоровья и болезни, радости и
печали – всё это берёт начало в человеческом сознании. В сутре, где Будда сказал, что всё
пустота, он не только отвергал внутреннюю реальность того, что создано человеческим
интеллектом, но он провозглашал также, что и человеческие эмоции – это тоже иллюзии.
«Вы всё считаете иллюзией? Ничего не остаётся?»
«Ничего не остаётся? Концепция «пустоты», очевидно, всё ещё остаётся в вашем
сознании», – сказал я юноше. – Если вы не знаете, откуда вы пришли или куда идёте, как же
вы можете быть уверены, что вы здесь стоите передо мной? Существование бессмысленно?»
«……?»
Однажды утром я услышал, как четырёхлетняя девочка спросила свою мать: «Зачем я
рождена в этот мир? Чтобы ходить в детский сад?» Конечно, её мать не могла честно сказать:
«Да, ты права, поэтому иди». И всё же вы могли бы сказать, что люди в наши дни рождаются
для того, чтобы ходить в детский сад.
Во время своего обучения в колледже люди настойчиво учатся, чтобы узнать, зачем они
рождены. Учёные и философы говорят, что они будут довольны, если поймут только одну эту
вещь, даже если они всю жизнь пожертвуют этим поискам.
Изначально люди не имели цели. Теперь, выдумывая себе ту или другую цель, они ведут
борьбу в поисках смысла жизни. Это похоже на спортивные соревнования одного человека с
самим собой. Не существует цели, о которой человек должен думать или идти на её поиски.
Вы бы хорошо сделали, если бы спросили детей, бессмысленна ли жизнь без цели.
С того времени, как они начинают ходить в детский сад, начинаются их страдания.

Человек был счастливым созданием, но он создал суровый мир и теперь борется, пытаясь
вырваться из него. В природе есть жизнь и смерть, и природа полна радости. В человеческом
обществе есть жизнь, и смерть, и люди живут в печали.
Плывущие облака и иллюзия науки
В это утро я мыл на реке ящики для хранения мандаринов. Когда я присел на плоский
камень, мои руки почувствовали прохладу осенней реки. Красные листья сумаха по берегам
реки красиво выделялись на фоне ясного голубого осеннего неба. Я был потрясён чудом
неожиданного великолепия ветвей на фоне неба.
В этой случайной сцене присутствовал весь мир жизненного опыта. В текущей воде –
поток времени, левый берег и правый берег, солнечный свет и тень, красные листья и
голубое небо – всё было вписано в священную молчаливую книгу природы. А человек –
тонкий мыслящий тростник.
Однажды он задаёт вопрос, что такое природа, затем он должен узнать, что такое это
«что» и что такое человек, который хочет узнать что такое «что». Он погружается с головой в
мир бесконечных вопросов. Пытаясь составить себе ясное представление о том, что
вызывает его интерес и наполняет изумлением, он имеет два возможных пути. Первый –
заглянуть в глубь самого себя, того, кто вопрошает «Что такое природа?» Второй – изучать
природу отдельно от человека.
Первый путь ведёт в мир философии и религии. Сидя на берегу реки и созерцая её, вы
можете сказать, что вода бежит под мостом, но не будет ничего нелепого, если вы скажете,
что вода стоит, а по ней плывёт мост.
Если следовать по второму пути, то природа разделяется на разнообразные феномены:
вода, скорость течения, волны, ветер и белые облака – все они по отдельности становятся
объектами наблюдения, ведущего к возникновению дальнейших вопросов, количество
которых распространяется бесконечно во всех направлениях. Это путь науки.
Мир обычно прост. Вы просто замечаете мимоходом, что вы промокли, собрав на себе
капли росы, когда проходили по лугу. Но с того времени, как люди взялись объяснять каждую
каплю росы научно, они попались в ловушку нескончаемого ада интеллекта.
Молекулы воды состоят из атомов водорода и кислорода. Люди думали, что самые
маленькие частички в мире – это атомы, но затем они обнаружили, что внутри атома есть
ядро. Теперь они открыли, что внутри ядра есть ещё более мелкие частицы. Среди этих
ядерных частиц есть сотни различных видов, и никто не знает, где закончится исследование
этого микромира.
Говорят, что бег электронов по орбите внутри атома с ультравысокой скоростью в
точности похож на полёт комет в галактике. Для физика-атомщика мир элементарных частиц
так же огромен как сама Вселенная. Но было доказано, что кроме той галактики, в которой
мы живём, существует бесчисленное множество других галактик. В глазах учёногокосмолога вся наша галактика становится бесконечно малой.
Факт заключается в том, что люди, которые думают, что капля воды проста, а камень
неподвижен и инертен – это счастливые невежественные глупцы, а учёные, которые знают,
что капля воды – это целая Вселенная и что камень – это активный мир элементарных
частиц, несущихся как ракеты – это умные глупцы. Простой взгляд говорит, что мир реален и
рядом. Если рассматривать мир во всей его сложности, он становится пугающе абстрактным
и отдалённым.
Учёные, которые радовались, когда с Луны были доставлены лунные камни, знают о ней

меньше, чем дети, поющие «Сколько тебе лет, мистер Луна?» Башо (знаменитый японский
поэт хайку, 1644–1694) мог ощутить чудо природы, наблюдая отражение полной Луны в
спокойной воде пруда. Всё, что учёные сделали, когда они выходили в космос и топали по
Луне в своих космических ботинках, это то, что они затмили частичку лунного блеска для
миллионов влюблённых и детей земли.
Как это получилось, что люди думают, будто наука – благодеяние для человечества?
Вначале в этой деревне зерно перемалывали в муку на каменной мельнице, которую
медленно крутили вручную. Затем была построена водяная мельница несравнимо более
мощная, чем старая каменная. Она использовала силу речного потока. Несколько лет назад
была возведена плотина для получения гидроэлектроэнергии и построена мельница,
использующая эту энергию.
Как вы думаете, эта механизация работает на благо людей? Чтобы размолоть рис в муку,
его сначала полируют, то есть делают рис белым. Это означает очищение зёрен, удаление
зародыша и отрубей, которые являются основой здорового питания человека, а в пищу идёт
то, что остаётся после этой обработки (в Японии название остатка после обработки риса,
произносится «касу», составлено из корней слов, означающих «белый» и «рис», название
отрубей, «нука», составлено из корней слов «рис» и «здоровье»). И таким образом,
результатом этой технологии является разрушение целого зерна и превращение его в
неполный субпродукт. Если слишком легко перевариваемый белый рис становится
ежедневным продуктом питания, то в такой диете не хватает питательных веществ, и их
необходимо добавлять в пищу. Водяное колесо и мукомольный завод выполняют работу
желудка и кишечника, и последствия такой замены должны сделать эти органы
малоактивными.
То же самое с горючим. Нефть образовалась, когда тела древних растений, погребённые
глубоко в земле, были трансформированы громадным давлением и жаром. Это вещество
выкапывают из-под земли, транспортируют по трубопроводу в порт и затем на кораблях
доставляют в Японию и получают из него керосин и масло.
Как вы думаете, что быстрее, теплее и удобнее – жечь этот керосин или ветви кедра или
сосны, растущих перед вашим домом? (в настоящее время в мире наблюдается недостаток
древесного топлива. В аргументах м-ра Фукуока предполагается необходимость сажать
деревья. В более широком смысле м-р Фукуока предлагает скромные, прямые пути
удовлетворения повседневных нужд) Топливо – тот же растительный материал. Нефть и
керосин проделали долгий путь, чтобы попасть сюда.
Теперь они говорят, что ископаемого топлива недостаточно и мы должны развивать
атомную энергетику. Найти редкую урановую руду, получить из неё радиоактивное топливо и
сжечь его в громадных ядерных печах гораздо труднее, чем поджечь сухие листья спичкой.
Кроме того, огонь в очаге оставляет только золу, а после сжигания ядерного топлива
остаются радиоактивные отходы, которые представляют опасность в течение тысяч лет.
Того же самого принципа придерживаются в сельском хозяйстве. Выращивайте нежные
большие растения на затопленных полях и вы получите растения, которые легко
повреждаются насекомыми и болезнями. Если используются «улучшенные» сорта, то
приходится надеяться на помощь химических удобрений и инсектицидов.
Но если вы выращиваете маленькие крепкие растения в здоровой среде, эти химикаты
становятся ненужными.
Взрыхлите затопленные рисовые поля плугом, и в почве образуется недостаток

кислорода, почвенная структура разрушится, дождевые черви и другие мелкие животные
погибнут, и земля станет твёрдой и безжизненной. Когда это случается, поле приходится
перепахивать каждый год.
Но если освоить метод, при котором земля сама себя рыхлит естественным путём, то
нет необходимости её пахать или рыхлить машинами. После того, как в живой почве
уничтожены органическое вещество и микроорганизмы, становится необходимым
использование быстродействующих удобрений. Если используются химические удобрения,
рис растёт быстро и сильно, но так же быстро растут сорняки. Тогда применяются
гербициды и считают, что это благотворно для посевов.
Но если вместе с зерновыми посеять клевер, и всю солому и органические остатки
вернуть в почву в качестве мульчи, то культуру можно выращивать без гербицидов,
химических удобрений и приготовленных компостов.
В земледелии совсем немного того, без чего нельзя обойтись. Химические удобрения,
гербициды, инсектициды, машины – всё это не нужно. Но если создаются условия, при
которых они становятся необходимыми, тогда требуется сила науки.
Я продемонстрировал на своих полях, что натуральное земледелие даёт урожай не
меньший, чем современное научное земледелие. Если результаты неактивного земледелия
сравнимы с результатами науки, при гораздо меньших затратах труда и ресурсов, в чем же
заключается польза научной технологии?

Теория относительности
Глядя на светлое сияние осеннего неба, осматривая окружающие поля, я был удивлён.
На всех полях, кроме моего, работали машины по уборке риса и комбайны. В последние три
года моя деревня изменилась до неузнаваемости.
Как и можно было ожидать, юноши, живущие на горе, не завидуют повороту к
механизации. Их радует спокойная мирная уборка урожая с помощью старого ручного серпа.
В тот вечер, когда мы заканчивали вечернюю трапезу, я рассказал за чаем, как много лет
назад в этой деревне в те дни, когда фермеры обрабатывали поля вручную, один человек
начал использовать для этого корову. Он был очень горд тем, как легко и быстро он мог
закончить тяжёлый труд вспашки почвы. Двадцать лет назад, когда появился первый
механический культиватор, все жители деревни собрались и серьёзно обсуждали, что
лучше – корова или машина. В течение двух-трёх лет стало ясно, что машина работает
быстрее, и, не заботясь ни о чём, кроме выигрыша во времени и удобства, фермеры
отказались от своих тягловых животных. Главным побуждением было просто желание
закончить работу быстрее, чем соседний фермер.
Фермер не понимает, что он стал просто фактором увеличения скорости и
эффективности в уравнении современного земледелия. Он позволил продавцам
сельскохозяйственного оборудования сделать за него все расчёты.
Когда-то люди смотрели в звёздное ночное небо и ощущали огромность Вселенной.
Теперь вопросы времени и космоса оставлены всецело в ведении учёных.
Говорят, что Эйнштейну дали Нобелевскую премию по физике из уважения к
непостижимости его теории относительности. Если бы его теория понятно объяснила
феномены относительности в мире и тем самым освободила бы человечество от
ограничений времени и пространства, вызвав к жизни более приятный и спокойный мир, это
было бы достойно похвалы. Однако его объяснение сбивает людей с толку и заставляет их
думать, что мир сложен сверх всякого понимания. Ему стоило бы присудить премию за
«разрушение покоя человеческого духа».
В действительности мир относительности не существует. Идея о феномене
относительности – это структура, созданная человеческим интеллектом. Другие животные
живут в мире неразделённой реальности. В той степени, в какой человек живёт в
относительном мире интеллекта, он теряет представление о времени, которое вне времени,
и о пространстве, которое вне пространства.
«Вы можете быть удивлены моей привычкой всё время поддевать учёных», – сказал я и
сделал паузу, чтобы отхлебнуть глоток чая. Юноши смотрели, улыбаясь, их лица блестели в
свете костра. «Это потому, что роль учёного в обществе аналогична роли различающего
понимания в вашем собственном сознании».

Деревня без войны и мира
Змея схватила лягушку и ускользнула в траву. Девушка закричала. Молодой парень, не
скрывая отвращения, швырнул в змею камень. Другой засмеялся. Я повернулся к тому,
который бросил камень: «Как ты думаешь, чего ты этим добьёшься?»
Ястреб охотится за змеёй. Волк нападает на ястреба. Человек убивает волка, а человека
сводит в могилу вирус туберкулёза. Бактерии размножаются в останках человека и других
животных, травы и деревья поглощают питательные вещества, образовавшиеся в результате
активности бактерий. Насекомые атакуют деревья, лягушки едят насекомых.
Животные, растения, микроорганизмы – все они части круговорота жизни. Поддерживая
постоянное равновесие, все они ведут естественно регулируемое существование. Люди
могут выбрать точку зрения на этот мир или как на модель мира, где сильный поедает
слабого, или как на мир, где процветает сосуществование и обоюдная польза. И та, и другая
точка зрения – это произвольная интерпретация, которая вызывает ветер и волны, вносит
беспорядок и смятение.
Взрослые думают, что лягушка заслуживает жалости и сочувствуя её смерти, презирают
змею. Эти чувства могут показаться естественными, само собой разумеющимися, но разве
они соответствуют тому, что происходит на самом деле?
Один юноша сказал: «Если видеть жизнь, как борьбу, в которой сильный пожирает
слабого, то лицо земли предстаёт как ад смертоубийства и насилия». Но это неизбежно, что
слабый должен быть принесён в жертву, чтобы сильный мог жить. То, что сильный
побеждает и выживает, а слабый умирает – это закон природы. После прошедших миллионов
лет творения, живущие теперь на земле, победили в борьбе за жизнь. Вы могли бы сказать,
что выживание наиболее приспособленных – это бережливость природы.
Второй юноша сказал: «Так это выглядит для победителя. Я смотрю на мир с точки
зрения сосуществования и взаимной пользы. В этом поле, под покровом зерновой культуры,
клевер и многочисленные виды трав и сорняков живут во взаимодействии и содружестве.
Плющ обвивается вокруг деревьев, лишайники живут на коре дерева и на его ветвях, Под
покровом лесной растительности располагаются мхи. Птицы и лягушки, растения и
насекомые, мелкие животные, бактерии, грибы – все создания играют свои важные роли и
выигрывают от существования друг друга».
Третий сказал: «Земля – это мир сильного, поедающего слабого, а также мир
сосуществования. Сильные создания имеют пищи не больше, чем им необходимо, и хотя они
нападают на другие существа, общее равновесие в природе постоянно поддерживается.
Бережливость природы – это её железное правило, поддерживающее мир и порядок на
земле».
Три человека и три точки зрения. Все три мнения я встретил решительным отрицанием.
Мир сам по себе никогда не спрашивает, основан ли он на принципе конкуренции или
кооперации. Если видеть его в относительной перспективе человеческого интеллекта, то в
мире есть те, кто силён и те, кто слаб, есть большие и маленькие.
Нет никого, кто сомневается в существовании такой относительной точки зрения, но
если бы мы предположили, что относительность человеческого восприятия ошибочна –
например, нет больших и маленьких, нет верха и низа, – если мы скажем, что такой точки
зрения вообще нет, человеческие ценности и суждения будут разрушены.

«Не является ли такой способ видения мира пустым полётом воображения? В
реальности есть большие страны и маленькие страны. Если есть бедность и довольство,
сила и слабость, неизбежно будут конфликты и, соответственно, победители и побеждённые.
Но можете ли вы сказать, что это относительное восприятие и возникающие вследствие
этого эмоции человечны и, значит, естественны, что они уникальные привилегии
человеческих существ?»
Другие животные сражаются, но не устраивают войны. Если вы скажете, что война,
которая вытекает из идеи сильного и слабого, это специально человеческая «привилегия»,
тогда жизнь – это фарс. В непонимании того, что это фарс, заключается трагедия
человечества.
Кто спокойно живёт в мире без противоречий и различий – это дети. Они
воспринимают свет и темноту, силу и слабость, но не делают выводов. Даже хотя
существуют змея и лягушка, ребёнок не имеет представления о сильном и слабом. В детях
есть изначальная радость жизни, но страх смерти должен со временем появиться.
Любовь и ненависть, которые отражаются в глазах взрослых, вначале не были двумя
различными вещами. Это было одно и то же, видимое как бы спереди и сзади. Любовь
давала пищу для ненависти. Если вы перевернёте монету любви, она станет ненавистью.
Только путём проникновения в абсолютный мир, в котором нет аспектов, возможно не
потеряться в двойственности феноменального мира.
Люди различают Я и Другие. В той степени, в какой существует эго, в той степени, в
которой есть Другие, люди не могут освободиться от любви и ненависти. Сердце, которое
любит злое эго, создаёт ненавистного врага. Для людей первый и главнейший враг – это их
«я», которое они так любят.
Люди выбирают нападение или защиту. Чтобы оправдать свою борьбу, они обвиняют
друг друга в разжигании конфликта. Это похоже на то, как человек хлопает в ладоши и затем
пытается выяснить, что производит шум – правая рука или левая. Во всех событиях нет ни
правильного, ни неправильного, ни хорошего, ни плохого. Все сознательные различия
возникают одновременно и все они ошибочны.
Строить крепости – неправильно с самого начала. Даже, если есть оправдание, что это
для защиты города, замок – это выражение личности правящего господина, и он простирает
своё насилие на окружающие местности. Говоря, что он боится нападения и что эти
укрепления для защиты города, он накапливает громадное количество оружия и закрывает
дверь на ключ.
Акт защиты – это уже нападение. Оружие для самозащиты всегда даёт повод для тех, кто
провоцирует войны. Бедствие войны происходит от преувеличения и усиления пустых
различий между я (другой), слабый (сильный), нападение (защита). Нет другого пути к миру,
как всем людям выйти из укреплённого замка относительного восприятия, уйти в поля и
луга и вернуться в лоно неактивной природы. Это значит – наточить серп вместо меча.
В давние времена фермеры были мирными людьми, но сейчас они спорят с Австралией
по поводу мяса, ссорятся с Россией из-за рыбы и зависят от американских поставок
пшеницы и сои.
У меня такое чувство, что мы в Японии живём в тени большого дерева, а нет места
более опасного во время грозы, чем под большим деревом. И не может быть ничего более
бессмысленного, чем искать защиту под «атомным зонтиком», который будет первой целью
во время следующей войны. Мы обрабатываем землю под тёмным зонтом. Я чувствую, что

приближается кризис и внутренний и внешний.
Перестаньте разделять всё на своё и чужое. Везде в мире фермеры в своей основе
одинаковы. Можно сказать, что ключ к миру лежит близко к земле.

Революция одной соломинки
Среди молодых людей, которые пришли в хижины на горе, есть такие нищие телом и
духом, кто лишился всякой надежды. Я только старый фермер, который горюет, что не может
снабдить их даже парой сандалий, но есть всё же одна вещь, которую я могу дать им.
Одна соломинка.
Я поднял одну соломинку перед хижиной и сказал: «С этой одной соломинки может
начаться революция».
«Когда надвигается гибель всего человечества, вы можете всё ещё надеяться спастись
соломой?» – спросил один юноша с горечью в голосе.
Эта соломинка кажется такой маленькой и большинство людей не знают, сколько она на
самом деле весит. Если люди узнают истинную цену этой соломинки, может произойти
революция, которая станет достаточно мощной, чтобы сдвинуть страну и мир.
Когда я был ребёнком, я знал человека, который жил около ущелья Инуйозо. Казалось,
что он только и делает, что нагружает древесный уголь на спины двух лошадей и везёт его
две мили или около того по дороге от вершины горы в порт Гунчу. И вот он стал богатым.
Если вы спросите, как ему это удалось, люди расскажут вам, что на своём пути из порта
домой он собирал по краю дороги навоз и разбросанную солому и отвозил всё это на своё
поле. Его любимое изречение было:
«Обращайся с каждой соломинкой как с ценностью и никогда не делай ни одного шага
без пользы». Это сделало его богатым человеком.
«Даже если вы сожжёте всю солому, я не думаю, что это могло бы дать хоть искру для
разжигания революции».
Лёгкий бриз шуршал в деревьях сада, солнце просвечивало через зелёные листья, я
начал рассказывать об использовании соломы при выращивании риса.
Прошло уже почти сорок лет с тех пор, как я понял, какую важную роль может играть
солома при выращивании риса и ячменя. В то время, проходя мимо старого рисового поля в
префектуре Кохи, которое в течение многих лет оставалось неиспользованным и
невозделанным, я увидел здоровые молодые побеги риса, пробивающиеся через путаницу
сорняков и соломы, которая накопилась на поверхности почвы. После многолетней работы я
начал пропагандировать совершенно новый метод выращивания риса и ячменя.
Считая, что это естественный и революционный способ земледелия, я писал о нём в
книгах и журналах, много раз говорил об этом по телевидению и радио. Кажется, очень
простая вещь. Но фермеры так утвердились в своих представления о том, как надо
использовать солому, что маловероятно, чтобы они легко приняли перемену. Разбрасывание
свежей соломы по полю может быть рискованно, поскольку пирикуляриоз риса и стеблевая
гниль – это болезни, источник которых всегда есть в рисовой соломе. В прошлом эти
болезни приносили большой вред, и это одна из главных причин, почему фермер всегда
превращает солому в компост прежде, чем вернуть её на поле. Много лет назад тщательное
распределение рисовой соломы по полю обычно практиковалось как мера борьбы с
пирикуляриозом риса, и было время, когда в Хоккайдо полное сжигание соломы было
предписано законом.
Стеблевой сверлильщик также проникает в солому для перезимовки. Для
предотвращения заражения этими насекомыми фермер обычно тщательно компостировал

солому в течение всей зимы, чтобы быть уверенным, что к следующей весне она полностью
разложилась. Вот почему японский фермер содержит свои поля такими чистыми и
аккуратными. Практическое знание повседневной жизни говорило, что если фермер оставил
солому лежать, где попало, он будет наказан небесами за свою нерадивость.
После многих лет экспериментирования даже научные эксперты подтверждают теперь
правоту моих выводов о том, что разбрасывание свежей соломы по полю за 6 месяцев до
посева совершенно безопасно. Это переворачивает все прежние представления. Но,
очевидно, пройдёт много времени, прежде, чем фермеры станут восприимчивы к идее
использования соломы таким образом.
Фермеры работали в течение столетий, стараясь увеличить производство компоста.
Министерство сельского хозяйства использовало стимулирующие выплаты для поощрения
производства компоста и устраивало ежегодные конкурсные выставки компостов. Фермеры
уверовали в компост, как в покровительствующее почве божество. Теперь снова развивается
движение за то, чтобы делать больше компоста, «лучшего компоста», с дождевыми червями
и «компост стартером». Нет причин ожидать лёгкого восприятия моего предложения, что не
надо готовить компост, что всё, что вы должны сделать – это разбросать свежесрезанную
солому по полю.
По дороге в Токио, глядя из окна скорого поезда, я видел преображение японской
сельской местности. Глядя на зимние поля, облик которых полностью изменился за 10 лет, я
чувствовал невыразимый гнев. Прежний пейзаж аккуратных полей зелёного ячменя,
китайской молочной вики, цветущего рапса – исчез. Вместо этого всюду громоздились
оставленные мокнуть под дождём неряшливые кучи полусожжённой соломы. То, что эта
солома выброшена, – доказательство разлада в современном земледелии. Пустота этих полей
свидетельствует о пустоте души фермера. Это вызывает сомнение в ответственности
государственных
лидеров
и
ясно
свидетельствует
об
отсутствии
мудрой
сельскохозяйственной политики.
Человек, который несколько лет назад говорил о «милосердном конце» озимых
зерновых, об их смерти «на обочине дороги» – что он думает теперь, когда видит пустые
поля? Глядя на пустые зимние поля Японии, я больше не мог молчать. С помощью этой
соломы я сам начну революцию!
Юноши, которые молча слушали, теперь громко рассмеялись. «Революция одного
человека! Завтра давайте возьмём большой мешок ячменя, риса и семян клевера и взвалим
его на плечи, как окунину-шино-микото (легендарный японский бог исцеления, который
путешествовал с большим мешком за плечами, разбрасывая из этого мешка хорошую судьбу)
и разбросаем семена по всем полям Токайдо»,
«Это не революция одного человека, – засмеялся я, – это революция одной соломинки!»
Выйдя из хижины на послеполуденное солнце, я задержался и посмотрел на деревья
моего сада, отягощённые спелыми плодами, и на кур, копающихся в зарослях сорняков и
клевера. Затем я начал спускаться к полям по знакомой дороге.


\end{document}

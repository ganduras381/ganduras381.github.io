\documentclass[a4paper]{article}
\usepackage{anysize}
\marginsize{2.0cm}{2.2cm}{0.0cm}{0.0cm}
\setlength{\footskip}{9pt}
\begin{document}
THE ENCHIRIDION

by Epictetus

Composed by Arrian

Translated by Thomas Wentworth Higginson

I. There are things which are within our power, and there are things which
are beyond our power. Within our power are opinion, aim, desire,
aversion, and, in one word, whatever affairs are our own. Beyond our
power are body, property, reputation, office, and, in one word, whatever
are not properly our own affairs.

Now the things within our power are by nature free, unrestricted,
unhindered; but those beyond our power are weak, dependent, restricted,
alien. Remember, then, that if you attribute freedom to things by nature
dependent and take what belongs to others for your own, you will be
hindered, you will lament, you will be disturbed, you will find fault
both with gods and men. But if you take for your own only that which is
your own and view what belongs to others just as it really is, then no
one will ever compel you, no one will restrict you; you will find fault
with no one, you will accuse no one, you will do nothing against your
will; no one will hurt you, you will not have an enemy, nor will you
suffer any harm.

Aiming, therefore, at such great things, remember that you must not allow
yourself any inclination, however slight, toward the attainment of the
others; but that you must entirely quit some of them, and for the present
postpone the rest. But if you would have these, and possess power and
wealth likewise, you may miss the latter in seeking the former; and you
will certainly fail of that by which alone happiness and freedom are
procured.

Seek at once, therefore, to be able to say to every unpleasing semblance,
“You are but a semblance and by no means the real thing.” And then
examine it by those rules which you have; and first and chiefly by this:
whether it concerns the things which are within our own power or those
which are not; and if it concerns anything beyond our power, be prepared
to say that it is nothing to you.

II. Remember that desire demands the attainment of that of which you are
desirous; and aversion demands the avoidance of that to which you are
averse; that he who fails of the object of his desires is disappointed;
and he who incurs the object of his aversion is wretched. If, then, you
shun only those undesirable things which you can control, you will never
incur anything which you shun; but if you shun sickness, or death, or
poverty, you will run the risk of wretchedness. Remove [the habit of]
aversion, then, from all things that are not within our power, and apply
it to things undesirable which are within our power. But for the present,
altogether restrain desire; for if you desire any of the things not
within our own power, you must necessarily be disappointed; and you are
not yet secure of those which are within our power, and so are legitimate
objects of desire. Where it is practically necessary for you to pursue or
avoid anything, do even this with discretion and gentleness and
moderation.

III. With regard to whatever objects either delight the mind or contribute to
use or are tenderly beloved, remind yourself of what nature they are,
beginning with the merest trifles: if you have a favorite cup, that it is
but a cup of which you are fond of—for thus, if it is broken, you can
bear it; if you embrace your child or your wife, that you embrace a
mortal—and thus, if either of them dies, you can bear it.


IV. When you set about any action, remind yourself of what nature the action
is. If you are going to bathe, represent to yourself the incidents usual
in the bath—some persons pouring out, others pushing in, others scolding,
others pilfering. And thus you will more safely go about this action if
you say to yourself, “I will now go to bathe and keep my own will in
harmony with nature.” And so with regard to every other action. For thus,
if any impediment arises in bathing, you will be able to say, “It was not
only to bathe that I desired, but to keep my will in harmony with nature;
and I shall not keep it thus if I am out of humor at things that happen.”


V. Men are disturbed not by things, but by the views which they take of
things. Thus death is nothing terrible, else it would have appeared so to
Socrates. But the terror consists in our notion of death, that it is
terrible. When, therefore, we are hindered or disturbed, or grieved, let
us never impute it to others, but to ourselves—that is, to our own views.
It is the action of an uninstructed person to reproach others for his own
misfortunes; of one entering upon instruction, to reproach himself; and
one perfectly instructed, to reproach neither others nor himself.

VI. Be not elated at any excellence not your own. If a horse should be
elated, and say, “I am handsome,” it might be endurable. But when you are
elated and say, “I have a handsome horse,” know that you are elated only
on the merit of the horse. What then is your own? The use of the
phenomena of existence. So that when you are in harmony with nature in
this respect, you will be elated with some reason; for you will be elated
at some good of your own.


VII. As in a voyage, when the ship is at anchor, if you go on shore to get
water, you may amuse yourself with picking up a shellfish or a truffle in
your way, but your thoughts ought to be bent toward the ship, and
perpetually attentive, lest the captain should call, and then you must
leave all these things, that you may not have to be carried on board the
vessel, bound like a sheep; thus likewise in life, if, instead of a
truffle or shellfish, such a thing as a wife or a child be granted you,
there is no objection; but if the captain calls, run to the ship, leave
all these things, and never look behind. But if you are old, never go far
from the ship, lest you should be missing when called for.


VIII. Demand not that events should happen as you wish; but wish them to happen
as they do happen, and you will go on well.


IX. Sickness is an impediment to the body, but not to the will unless itself
pleases. Lameness is an impediment to the leg, but not to the will; and
say this to yourself with regard to everything that happens. For you will
find it to be an impediment to something else, but not truly to yourself.


X. Upon every accident, remember to turn toward yourself and inquire what
faculty you have for its use. If you encounter a handsome person, you
will find continence the faculty needed; if pain, then fortitude; if
reviling, then patience. And when thus habituated, the phenomena of
existence will not overwhelm you.


XI. Never say of anything, “I have lost it,” but, “I have restored it.” Has
your child died? It is restored. Has your wife died? She is restored. Has
your estate been taken away? That likewise is restored. “But it was a bad
man who took it.” What is it to you by whose hands he who gave it has
demanded it again? While he permits you to possess it, hold it as
something not your own, as do travelers at an inn.


XII. If you would improve, lay aside such reasonings as these: “If I neglect
my affairs, I shall not have a maintenance; if I do not punish my
servant, he will be good for nothing.” For it were better to die of
hunger, exempt from grief and fear, than to live in affluence with
perturbation; and it is better that your servant should be bad than you
unhappy.

Begin therefore with little things. Is a little oil spilled or a little
wine stolen? Say to yourself, “This is the price paid for peace and
tranquillity; and nothing is to be had for nothing.” And when you call
your servant, consider that it is possible he may not come at your call;
or, if he does, that he may not do what you wish. But it is not at all
desirable for him, and very undesirable for you, that it should be in his
power to cause you any disturbance.


XIII. If you would improve, be content to be thought foolish and dull with
regard to externals. Do not desire to be thought to know anything; and
though you should appear to others to be somebody, distrust yourself. For
be assured, it is not easy at once to keep your will in harmony with
nature and to secure externals; but while you are absorbed in the one,
you must of necessity neglect the other.


XIV. If you wish your children and your wife and your friends to live forever,
you are foolish, for you wish things to be in your power which are not
so, and what belongs to others to be your own. So likewise, if you wish
your servant to be without fault, you are foolish, for you wish vice not
to be vice but something else. But if you wish not to be disappointed in
your desires, that is in your own power. Exercise, therefore, what is in
your power. A man’s master is he who is able to confer or remove whatever
that man seeks or shuns. Whoever then would be free, let him wish
nothing, let him decline nothing, which depends on others; else he must
necessarily be a slave.


XV. Remember that you must behave as at a banquet. Is anything brought round
to you? Put out your hand and take a moderate share. Does it pass by you?
Do not stop it. Is it not yet come? Do not yearn in desire toward it, but
wait till it reaches you. So with regard to children, wife, office,
riches; and you will some time or other be worthy to feast with the gods.
And if you do not so much as take the things which are set before you,
but are able even to forego them, then you will not only be worthy to
feast with the gods, but to rule with them also. For, by thus doing,
Diogenes and Heraclitus, and others like them, deservedly became divine,
and were so recognized.


XVI. When you see anyone weeping for grief, either that his son has gone
abroad or that he has suffered in his affairs, take care not to be
overcome by the apparent evil, but discriminate and be ready to say,
“What hurts this man is not this occurrence itself—for another man might
not be hurt by it—but the view he chooses to take of it.” As far as
conversation goes, however, do not disdain to accommodate yourself to him
and, if need be, to groan with him. Take heed, however, not to groan
inwardly, too.


XVII. Remember that you are an actor in a drama of such sort as the Author
chooses—if short, then in a short one; if long, then in a long one. If it
be his pleasure that you should enact a poor man, or a cripple, or a
ruler, or a private citizen, see that you act it well. For this is your
business—to act well the given part, but to choose it belongs to another.


XVIII. When a raven happens to croak unluckily, be not overcome by appearances,
but discriminate and say, “Nothing is portended to me, either to my
paltry body, or property, or reputation, or children, or wife. But to
me all portents are lucky if I will. For whatsoever happens, it belongs
to me to derive advantage therefrom.”


XIX. You can be unconquerable if you enter into no combat in which it is not
in your own power to conquer. When, therefore, you see anyone eminent in
honors or power, or in high esteem on any other account, take heed not to
be bewildered by appearances and to pronounce him happy; for if the
essence of good consists in things within our own power, there will be no
room for envy or emulation. But, for your part, do not desire to be a
general, or a senator, or a consul, but to be free; and the only way to
this is a disregard of things which lie not within our own power.


XX. Remember that it is not he who gives abuse or blows, who affronts, but
the view we take of these things as insulting. When, therefore, anyone
provokes you, be assured that it is your own opinion which provokes you.
Try, therefore, in the first place, not to be bewildered by appearances.
For if you once gain time and respite, you will more easily command
yourself.


XXI. Let death and exile, and all other things which appear terrible, be daily
before your eyes, but death chiefly; and you will never entertain an
abject thought, nor too eagerly covet anything.


XXII. If you have an earnest desire toward philosophy, prepare yourself from
the very first to have the multitude laugh and sneer, and say, “He is
returned to us a philosopher all at once”; and, “Whence this supercilious
look?” Now, for your part, do not have a supercilious look indeed, but
keep steadily to those things which appear best to you, as one appointed
by God to this particular station. For remember that, if you are
persistent, those very persons who at first ridiculed will afterwards
admire you. But if you are conquered by them, you will incur a double
ridicule.


XXIII. If you ever happen to turn your attention to externals, for the pleasure
of anyone, be assured that you have ruined your scheme of life. Be
content, then, in everything, with being a philosopher; and if you wish
to seem so likewise to anyone, appear so to yourself, and it will suffice
you.


XXIV. Let not such considerations as these distress you: “I shall live in
discredit and be nobody anywhere.” For if discredit be an evil, you can
no more be involved in evil through another than in baseness. Is it any
business of yours, then, to get power or to be admitted to an
entertainment? By no means. How then, after all, is this discredit? And
how it is true that you will be nobody anywhere when you ought to be
somebody in those things only which are within your own power, in which
you may be of the greatest consequence? “But my friends will be
unassisted.” What do you mean by “unassisted”? They will not have money
from you, nor will you make them Roman citizens. Who told you, then, that
these are among the things within our own power, and not rather the
affairs of others? And who can give to another the things which he
himself has not? “Well, but get them, then, that we too may have a
share.” If I can get them with the preservation of my own honor and
fidelity and self-respect, show me the way and I will get them; but if
you require me to lose my own proper good, that you may gain what is no
good, consider how unreasonable and foolish you are. Besides, which would
you rather have, a sum of money or a faithful and honorable friend?
Rather assist me, then, to gain this character than require me to do
those things by which I may lose it. Well, but my country, say you, as
far as depends upon me, will be unassisted. Here, again, what assistance
is this you mean? It will not have porticos nor baths of your providing?
And what signifies that? Why, neither does a smith provide it with shoes,
nor a shoemaker with arms. It is enough if everyone fully performs his
own proper business. And were you to supply it with another faithful and
honorable citizen, would not he be of use to it? Yes. Therefore neither
are you yourself useless to it. “What place, then,” say you, “shall I
hold in the state?” Whatever you can hold with the preservation of your
fidelity and honor. But if, by desiring to be useful to that, you lose
these, how can you serve your country when you have become faithless and
shameless?


XXV. Is anyone preferred before you at an entertainment, or in courtesies, or
in confidential intercourse? If these things are good, you ought to
rejoice that he has them; and if they are evil, do not be grieved that
you have them not. And remember that you cannot be permitted to rival
others in externals without using the same means to obtain them. For how
can he who will not haunt the door of any man, will not attend him, will
not praise him, have an equal share with him who does these things? You
are unjust, then, and unreasonable if you are unwilling to pay the price
for which these things are sold, and would have them for nothing. For how
much are lettuces sold? An obulus, for instance. If another, then, paying
an obulus, takes the lettuces, and you, not paying it, go without them,
do not imagine that he has gained any advantage over you. For as he has
the lettuces, so you have the obulus which you did not give. So, in the
present case, you have not been invited to such a person’s entertainment
because you have not paid him the price for which a supper is sold. It is
sold for praise; it is sold for attendance. Give him, then, the value if
it be for your advantage. But if you would at the same time not pay the
one, and yet receive the other, you are unreasonable and foolish. Have
you nothing, then, in place of the supper? Yes, indeed, you have—not to
praise him whom you do not like to praise; not to bear the insolence of
his lackeys.


XX.
The will of nature may be learned from things upon which we are all
agreed. As when our neighbor’s boy has broken a cup, or the like, we are
ready at once to say, “These are casualties that will happen”; be
assured, then, that when your own cup is likewise broken, you ought to be
affected just as when another’s cup was broken. Now apply this to greater
things. Is the child or wife of another dead? There is no one who would
not say, “This is an accident of mortality.” But if anyone’s own child
happens to die, it is immediately, “Alas! how wretched am I!” It should
be always remembered how we are affected on hearing the same thing
concerning others.


XXVII. As a mark[1] is not set up for the sake of missing the aim, so neither
does the nature of evil exist in the world.


XXVIII. If a person had delivered up your body to some passer-by, you would
certainly be angry. And do you feel no shame in delivering up your own
mind to any reviler, to be disconcerted and confounded?


XXIX[2]. In every affair consider what precedes and what follows, and then
undertake it. Otherwise you will begin with spirit, indeed, careless of
the consequences, and when these are developed, you will shamefully
desist. “I would conquer at the Olympic Games.” But consider what
precedes and what follows, and then, if it be for your advantage, engage
in the affair. You must conform to rules, submit to a diet, refrain from
dainties; exercise your body, whether you choose it or not, at a stated
hour, in heat and cold; you must drink no cold water, and sometimes no
wine—in a word, you must give yourself up to your trainer as to a
physician. Then, in the combat, you may be thrown into a ditch, dislocate
your arm, turn your ankle, swallow an abundance of dust, receive stripes
[for negligence], and, after all, lose the victory. When you have
reckoned up all this, if your inclination still holds, set about the
combat. Otherwise, take notice, you will behave like children who
sometimes play wrestlers, sometimes gladiators, sometimes blow a trumpet,
and sometimes act a tragedy, when they happen to have seen and admired
these shows. Thus you too will be at one time a wrestler, and another a
gladiator; now a philosopher, now an orator; but nothing in earnest. Like
an ape you mimic all you see, and one thing after another is sure to
please you, but is out of favor as soon as it becomes familiar. For you
have never entered upon anything considerately; nor after having surveyed
and tested the whole matter, but carelessly, and with a halfway zeal.
Thus some, when they have seen a philosopher and heard a man speaking
like Euphrates[3]—though, indeed, who can speak like him?—have a mind to
be philosophers, too. Consider first, man, what the matter is, and what
your own nature is able to bear. If you would be a wrestler, consider
your shoulders, your back, your thighs; for different persons are made
for different things. Do you think that you can act as you do and be a
philosopher, that you can eat, drink, be angry, be discontented, as you
are now? You must watch, you must labor, you must get the better of
certain appetites, must quit your acquaintances, be despised by your
servant, be laughed at by those you meet; come off worse than others in
everything—in offices, in honors, before tribunals. When you have fully
considered all these things, approach, if you please—that is, if, by
parting with them, you have a mind to purchase serenity, freedom, and
tranquillity. If not, do not come hither; do not, like children, be now a
philosopher, then a publican, then an orator, and then one of Caesar’s
officers. These things are not consistent. You must be one man, either
good or bad. You must cultivate either your own reason or else externals;
apply yourself either to things within or without you—that is, be either
a philosopher or one of the mob.


XXX. Duties are universally measured by relations. Is a certain man your
father? In this are implied taking care of him, submitting to him in all
things, patiently receiving his reproaches, his correction. But he is a
bad father. Is your natural tie, then, to a good father? No, but to a
father. Is a brother unjust? Well, preserve your own just relation toward
him. Consider not what he does, but what you are to do to keep your
own will in a state conformable to nature, for another cannot hurt you
unless you please. You will then be hurt when you consent to be hurt. In
this manner, therefore, if you accustom yourself to contemplate the
relations of neighbor, citizen, commander, you can deduce from each the
corresponding duties.


XXXI. Be assured that the essence of piety toward the gods lies in this—to form
right opinions concerning them, as existing and as governing the universe
justly and well. And fix yourself in this resolution, to obey them, and
yield to them, and willingly follow them amidst all events, as being
ruled by the most perfect wisdom. For thus you will never find fault with
the gods, nor accuse them of neglecting you. And it is not possible for
this to be affected in any other way than by withdrawing yourself from
things which are not within our own power, and by making good or evil to
consist only in those which are. For if you suppose any other things to
be either good or evil, it is inevitable that, when you are disappointed
of what you wish or incur what you would avoid, you should reproach and
blame their authors. For every creature is naturally formed to flee and
abhor things that appear hurtful and that which causes them; and to
pursue and admire those which appear beneficial and that which causes
them. It is impracticable, then, that one who supposes himself to be hurt
should rejoice in the person who, as he thinks, hurts him, just as it is
impossible to rejoice in the hurt itself. Hence, also, a father is
reviled by his son when he does not impart the things which seem to be
good; and this made Polynices and Eteocles[4] mutually enemies—that
empire seemed good to both. On this account the husbandman reviles the
gods; [and so do] the sailor, the merchant, or those who have lost wife
or child. For where our interest is, there, too, is piety directed. So
that whoever is careful to regulate his desires and aversions as he ought
is thus made careful of piety likewise. But it also becomes incumbent on
everyone to offer libations and sacrifices and first fruits, according to
the customs of his country, purely, and not heedlessly nor negligently;
not avariciously, nor yet extravagantly.


XXXII. When you have recourse to divination, remember that you know not what the
event will be, and you come to learn it of the diviner; but of what
nature it is you knew before coming; at least, if you are of philosophic
mind. For if it is among the things not within our own power, it can by
no means be either good or evil. Do not, therefore, bring with you to the
diviner either desire or aversion—else you will approach him
trembling—but first clearly understand that every event is indifferent
and nothing to you, of whatever sort it may be; for it will be in your
power to make a right use of it, and this no one can hinder. Then come
with confidence to the gods as your counselors; and afterwards, when any
counsel is given you, remember what counselors you have assumed, and
whose advice you will neglect if you disobey. Come to divination as
Socrates prescribed, in cases of which the whole consideration relates to
the event, and in which no opportunities are afforded by reason or any
other art to discover the matter in view. When, therefore, it is our duty
to share the danger of a friend or of our country, we ought not to
consult the oracle as to whether we shall share it with them or not. For
though the diviner should forewarn you that the auspices are unfavorable,
this means no more than that either death or mutilation or exile is
portended. But we have reason within us; and it directs us, even with
these hazards, to stand by our friend and our country. Attend, therefore,
to the greater diviner, the Pythian God, who once cast out of the temple
him who neglected to save his friend.[5]


XXXIII. Begin by prescribing to yourself some character and demeanor, such as you
may preserve both alone and in company.

Be mostly silent, or speak merely what is needful, and in few words. We
may, however, enter sparingly into discourse sometimes, when occasion
calls for it; but let it not run on any of the common subjects, as
gladiators, or horse races, or athletic champions, or food, or drink—the
vulgar topics of conversation—and especially not on men, so as either to
blame, or praise, or make comparisons. If you are able, then, by your own
conversation, bring over that of your company to proper subjects; but if
you happen to find yourself among strangers, be silent.

Let not your laughter be loud, frequent, or abundant.

Avoid taking oaths, if possible, altogether; at any rate, so far as you
are able.

Avoid public and vulgar entertainments; but if ever an occasion calls you
to them, keep your attention upon the stretch, that you may not
imperceptibly slide into vulgarity. For be assured that if a person be
ever so pure himself, yet, if his companion be corrupted, he who
converses with him will be corrupted likewise.

Provide things relating to the body no further than absolute need
requires, as meat, drink, clothing, house, retinue. But cut off
everything that looks toward show and luxury.

Before marriage guard yourself with all your ability from unlawful
intercourse with women; yet be not uncharitable or severe to those who
are led into this, nor boast frequently that you yourself do otherwise.

If anyone tells you that a certain person speaks ill of you, do not make
excuses about what is said of you, but answer: “He was ignorant of my
other faults, else he would not have mentioned these alone.”

It is not necessary for you to appear often at public spectacles; but if
ever there is a proper occasion for you to be there, do not appear more
solicitous for any other than for yourself—that is, wish things to be
only just as they are, and only the best man to win; for thus nothing
will go against you. But abstain entirely from acclamations and derision
and violent emotions. And when you come away, do not discourse a great
deal on what has passed and what contributes nothing to your own
amendment. For it would appear by such discourse that you were dazzled by
the show.

Be not prompt or ready to attend private recitations; but if you do
attend, preserve your gravity and dignity, and yet avoid making yourself
disagreeable.

When you are going to confer with anyone, and especially with one who
seems your superior, represent to yourself how Socrates or Zeno[6] would
behave in such a case, and you will not be at a loss to meet properly
whatever may occur.

When you are going before anyone in power, fancy to yourself that you may
not find him at home, that you may be shut out, that the doors may not be
opened to you, that he may not notice you. If, with all this, it be your
duty to go, bear what happens and never say to yourself, “It was not
worth so much”; for this is vulgar, and like a man bewildered by
externals.

In company, avoid a frequent and excessive mention of your own actions
and dangers. For however agreeable it may be to yourself to allude to the
risks you have run, it is not equally agreeable to others to hear your
adventures. Avoid likewise an endeavor to excite laughter, for this may
readily slide you into vulgarity, and, besides, may be apt to lower you
in the esteem of your acquaintance. Approaches to indecent discourse are
likewise dangerous. Therefore, when anything of this sort happens, use
the first fit opportunity to rebuke him who makes advances that way, or,
at least, by silence and blushing and a serious look show yourself to be
displeased by such talk.


XXXIV. If you are dazzled by the semblance of any promised pleasure, guard
yourself against being bewildered by it; but let the affair wait your
leisure, and procure yourself some delay. Then bring to your mind both
points of time—that in which you shall enjoy the pleasure, and that in
which you will repent and reproach yourself, after you have enjoyed
it—and set before you, in opposition to these, how you will rejoice and
applaud yourself if you abstain. And even though it should appear to you
a seasonable gratification, take heed that its enticements and
allurements and seductions may not subdue you, but set in opposition to
this how much better it is to be conscious of having gained so great a
victory.


XXXV. When you do anything from a clear judgment that it ought to be done,
never shrink from being seen to do it, even though the world should
misunderstand it; for if you are not acting rightly, shun the action
itself; if you are, why fear those who wrongly censure you?


XXXVI. As the proposition, “either it is day or it is night,” has much force in
a disjunctive argument, but none at all in a conjunctive one, so, at a
feast, to choose the largest share is very suitable to the bodily
appetite, but utterly inconsistent with the social spirit of the
entertainment. Remember, then, when you eat with another, not only the
value to the body of those things which are set before you, but also the
value of proper courtesy toward your host.


XXXVII. If you have assumed any character beyond your strength, you have both
demeaned yourself ill in that and quitted one which you might have
supported.


XXXVIII. As in walking you take care not to tread upon a nail, or turn your foot,
so likewise take care not to hurt the ruling faculty of your mind. And if
we were to guard against this in every action, we should enter upon
action more safely.


XXXIX. The body is to everyone the proper measure of its possessions, as the
foot is of the shoe. If, therefore, you stop at this, you will keep the
measure; but if you move beyond it, you must necessarily be carried
forward, as down a precipice; as in the case of a shoe, if you go beyond
its fitness to the foot, it comes first to be gilded, then purple, and
then studded with jewels. For to that which once exceeds the fit measure
there is no bound.


XL. Women from fourteen years old are flattered by men with the title of
mistresses. Therefore, perceiving that they are regarded only as
qualified to give men pleasure, they begin to adorn themselves, and in
that to place all their hopes. It is worth while, therefore, to try that
they may perceive themselves honored only so far as they appear beautiful
in their demeanor and modestly virtuous.


XLI. It is a mark of want of intellect to spend much time in things relating
to the body, as to be immoderate in exercises, in eating and drinking,
and in the discharge of other animal functions. These things should be
done incidentally and our main strength be applied to our reason.


XLII. When any person does ill by you, or speaks ill of you, remember that he
acts or speaks from an impression that it is right for him to do so. Now
it is not possible that he should follow what appears right to you, but
only what appears so to himself. Therefore, if he judges from false
appearances, he is the person hurt, since he, too, is the person
deceived. For if anyone takes a true proposition to be false, the
proposition is not hurt, but only the man is deceived. Setting out, then,
from these principles, you will meekly bear with a person who reviles
you, for you will say upon every occasion, “It seemed so to him.”


XLIII. Everything has two handles: one by which it may be borne, another by
which it cannot. If your brother acts unjustly, do not lay hold on the
affair by the handle of his injustice, for by that it cannot be borne,
but rather by the opposite—that he is your brother, that he was brought
up with you; and thus you will lay hold on it as it is to be borne.


XLIV. These reasonings have no logical connection: “I am richer than you,
therefore I am your superior.” “I am more eloquent than you, therefore I
am your superior.” The true logical connection is rather this: “I am
richer than you, therefore my possessions must exceed yours.” “I am more
eloquent than you, therefore my style must surpass yours.” But you, after
all, consist neither in property nor in style.


XLV. Does anyone bathe hastily? Do not say that he does it ill, but hastily.
Does anyone drink much wine? Do not say that he does ill, but that he
drinks a great deal. For unless you perfectly understand his motives, how
should you know if he acts ill? Thus you will not risk yielding to any
appearances but such as you fully comprehend.


XLVI. Never proclaim yourself a philosopher, nor make much talk among the
ignorant about your principles, but show them by actions. Thus, at an
entertainment, do not discourse how people ought to eat, but eat as you
ought. For remember that thus Socrates also universally avoided all
ostentation. And when persons came to him and desired to be introduced by
him to philosophers, he took them and introduced them; so well did he
bear being overlooked. So if ever there should be among the ignorant any
discussion of principles, be for the most part silent. For there is great
danger in hastily throwing out what is undigested. And if anyone tells
you that you know nothing, and you are not nettled at it, then you may be
sure that you have really entered on your work. For sheep do not hastily
throw up the grass to show the shepherds how much they have eaten, but,
inwardly digesting their food, they produce it outwardly in wool and
milk. Thus, therefore, do you not make an exhibition before the ignorant
of your principles, but of the actions to which their digestion gives
rise.


XLVII. When you have learned to nourish your body frugally, do not pique
yourself upon it; nor, if you drink water, be saying upon every occasion,
“I drink water.” But first consider how much more frugal are the poor
than we, and how much more patient of hardship. If at any time you would
inure yourself by exercise to labor and privation, for your own sake and
not for the public, do not attempt great feats; but when you are
violently thirsty, just rinse your mouth with water, and tell nobody.


XLVIII. The condition and characteristic of a vulgar person is that he never
looks for either help or harm from himself, but only from externals. The
condition and characteristic of a philosopher is that he looks to himself
for all help or harm. The marks of a proficient are that he censures no
one, praises no one, blames no one, accuses no one; says nothing
concerning himself as being anybody or knowing anything. When he is in
any instance hindered or restrained, he accuses himself; and if he is
praised, he smiles to himself at the person who praises him; and if he is
censured, he makes no defense. But he goes about with the caution of a
convalescent, careful of interference with anything that is doing well
but not yet quite secure. He restrains desire; he transfers his aversion
to those things only which thwart the proper use of our own will; he
employs his energies moderately in all directions; if he appears stupid
or ignorant, he does not care; and, in a word, he keeps watch over
himself as over an enemy and one in ambush.


XLIX. When anyone shows himself vain on being able to understand and interpret
the works of Chrysippus,[7] say to yourself: “Unless Chrysippus had
written obscurely, this person would have had nothing to be vain of. But
what do I desire? To understand nature, and follow her. I ask, then, who
interprets her; and hearing that Chrysippus does, I have recourse to him.
I do not understand his writings. I seek, therefore, one to interpret
them.” So far there is nothing to value myself upon. And when I find an
interpreter, what remains is to make use of his instructions. This alone
is the valuable thing. But if I admire merely the interpretation, what do
I become more than a grammarian, instead of a philosopher, except,
indeed, that instead of Homer I interpret Chrysippus? When anyone,
therefore, desires me to read Chrysippus to him, I rather blush when I
cannot exhibit actions that are harmonious and consonant with his
discourse.


L. Whatever rules you have adopted, abide by them as laws, and as if you
would be impious to transgress them; and do not regard what anyone says
of you, for this, after all, is no concern of yours. How long, then, will
you delay to demand of yourself the noblest improvements, and in no
instance to transgress the judgments of reason? You have received the
philosophic principles with which you ought to be conversant; and you
have been conversant with them. For what other master, then, do you wait
as an excuse for this delay in self-reformation? You are no longer a boy
but a grown man. If, therefore, you will be negligent and slothful, and
always add procrastination to procrastination, purpose to purpose, and
fix day after day in which you will attend to yourself, you will
insensibly continue to accomplish nothing and, living and dying, remain
of vulgar mind. This instant, then, think yourself worthy of living as a
man grown up and a proficient. Let whatever appears to be the best be to
you an inviolable law. And if any instance of pain or pleasure, glory or
disgrace, be set before you, remember that now is the combat, now the
Olympiad comes on, nor can it be put off; and that by one failure and
defeat honor may be lost or—won. Thus Socrates became perfect, improving
himself by everything, following reason alone. And though you are not yet
a Socrates, you ought, however, to live as one seeking to be a Socrates.


LI. The first and most necessary topic in philosophy is the practical
application of principles, as, We ought not to lie; the second is that
of demonstrations as, Why it is that we ought not to lie; the third,
that which gives strength and logical connection to the other two, as,
Why this is a demonstration. For what is demonstration? What is a
consequence? What a contradiction? What truth? What falsehood? The third
point is then necessary on account of the second; and the second on
account of the first. But the most necessary, and that whereon we ought
to rest, is the first. But we do just the contrary. For we spend all our
time on the third point and employ all our diligence about that, and
entirely neglect the first. Therefore, at the same time that we lie, we
are very ready to show how it is demonstrated that lying is wrong.

Upon all occasions we ought to have these maxims ready at hand:

  Conduct me, Zeus, and thou, O Destiny,
  Wherever your decrees have fixed my lot.
  I follow cheerfully; and, did I not,
  Wicked and wretched, I must follow still.[8]

  Who’er yields properly to Fate is deemed
  Wise among men, and knows the laws of Heaven.[9]

And this third:

  “O Crito, if it thus pleases the gods, thus let it be.”[10]

  “Anytus and Melitus may kill me indeed; but hurt me they cannot.”[11]




                               Footnotes


[1]Happiness, the effect of virtue, is the mark which God has set up for
   us to aim at. Our missing it is no work of His; nor so properly
   anything real, as a mere negative and failure of our own.

[2][Chapter XV of the third book of the Discourses, which, with the
   exception of some very trifling differences, is the same as chapter
   XXIX of the Enchiridion.—Ed.]

[3]Euphrates was a philosopher of Syria, whose character is described,
   with the highest encomiums, by Pliny the Younger, Letters I. 10.

[4][The two inimical sons of Oedipus, who killed each other in
   battle.—Ed.]

[5][This refers to an anecdote given in full by Simplicius, in his
   commentary on this passage, of a man assaulted and killed on his way
   to consult the oracle, while his companion, deserting him, took refuge
   in the temple till cast out by the Deity.—Tr.]

[6][Reference is to Zeno of Cyprus (335-263 B.C.), the founder of the
   Stoic school.—Ed.]

[7][Chrysippus (c. 280-207 B.C.) was a Stoic philosopher who became
   head of the Stoa after Cleanthes. His works, which are lost, were most
   influential and were generally accepted as the authoritative
   interpretation of orthodox Stoic philosophy.—Ed.]

[8]Cleanthes, in Diogenes Laertius, quoted also by Seneca, Epistle 107.

[9]Euripides, Fragments.

[10]Plato, Crito, Chap. XVII.

[11]Plato, Apology, Chap. XVIII.
\end{document}
